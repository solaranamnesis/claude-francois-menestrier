\documentclass[a4paper, 11pt, oneside, polutonikogreek, latin]{article}
\usepackage[T1]{fontenc}
\usepackage{aurical}

% Load encoding definitions (after font package)

\usepackage{textalpha}

\usepackage{listings}
\lstset{basicstyle=\ttfamily}

% Babel package:
\usepackage[latin]{babel}

% With XeTeX$\$LuaTeX, load fontspec after babel to use Unicode
% fonts for Latin script and LGR for Greek:
\ifdefined\luatexversion \usepackage{fontspec}\fi
\ifdefined\XeTeXrevision \usepackage{fontspec}\fi

% "Lipsiakos" italic font `cbleipzig`:
\newcommand*{\lishape}{\fontencoding{LGR}\fontfamily{cmr}%
		 \fontshape{li}\selectfont}
\DeclareTextFontCommand{\textli}{\lishape}

\usepackage{sectsty}
\usepackage[titles]{tocloft}

\allsectionsfont{\Fontauri}
\sectionfont{\Fontauri\Huge}
\subsectionfont{\Fontauri\LARGE}
\subsubsectionfont{\Fontauri\Large}

\usepackage{booktabs}
\setlength{\emergencystretch}{15pt}
\usepackage{fancyhdr}
\usepackage{microtype}
\usepackage{graphicx}
\setlength{\emergencystretch}{15pt}
\graphicspath{ {./ } }
\usepackage[figurename=]{caption}
\usepackage{float}
\begin{document}
\Fontauri
\renewcommand{\cftfigfont}{\Fontauri}
\renewcommand{\cftfigpagefont}{\Fontauri}

\renewcommand{\cftsecfont}{\Fontauri}
\renewcommand{\cftsubsecfont}{\Fontauri}
\renewcommand{\cftsubsubsecfont}{\Fontauri}
% fix toc page numbers
\let\origcftsecfont\cft
\let\origcftsecpagefont\cftsecpagefont
\let\origcftsecafterpnum\cftsecafterpnum
\renewcommand{\cftsecpagefont}{\Fontauri{\origcftsecpagefont}}
\renewcommand{\cftsecafterpnum}{\Fontauri{\origcftsecafterpnum}}
\let\origcftsubsecpagefont\cftsubsecpagefont
\let\origcftsubsecafterpnum\cftsubsecafterpnum
\renewcommand{\cftsubsecpagefont}{\Fontauri{\origcftsubsecpagefont}}
\renewcommand{\cftsubsecafterpnum}{\Fontauri{\origcftsubsecafterpnum}}
\let\origcftsubsubsecpagefont\cftsubsubsecpagefont
\let\origcftsubsubsecafterpnum\cftsubsubsecafterpnum
\renewcommand{\cftsubsubsecpagefont}{\Fontauri{\origcftsubsubsecpagefont}}
\renewcommand{\cftsubsubsecafterpnum}{\Fontauri{\origcftsubsubsecafterpnum}}

\renewcommand{\thefigure}{\Fontauri{\arabic{figure}}}
\renewcommand\thefootnote{\Fontauri{\arabic{footnote}}}
\let\oldfootnote\footnote
    \renewcommand{\footnote}[1]{\oldfootnote{\Fontauri\large#1}}
\begin{titlepage} % Suppresses headers and footers on the title page
	\centering % Centre everything on the title page
	%\scshape % Use small caps for all text on the title page

	%------------------------------------------------
	%	Title
	%------------------------------------------------
	
	\rule{\textwidth}{1.6pt}\vspace*{-\baselineskip}\vspace*{2pt} % Thick horizontal rule
	\rule{\textwidth}{0.4pt} % Thin horizontal rule
	
	\vspace{1\baselineskip} % Whitespace above the title
	
	{\scshape\Huge Symbolica Dianæ Ephesiæ Statua.}
	
	\vspace{1\baselineskip} % Whitespace above the title

	\rule{\textwidth}{0.4pt}\vspace*{-\baselineskip}\vspace{3.2pt} % Thin horizontal rule
	\rule{\textwidth}{1.6pt} % Thick horizontal rule
	
	\vspace{1\baselineskip} % Whitespace after the title block
	
	%------------------------------------------------
	%	Subtitle
	%------------------------------------------------
	
	{\scshape \Large A Claudio Menetreio\\Ceimeliothecæ Barberinæ Præfecto Exposita.} % Subtitle or further description
	
	\vspace*{1\baselineskip} % Whitespace under the subtitle
	
{\scshape Cui accessere,\\Lucæ Holstenij Epistola ad Franciscum Carinalem Barerinum De Fulcris, seu Verubus Dianæ Ephesiæ simulacro appositis,\\Jo. Petri Bellorij Notæ in Numismata tùm Ephesia, tùm uliarum Urbium Apibus insignita.\\\vspace*{10mm}\small Editio altera auctior\\Et ab eodem pluribus quàm antea Nummis, et antiquis Monumentis illustrata.} % Subtitle or further description

	%------------------------------------------------
	%	Editor(s)
	%------------------------------------------------
\vspace*{\fill}

	\vspace{1\baselineskip}

	{\small\scshape Romæ, 1688.}
	
	{\small\scshape{Apud Jo. Jacobum de Rubeis ad Templum S. Mariæ de Pace, suis sumptibus, et cura, cum Priuilegio Summi Pontificis.}}
	
	\vspace{0.5\baselineskip} % Whitespace after the title block

\scshape Internet Archive Online Edition% Publication year
	
	{\scshape\small Attribution-NonCommercial-ShareAlike 4.0 International} % Publisher
\end{titlepage}
\setlength{\parskip}{1mm plus1mm minus1mm}
\pagestyle{fancy}
\fancyhf{}
\cfoot{\Fontauri{\thepage}}
\Large
\clearpage
\tableofcontents
\clearpage
\begin{center}
{\scshape\textbf{Eminentiss. Ac Reverndiss.\\Principi Francicso Barberino Cardinali.\\Federicus Ubaldinus.}}
\end{center}
\paragraph{}
Non uereor, Domine, ut hilari vultu excipiatur à te quem tibi offero libellum: tuus enim est iure patrocinij; nam Clariss. Menetreius eum reliquit posthumum ingenij sui partum. Qua ergo pietate auctorem olim fouebas, et hunc fouebis certo scio; cum Claudium ipsum, quem tu plurimùm amabas, ab obliuione in hominum memoriam sis euocaturus. Etenim si præclarè consultum, est, ut eius, qui sacrilega gloriæ cupidine actus Dianæ Ephesiæ templum incendit, nomen aboleretur: merito illius per universum terrarum orbem celebrabitur industria, qui pio aduersus omnem Antiquitatem studio eandem Dianam instaurare contendit.
\clearpage
\section{Symbolica Dianæ Ephesiæ Statua.}
\subsection{Exposita.}
\paragraph{}
Aegyptiorum genti antiquissimæ, externæque eruditionis contemptrici moris fuit, ut quæ in superiori, inferiorique recondita erant natura, ea solis Sacerdotibus primæque notæ viris reuelaret. Indignum scilicet existimabat ea profanis et imperitis hominibus communicari, ne in vulgus edita, arcanorum religiosa maiestas euilesceret. Ipsi itaq; Hierophantæ, rerum naturam hieroglyphicis notis adumbrabant, ipsarumque adeò rerum effigies pro Dijs consecrabant: et Mercurio quidem Hermas, Soli literatos Obeliscos, Naturæ principis et mundum administrantis imagines, offerebant. Eius disciplinæ sectatorem fuisse conijcio primum Ephesinæ Dianæ vel inuentorem, vel sculptorem: certè tot illi adaptauit Symbola; ut arcanum et grande aliquid sub ijs intelligi voluerit. Ego in illis communis rerum parentis, et conseruatricis Naturæ typum agnouerim, quod et omnis Symbolorum series ostendit, et mellifluæ potissimum Apes indicant: quas benigno salutarique eius numini frequenter adscripsit Antiquitas. Verùm ut commodius hæc à me Sparta illustretur, quatuor Ephesini simulacri apographa proponam, quæ Romæ visuntur. Priora quidem duo expressa sunt ex marmoreis signis longè elegantissimis; quæ in celebri reconditæ vetustatis penu extant apud Eminentissimos Franciscum et Antonium Cardinales Barberinos, principes non minus huiusmodi elegantiarum amore et studiorum patrocinio, quam sanguine et dignitate, coniunctos. Tertium in ædibus Scipionis Lancellotti, Lauri Marchionis: ultimum apud Vincentium Justinianum Bassani Marchionem asseruatur.

Portentosum planè atque ab omni veterum Deorum cultu, et ornatu alienum hoc simulacrum videtur. Sed qui penitùs omnia eius symbola examinauerit, varia sub ijs mysteria reconditæ illius priscorum, Aegyptiorum sapientiæ latere deprehenderet: quin etiam sub istis diuersorum animantium figuris quibus Diana decoratur, aliqua Pythagoreorum placita et dogmata contineri animaduerte: qui Lunam, circumquaque ad inilar terra, quam nos incolimus, erandioribus habitari animalibus, et pulchrioribus consitam plantis arbitrati sunt. Et hanc quidem ipsorum doctrinam, haud temerè referamus ad varia hæc symbola, quibus Diana nostra (quæ ipsissima Luna est) exornatur, cuius totum fermè corpus non modò diuersis animalium, verùm etiam fructuum generibus vestitum conspicimus.
\clearpage
\begin{figure}[H]
\centering
\includegraphics[height=0.9\textheight,keepaspectratio]{figures/01.jpeg}
\end{figure}
\begin{figure}[H]
\centering
\includegraphics[height=0.9\textheight,keepaspectratio]{figures/02.jpeg}
\end{figure}
\clearpage
\begin{figure}[H]
\centering
\includegraphics[height=0.9\textheight,keepaspectratio]{figures/03.jpeg}
\end{figure}
\begin{figure}[H]
\centering
\includegraphics[height=0.9\textheight,keepaspectratio]{figures/04.jpeg}
\end{figure}
\clearpage
\subsection{Dianæ Ephesiæ varia Nomina.}
\paragraph{}
Communi et recepto indigitamento hæc Symbolica figura \foreignlanguage{greek}{ΑΡΤΕΜΙΣ ΕΦΕΣΙΩΝ}, id est, DIANA EPHESIORUM dicebatur, quamuìs sub hac effigie apud remotissimas etiam gentes coleretur. Certè Pausanias inter alios auctor est, varijs in locis Dianam Ephesiam summo in honore fuisse habitam. Nonnullis tamen alijs etiam titulis fuisse infignitam ex veterum monumentis obseruabis. Ac in primis Opin denominatam apud Macrobium Saturn. lib. 5. cap. 22. reperies. Is etenim ex Alexandro Aetolo refert, Ephesios dedicato augustissimo templo Dianæ celebrioribus Musarum alumnis præmia amplissima constituisse, ut in Deæ laudem carmina diversa componerent. In hisce peculiari encomio à quodam vate Opin dictam fuisse Dianam commemorat hoc versu:
\begin{quote}
Υ῾μνῆσαι ταχέων Ω῏πιν βλητῆραν ὀϊστῶν.
\end{quote}
\begin{quote}
\emph{Ut celebrem iaculis ornarent laudibus Opin.}
\end{quote}
\vspace*{-4mm}
\paragraph{}
Verùm enimuerò præ cæteris præconijs, quibus decorabatur, acceptus illi fuit ac peculiaris Magnæ Dianæ titulus: cuius rei præter profanos auctores testem etiam locupletissimum habemus S. Lucam in Act. Apostol. cap. 19. qui Dianam Magnam à Demetrio quodam seditioso ciue cognominatam pronunciat: eò fortassis quòd Ephesij Dianæ numen potentia cæteris Dijs antecellere crederent; vel fortè ob celebritatem augustissimi ipsius temipli; aut quia inter Magnos Deos (ut scriptores vetusti referunt) numerabatur. Hoc ipsum divinus ille Plato de Diana etiam (ut reliquos taceam, qui veterum deorum Theogonias contexuerunt) calculo suo confirmauit, dum Solem et Lunam Magnos Deos nuncupauit. Diana demum hæc nostra appositè à Diuo Hieronymo in Epist. Paul. ad Ephesios Polymamma denominatur, quem audire si lubet, licet venuste statuam nostram hoc modo depingentem: \emph{Dianam Multimammiam colebant Ephesij, non hanc venatricem, quæ arcum tenet, atque succincta est, sed illam Multimammiam, quam Græci πολύμαστον vocant, ut scilicet ex ipsa quoque effigie mentirentur, omnium eam bestiarum et viucntium esse nutricem.} A Minucio item Felice in Octauio: \emph{Multimammia etiam Diana Ephesia depingitur.} Obijcere autem posset aliquis Macrobij auctoritate fultus, Isidis potius hanc esse, quàm Dianæ effigiem; apud quem hæc habentur: \emph{Isis cunctâ religione celebratur, que est vel terra, vel natura rerum subiacens Soli. Hinc est, quod continuatis uberibus, corpus Deæ omne densetur: quia vel terræ, vel rerum naturæ halitu nutritur universitas.} Et hanc Macrobij descriptionem firmare videntur gemmæ non paucæ, aliaque veterum monumenta: in quibus Isis eadem forma, quà hic à Macrobio describitur, expressa est. Quo etiam schemate Canopum suum fingebant prisci illi Aegyptiorum mythologi, non apposita tamen mammarum multitudine. At nostra Diana ab Iside Aegyptia differt quidem formâ aliquantulum, non autem numine, quia nullo prorsus discrimine inter se distant. Nam Diana Ephesia turritum coronamentum, et varia animantium emblemata expressa habet: Isidis vero Phariæ tempora cingunt vulturis, aut accipitris exuuiæ, cum orbe patulo, reliquum autem corpus hieroglyphicis notis infignitumi (vel ut cum Apuleio loquar) \emph{miris extrinsecùs simulacris Aegyptiorum effigiatum conspiciebatur}: cuius ectypon maioris notitiæ gratia ex annulari gemma desumptum hìc subijcio.

Sed de Iside hoc adijciam: nusquam ceruos eius obsequio à priscis illis Aegyptijs destinatos fuisse. Id ex omnibus Aegyptiorum monumentis obseruare liquet, in quibus ceruos appictos nusquam reperies. Præterea Aristoteles hist. animal. lib. 8. cap. 28. et Plinius lib. 8. cap. 33. memoriæ prodiderunt, Africam primis illis seculis ceruos non tulisse: atqui de more rituquè priscæ illius gentis nulla animalia exotica dijs patrijs consecrabantur. Dianam verò hanc nostram ceruis tanquam individuis comitibus semper circumstipatam videmus. Mystica tamen hæc simulacri emblemata nobis ansam conijciendi abundè subministrare possunt, Ephesios Aegyptiorum disciplinas, et instituta præ oculis habuisse, dum tam varijs symbolis Dianæ statuam adornarunt. Verùm ut augustior patrij numinis maiestas redderetur, non solùm quæ præcipua erant Isidis attributa, sed etiam Magnæ Matris, seu Cybeles Phrygiæ muralem coronam, Cereris Eleusinæ boves et fruges, Dianæ Siculæ ceruos, et rosas mutuati, novum Naturæ numen sub Dianæ indigitamento composuerunt, unico simulacro omnium prædictarum virtutes, et proprietates coniungentes. Non enim varij sacrorum ritus, vel formæ deorum diuersæ apud priscos efficiebant, ut ipsa numina re differrent; quod multis rationibus Macrobius, Satur. lib. 2. Phurnutus, alijque Mythologi euincere contendunt, qui omnes Deos ad unicum Solis numen revocant. Apuleius quoque Isin, Deûm Matrem, Mineruam, Junonem, Dianam, Cererem, Venerem, Proserpinam, Hecaten unam eandemq; esse prædicat: diversis tamen nominibus celebratam, ac multiplici specie à varijs gentibus cultam. Quod et ex diversis argumentis in nostro hoc Dianæ signo expressis ostendere conabimur. Sed antequam rem ipsam aggrediar: duo antiqua Ephesiorum numismata propono, in quibus eadem prorsus effigie Diana Ephesia expressa visitur.

In anteriori orbe primi nummi Traiani effigies, et huiusmodi est inscriptio: \foreignlanguage{greek}{ΑΥΤ. ΚΑΙΣ. ΤΡΑΙΑΝ. ΣΕΒ.} alter verò, quem tanquam succenturiatum testem adduxi, exhibet in antica parte, Marci Aurelij imaginem, cum his litteris: \foreignlanguage{greek}{ΑΥΤ. ΚΑΙΣ. Μ. ΑΥΡ. ΑΝΤΩΝΕΙΝΟΣ. ΣΕΒ.} cuius delineandi potestatem mihi fecere Brutus et Franciscus Gotifredi fratres, unanimes in conquirendis, et asseruandis veteribus monumentis. Nonnulla etiam alia numismata eadem specie ab Ephesijs conflata, ab Huberto Goltzio, Guillelmo du Choul, Fuluio Ursino, Adolpho Occone, alijsque in lucem sunt edita. His breuiter de nominibus, dequè Dianæ forma præmissis, nunc ex qua materia simulacra ista Ephesij effinxerint despiciamus.
\clearpage
\vspace*{\fill}
\begin{figure}[H]
\centering
\includegraphics[width=0.95\textwidth,keepaspectratio]{figures/05.jpeg}
\end{figure}
\vspace*{\fill}
\clearpage
\subsection{Statuæ Materies.}
\paragraph{}
Ex duobus Ephesiorum nummis prolatis titulisque adiectis, hanc nostram statuam genuinam antiquæ illius Ephesiæ Dianæ speciem referre certissimè constat. Quanta verò arte et diligentia horum signorum opifices genuinam similitudinem ipsius Ephesini simulacri accuratè exprimere allaborarint, præter varia symbola affabrè cælata, diversus etiam ille marmorum color, ex quo hæc Dianæ signa compacta sunt, nullo negotio demonstrat: cum ad illius imitationem universum pectus ex marmore candido; vultum autem, manusque et pedes insititios ex lapide Lydio nigerrimo eos composuisse et efformasse videamus. Peruetustum enim ac celebre illud simulacrum Dianæ ex ebeno fuisse dolatum maxima veterum Scriptorum turba Plinio referente prodidit: idemque asserit lornandes in rebus Geticis. Hoc autem non novum inusitatumque fuisse priori æuo Pausanias in Arcadicis haud obscurè ostendit: apud quem simulacrum Dianæ Limnatidis ex ebeno fabrefactum reperitur. At Vitruvius lib. 2. c. 11. contrà sentire videtur, dum templi Ephesini signum, nec non lacunaria ex cedro facta fuisse commemorat. Alij verò vitigineum fuisse prædicant; inter quos Mucianus Consul apud Plinium habetur, qui se illud proximè spectasse dicebat. Sed hoc morosioribus examinandum relinquo. Ut ut sit, probabili coniectura assequi licet, Deæ membra prominentia vel ex nigra quadam materia composita, aut fusco aliquo colore obducta fuisse: et coniecturam meam iuuant non pauca Dianæ Ephesiæ marmorea signa, quæ Romæ extant, cum facie manibus, et pedibus ex lapide nigerrimo insertis; et hæc quidem opera codem fermè schemate, quo hæc nostra, efficta.

Nunc ut ad incepta me referam, vetustissimos illos Dianæ mysteriorum interpretes, ex primogenio gentium instituto totum Deæ corpus ex ebeno potiùs quàm alio quouis ligno efformasse, non equidem temerè conijcio, cum hoc genus ligni præ cæteris alijs, naturæ Lunaris sideris maxime congruere videatur. Nam per atrum et suboscurum ebeni colorem noctis opacæ tenebras (quod proprium est Lunæ, seu Dianæ tempus) convenienter illi Mythologi defignare potuerunt. Hæc etenim Græcis Νυκτίδρομος, Ε᾽ννυχία, et Μελάγχους; Latinis, \emph{Noctiluca, Noctiuaga, Dea Noctium, et Noctium sidus} passim dicitur. Cumq; constet, (ut supra ex D. Hieronymo referebamus) Naturam rerum omnium parentem sub hoc simulacro ab Ephesijs fuisse cultam; sic et ipsa nox apud vetustissimos Poetas, ut apud Orpheum (in noctis suffimento) rerum cunctarum parens habetur: quin chaos et tenebræ prima rerum exordia esse dicuntur. Non ineptè igitur ex huius disciplinæ fonte, ebeni lignum electum fuit ab Ephesijs mystis ad nocturnum Lunæ tempus designandum, et Symbolicum naturæ simulacrum fingendum: cum eius vestigia, caliginosa etiam sapientioribus semper fuerint, eiusq; virtus perspectu cognituque difficillima. Cereri autem (quam à Diana non abhorrere mox ostendam) ater etiam color gratus, acceptusque fuit. Unde et initia eius ex prisco Eleusiniorum ritu, nocturno tempore potissimùm peragebantur; eiusque Sacerdotibus sancitum erat, ne alia veste quàm nigra uterentur. Et apud Phigalenses Ceres cognomento nigra peculiari studio colebatur, prout Pausan. in Arcad. observavit. Addatur insuper, ebeni lignum putredini, corruptioniq; minimè obnoxium esse, ut Theophrastus in histor. Plant. et Plinius lib. 16. cap. 4. annotarunt. Propterea hæc arbor inter æternas numerabatur. Aeternitas verò in nummis Faustinæ Junioris, Lunam pro Symbolo fortita est. Quas ob causas proculdubiò Luna, seu Natura, ebeni lignum tamquam proprium et peculiare expetit.

Sed iam vetustissimæ illius Statuæ incunabula perscrutemur. Hanc non humana arte expolitam, sed cælo diuinitùs missam Ephesij tam arroganter quàm falsò prædicabant. Ad quod vetus quidam Poeta Græcus alludens, celebritatem templi Ephesini, famam simulacri Dianæ, ipsiusq; ciuitatis gloriam eleganti epigrammate celebrauit.
\begin{quote}
\hspace*{0mm}Τίς ποτ᾽ ἀπ᾽ οὐλύμποιο μετάγαγε Παρθενεῶνα,\\
\hspace*{5mm}Τὸν πάρος οὐρανίοις ἐμβεβαῶτα δόμοις\\
\hspace*{0mm}Ες πόλιν Ανδρόκλοιο, θοῶν βασίλειαν Ιώνων\\
\hspace*{5mm}Ταὺ δορὶ, καὶ Μούσαις αἰπυτάταν Ε῎φεσον;\\
\hspace*{0mm}Ηῥὰ σὺ φιλαμεύα πτυοκτὸνε μέζον ὀλύμπου,\\
\hspace*{5mm}Ταὺ τροφὸν ἐνταυθοῖ τὸν σὸν ἔθευ θάλαμον;\\

\hspace*{0mm}\emph{Quis tulit è cælo sublimem Partheneonem,}\\
\hspace*{5mm}\emph{Qui fuerat superis cognitus ante Deis}\\
\hspace*{0mm}\emph{Mœnia ubi Androcli, imperiumque existit Ionum,}\\
\hspace*{5mm}\emph{In claram Musis, militiaque Ephesum?}\\
\hspace*{0mm}\emph{An quia grata magis cælo tibi terra sit altrix,}\\
\hspace*{5mm}\emph{Percupis hic thalamum esse Diana tuum?}\\
\end{quote}
\vspace*{-8mm}
\paragraph{}
Sed missis fabulis, simulacrum illud potius ab Amazonibus positum dixerim: cùm Ephesini templi constructio ijsdem à Dionysio Periegete tribuatur.
\begin{quote}
\hspace*{0mm}Παῤῥαλίην Ε῎φεσον, μεγάλην πόλιν ἰοχεαίρης,\\
\hspace*{0mm}Ε῎νζα θεῆ ποτὲ νηὸν Α᾽μαζονίδες τετύχοντο,\\
\hspace*{0mm}Πρέμνῳ ἔνι πτελέης.\\

\hspace*{0mm}\emph{Maritimam Ephesam magnam urbem Dianæ,}\\
\hspace*{0mm}\emph{Ubi Deæ ædem quondam Amazonides struxerunt}\\
\hspace*{0mm}\emph{In trunco stipite ulmi.}\\
\end{quote}
\vspace*{-8mm}
\paragraph{}
Vero autem propius sit, Amazones simulacrum Dianæ vouisse, ut sacram ædem ei dedicarent. Fauet Callimachus hymn. in Dianam.
\begin{quote}
\hspace*{0mm}Σοὶ καὶ Αμαζονίδες πολέμου ἐπθυμήτειραι\\
\hspace*{0mm}Ε᾽ν πότε παῤῥαλίη Ε᾽φέσου βρέτας ἱδρύσαντο,\\
\hspace*{0mm}Φηγῷ ὑπὸ πρέμνῳ.\\

\hspace*{0mm}\emph{Tibi etiam Amazones belli affectatrices}\\
\hspace*{0mm}\emph{Olim in littore Ephesi statuam posuerunt,}\\
\hspace*{0mm}\emph{Fagino sub trunco.}\\
\end{quote}
\vspace*{-8mm}
\paragraph{}
Illud mirum, eiusmodi simulacrum intactum illæsumque remansisse, nec fuisse mutatum ipso Deæ templo septies restituto, ut Plinius lib. 16. cap. 40. testatum reliquit. Porro eàdem plane symmetria ac forma, qua reliquas Deorum imagines, primitùs formatum fuisse hoc simulacrum crediderim. Sed deinde Ephesiorum hieromystæ gloriæ patrij numinis consulentes, ut varias Dianæ seu Lunæ vires, effectusq; denotarent, aurea, argentea, aliaque ornamenta, pro tempore tamen, et ad libitum exemptilia excogitarunt; et diversorum animantium toreumata, et emblemata in ijs exprimi curarunt. Coniecturam nostram firmat candidus ille ornatus marmoreus, quo totum Deæ corpus obducitur, quandoquidem in nonnullis alijs Dianæ Ephesiæ dissimilia à nostris symbola contineri observarim.
\clearpage
\subsection{Corona Florea, et Turrita Dianæ vertex insignitur.}
\paragraph{}
Dianæ caput duplex cingit corona: quæ apposita fronti, florida est; altera muralis: utriusque minimè vulgaris ratio, sed ab Aegyptiorum sacris petita, quos neque radijs, neque lauro, neque olea, neque quercu patriorum Deorum capita cinxisse liquet. Isidis certè capiti Vulturis aut Accipitris pennas, modò exuvias aspidis, iam bouina coruna, et orbem apporiebant. Osiridis vertici calathus, seu modius pro corona erat. E fronte Harpocratis malum persicum cum folijs eminebat. Quæ hìc attexenda duximus, ut Ephesios haud absimile institutum præ oculis habuisse indicaremus. Nam Dianæ vertex apud cæteras nationes solâ falcatæ Lunæ sectiunculâ insignis conspiciebatur. Verùm, ut quod huius est loci exequar, in eius Strophio Rosæ, Chrysanthemique cum corymbis sunt: decenter quidem, cùm iuxta Plinij mentem lib. 21. cap. 25. Lunæ globum rotundum imitentur, et ad Solis repercuisum aurei videantur. Ea propter Lil. Giraldus cororiam ex heliocryso siue chrysanthemo Dianæ attribui à priscis scriptoribus memorat; nec absque ratione, quòd hoc floris genus helichrysum, seu heliochrysum denominatum perhibeant ab Helichryse quadam Ephesia puella, que prima è flore chrysanthemo Dianam coronauit. Quid si per rosas simul pactas, Cereris et Dianæ cognatum numen sagax Ephesiorum religio denotarit? Numquid pariter rosas Dianæ veluti montium, et hortorum dominæ, florumque procreatrici concedemus? Apuleius lib. 11. Metam. multifarias illi coronas adscribit. \emph{Corona multiformis varijs floribus sublimem distinxerat verticem.} Idemque paulò inferiùs, coronam rosaceam in pompa illa, et apparatu Deæ celebri à Sacerdotibus prægestari solitam pronuntiat: alibi autem \emph{nunc, albo candore lucida, nunc croceo flore lutea, nunc roseo rubore flammida}, eadem ab illo corona, describitur. Quid verò aliud hæc florum varietas in Dianæ corona, quàm diversos illos circulos qui circa Lunam apparent designare posset: Hos etenim constat coronas à probatis auctoribus vocari: fidemque nobis faciet Plinius lib. 2. cap.... cuius verba placet adscribere: \emph{existunt eædem corona circa Lunam, et circa nobiliora astra}. Aristoteli et nonnullis alijs Græci sermonis antesignanis hi circuli ἁλῶνες dicuntur. Quam vocem interpretes modò coronam, modò circulum vertunt. Seneca in princ. quæst. \emph{hunc} (inquit) ἀλὼ \emph{nuncpant, quem nos dicere coronam [...]ssime possumus}. Rosaceam quin etiam coronam ad Cererem non inuitus traham. Cuinam enim decentius, quàm Cereri coronam ex lectissimis floribus plexam Ephesiorum religio tribuere potuisset, cùm ex felici eius gremio flores ortum ducant? Equidem corona rosacea, et florida apud veteres hilaritatis erat indicium: huius autem Deæ festum summo cum lusu, et gaudio antiquitus celebrabatur, prout Phurnutus testatur. Non pauci item alij flores intertexti, inter quos et narcissus. Ceres etenim apud Sophoclem narcissum, præ alijs floribus exoptat, et Magarnum Dearum coronamentum esse perhibet, quas eius Scholiastes Cererem, et Proserpinam interpretatur. Item hiacynthus ex Pausaniæ auctoritate, qui hiacynthinas corollas sacris Cereris adhibitas fuisse prædicat in Chthonijs diebus festis. Cæterùm omnibus palam est, rosam esse florum reginam, ac huius symbolo potentiam, regnum, et imperium convenienter designari. Appositè igitur Cereri rosa competit, cùm hæc sit Omnium regina ut festiuè notauit Euripides.
\begin{quote}
Καὶ φίλα Δαμάτηρ θεὰ, ἁπάντων\\
Α῎νασα, ἁπάνὧν δὲγᾶ τροφὸς.\\

\emph{Et amabilis Ceres Dea Omnium regina,}\\
\emph{Omnium vero terra altrix.}\\
\end{quote}
\vspace*{-8mm}
\paragraph{}
Nemini autem mirum videri debet, si Ceres, seu Diana Ephesia, quæ domina mundi habebatur, duplici corona donata, exornataq; fuerit: cùm peculiari quodam cultu Dianam Ephesiam coronis, non autem sanguinolentis victimis cultam fuisse proditum extet. Etymol. in Ψ, ἐφέσου Ephesij enim στεφάνοις διά θαλλῶν τὰς ἱκεσίας ποιὁὐσιν \emph{coronis per oleæ ramos supplicationes faciunt}. Dictamno etiam hanc Deam veteres coronabant, quòd partum facilem præstaret, ut Euphorion docuit. Sed longiùs fortasse prouectus sum, quàm oportuerit. Ad muralem coronam iam deflectamus. Cur Dianæ turrita corona? cum peculiare potius coronamentum Magnæ Matris, aut Cybeles, vel Deæ Syriæ apud vetustissimos scriptores habeatur. At si quis penitus Dearum prædictarum attributa, et cæremonias perpenderit unum idemq; numen facile deprehendet. Idipsum autem ex vetustissimorum scriptorum placitis ostendere aggrediar: et eò quidem lubentiùs, ut omnia symbola in Statua nostra expressa faciliùs elucescant. Rheam in primis à Cerere haud esse diversam non est dubitandi locus, quam Latini Opim ut monuit Athenagoras in Orat. pro Christianis vocitant, quòd ipsius auxilio vita constet: vel ut Fulgentio placet, quòd esurientibus opem ferat. Hanc poetæ prisci Deûm Matrem, et Magnam Matrem nominarunt: quòd non homines modò, sed omnia è terra proveniant. Quare Porphyrius Rheam, et Cererem, pro Terra capi voluit. Notus est Orphei hymnas in Rheam, qui eam cunctorum matrem, et mundi medium tenere memorat. Cererem verò terram ab antiquis creditam, cùm multis rationibus, tùm ex ipsius nomine satis constat. Quid enim est aliud Græcis Δημήτηρ, quam Γημήτηρ? Ab ipsa alimentorum largitione Platoni in Cratylo Δημήτηρ quasi διδοῦσα μήτηρ vocatur, tanquam \emph{exhibens mater}. Hùc referas liquet, illa quæ Arnobius lib. 3. ex Gentilium fabulis exarata reliquit. \emph{Terram quidam} (inquit) \emph{è vobis, quòd cunctis sufficiat animantibus victum, matrem dixerunt esse magnam; eandem hanc alij, quòd salutarium seminum frugem gerat, Cererem esse pronunciarunt.} Quibus statuam hanc nostram quodammodò depingere videtur. Hc nempe ex mamillis penfilibus cunctis animantibus, qæe totum eius corpus ambiunt, alimentum porrigere, et ante pectus immensam frugum, fructuumque copiam gerere conspicitut. Quocirca à gerendis frugibus Cicero de Nat. Deor. lib. 3. Cererem tanquam Gererem prima litera immutata dictam autumat. Ac veluti Ceres omnium, bonorum largitrix, dispensatrixque credebatur, quod munus etiam Rheæ competere à nonnullis traditum est. Hanc ob causam illi vitigineum simulacrum tanquàm cibi potusque largitrici Dædala sacrauit antiquitas Euphorione teste. Cererem præterea à Cybele non esse alienam docent inprimis harum cognati, germanique sacrorum ritus, Cereri, enim perinde ac Cybele tympana, et cymbala aderant, quorum sonitu Ceres in filiæ inquisitione primùm usa est. Cybele porrò curru vehi, et rotis sustineri credebatur: quia mundus rotatur, et volubilis est, et hæc ipsa terra eò quòd pendeat in ære dicebatur ut Seruius docet. Ceres denique, et Isis propter Sororiam frugum inventionem, æquè ambabus attributam, unum idemque numen constituunt. Triticum enim ab Iside primùm cum hordeo hominibus commostratum scribunt Diodorus lib. 1. et Martianus Capella. Et ideò in ipsius pompa triticum, et hordeum præferebant, ut ab ipsa repertas fruges testarentut. Herodoto Isis eadem est quæ Δημήτηρ. Cui Cedrenus lib. 2. adstipulatur: à quibus Diodorus paulòante memoratus non abludit, dum Cererem cum Iside, quæ ipsissima est cum Luna coniungit. Unde ex primogenio tàm Aegyptiorum, quàm Græcorum instituto, æque ambabus boves abscripti fuerunt. Cæterùm ut obscuritatis depulsetur offensa Martianus iterum Capella nobis adeundus erit, quem de Lunari orbe verba facientem, sic audiamus. \emph{Ineo Sistra Niliaca, Eleusinaque lampas, arcusque Dictynæ, tympanaque Cybeles}. Ex quibus observare est, Isidis, Cereris, Dianæ, et Cybeles attributa Lunæ etiam adseribi. Confer cum his Luciani Deam Syriam, et Apuleianam Isim, quæ minime diversa numina fuerunt, quamuis varijs nominibus celebrarentur. Dianam demum in Luna cultam nemo nescit, nisi qui vetustissimos mythologos non legerit. Hecate porro Diana, Luna, et Proserpina à Festo, et Diodoro Siculo pro uno eodemque numine accipiuntur. Exinde Diana triformis fingebatur, in quam extat Horatij versus.
\begin{quote}
\emph{Ter vocat audis, adimisque letho diua triformis.}
\end{quote}
\vspace*{-4mm}
\paragraph{}
Concordem hanc germanitatem etiam adnotat Virgilius.
\begin{quote}
\emph{Tergeminamque Hecaten, tria virginis ora Dianæ.}
\end{quote}
\vspace*{-4mm}
\paragraph{}
Magna autem illa, quam nos hìc cogitamus Ephesiorum Diana, Nature, seu universi, et materiæ primæ vires, potestatemque sibi vendicat: cuius insignia, et ornamenta, si attente spectes, idipsum deprehendes. Quanta verò cognatio Dianæ Ephesiæ cum Iside intersit, vel hinc colligere est, quòd illa Dìuo Hieronymo teste superius laudato, apud Ephesios Naturæ typum referebat. Hæc verò apud Aegyptios nihil aliud quam ipsummet Naturæ numen loquebatur, prout Macrobius Saturn. lib. 7. cap. 21. et Apuleius Metam. 6. Aegyptiorum monumentorum peritissimi enunciant. Possent hìc sexcentæ auctoritates cogi ad firmandum, immensam hanc Dearum turbam ad unicum Lunæ siue Dianæ numen à priscis mythologis fuisse redactam. Sed cùm res notior sit, quàm ut de ea prolixior sermo habeatur, ad institutum redeamus. Et coronam hanc turritam, quæ priùs Magnæ Matris, Cybeles, aut Deæ Syriæ peculiaris esse videbatur, Dianæ etiam nostræ competere ostendamus. Ac huius quidem meminit Halicarnasseus, dum canistrophorarum capita corollis ornari velut Dianæ Ephesiæ simulacra perhibet. Quàm verò decenter Naturæ hæc muralis corona tribueretur, causa in promptu est: quoniam rerum humanarum regina, incolendarumque urbium rationes edocuisse credebatur; corona quippe pro muris urbium usurpaturmuri etenim tanquam urbium coronæ sunt. Hæc de corona satis superquedicta. Iam ad velum explicandum orationem conuertamus.
\clearpage
\subsection{Velo Dianæ nox Indicatur.}
\paragraph{}
Velo densas noctis opacæ tenebras, quibus aer circumfunditur, adumbrari peruium est. Hunc autem Dianæ velandi ritum non apud Ephesios modò, verùm etiam apud Aegienses receptum fuisse à Pausania in Achaicis edocemur. \emph{Habent Aegienses vetustum Lucinæ fanum: Dea ligneum signum à vertice ad calcem velatum, præter eos tamen, summas manus, et pedes. Sunt verò quæ, non tanguntur partes è marmore pentelico. Alteram manum porrigit, altera facem præfert.} Atqui Lucina à Diana non est aliena: faces certe utrique fuere communes. Et Diana ipsa, seu mauis Luna, lumine suo ac splendore tanquam face ardenti, opaca et caliginosa conspicua lucidaque reddit. Huius quoque rei gratia quòd noctu, et in tenebris luceat, et incedat, Noctiluca, et Noctiuaga vocitata. Meritò igitur velum, tanquam obscuritatis à se depulsæ indicium à tergo demissum finxerunt. Ex quo Theologiæ fonte Apuleius in illa luculenta, et accurata Lunæ siue Isidis descriptione pallam ei nigerrimam assignare non dubitauit. Velum præterea modestiæ, ac pudoris virginalis signum esse in confesso est. Unde hoc etiam nomine Dianæ tribuendum.
\clearpage
\subsection{Ceruorum Comitatu Diana Gaudet.}
\paragraph{}
Ceruos Diane obsequio ceu ministros, et (uti sic dicam) pedissequos semper fuisse addictos vulgatissimum est. Inde est quòd peculiari elogio ἐλαφηβόλος \emph{Cermiaculatrix} à Græcis dicebatur. Inde etia est quòd inter animalia, quæ signum Dianæ exornant, bini cerui Deæ caput circumstant: idque meritò, cùm reliqua quadrupedum genera pernicitate superent, adeò ut illi Mythologi velocem Lunæ cursum innuere volentes Ceruos pro symbolo elegerint. Luna equidem tantæ celeritatis est, ut spatio viginti septe dierum horarumq; octo totum permcect signiferum, quem Sol annuo tantùm temporis curriculo perlustrat. Quo minùs mirum sit Dianam cerui effigie designatam. Equidem vetus extat Pub. Licinij Gallieni nummus, in quo cerua est cum hac epigraphe DIANAE CONSERVATRICI. In alio item numismate æneo ab ipsis Ephesijjs in honorem Commodi Imp. cuso bigæ ceruorum aurigante Diana, et hæ literæ circum impressæ sunt, ARTEMIC EΦECIΩN. Visitur et alius Marc. Antonini Caracallæ, in quo Dianam duæ ceruæ trahunt. Præterea Callimachus in hymn. et Claudianus de laud. Stil. currui Dianæ ceruos itidem adiungunt. Sed et ob venationis studium, quo tenetur, et cui dominatur Diana, Cerui sub eius tutela, et præsidio sunt. Quapropter in veteribus Græcorum nummis Dianam videre est venatricis habitu succinctam, ocreatam, pharetra à tergo insignem, adstante cerua. In multis verò ipsam videas in ceruos sagittas eiaculantem: qua profecto de causa montes, nemoraque eidem accepta dedicataque fuerunt. Quòd Horatius Od. 22. lib. 3. eximiè demonstrat.
\begin{quote}
\emph{Montium custos, nemorumque virgo.}
\end{quote}
\vspace*{-4mm}
\paragraph{}
Montium autem ac syluarum tutricem dictam puto, quòd in montibus cerui, et reliquæ feræ stabulentur; vel certè quia noctu incedit, ac suo splendore confragosa ac densa quæquæ loca perlustret. Quæ Phurnutus respiciens Dianam ὀρεσιφοίτην \emph{Monticolam} nominauit. Nemoribus præesse fingitur ex Fulgentij mente, quòd arboribus, et fructuum succo augumenta ingerat. Sin autem alias rationes desideres, ob quas præsertim ceruus in delitijs Dianæ fuerit, dabo. Rorem (aiunt) Jovis, et Lunæ filium esse, idque ex doctrina Poetæ vetustissimi inferiùs (cum de Apibus erit sermo) ostendam. Constat autem, ceruos æstiuis temporibus ipsoque statim Solis exortu rorem anhelanti ore excipere, atque eo se reficere; quod ipsum Physici obseruarunt, et in hieroglyphicis Pierius. Cæterùm cum cerui longissimæ sint vitæ, non malè Lunæ congruunt. Quam pro æternitatis symbolo in nummis Faustinæ Junioris expressam monuimus. Per quatuor demùm hosce ceruos juxta Deæ caput appositos, fortè quatuor Lunæ sese mutantis facies designari poterunt. Verùm illud coniectando, potiùs, quàm asseuerando dictum esto.
\clearpage
\subsection{Leones Dianæ in Simulacro cælati consociatam Solis, et Lunæ potentiam indicant.}
\paragraph{}
Leo quamuìs Solis animal Solarisquè naturæ particeps ab omni fermè Mythologorum cœtu perhibeatur; Dianæ tamen, Magnæ Matri, Cybele, Cereri, ac Isidi adscriptum fuisse uetera monumenta nobis insinuant. Quarum facturum me operæ pretium puto (cum vetustissimum simulacrum discutiendum susceperim) si prisca numismata ad rem nostram stabiliendam facientia Scriptorum vice proferam. Igitur Magnæ Matris effigies in pulcherrimo æreo nummo Faustinæ Antonini Pij filiæ, cum hac epigraphe MATRI DEUM SALUTARI, Leonibus inclyta conspicitur. Cybelem Leoni insidentem Septimij Seueri numismata referunt apud Adol. Occonem. Eius similiter currum Leones trahunt in nummo Faustine Junioris apud Petrum Mareschal Patritium Bisuntinum. Hæc etiam in quarto denario Volteiæ gentis leonum famulitio gaudet. In nummis Faustinæ Junioris Isidis Sistratæ imago cum leone ad pedes exprimitur. Diana demùm in arca illa celebri, quam Cypselidæ in Olympia dedicarunt, alata insurgebat, et Leonem ad dextram, pantheram habebat ad sinistram, ut Pausanias in Eliac. memoriæ prodidit. At verò quid Leonibus cum hisce Deabus peculiare, et cur illis sacri fuerint, nobis inquirendum est. An ideo Magnæ Matri leones solutos et mansuetos adiunxerunt, ut designarent nullum esse terræ genus tam asperum atque ferum, quod non subigi coliquè possit? Sic lego Varroni placuisse. An verò etiam ut significarent maternam pietatem omnia superare. An quòd Cybeles cum Dea montana sit, vehi leonum jugo debuit. An verò currum Cybeles Leones trahere dicuntur, quòd ab ijs nutrita sit? An quia Cybele, que et terra est, vim gignendi à Sole recipit? Solis verò præcipuum symbolum Leo, qui et Cereri ideo forsan adscriptus fuit, quia hæc mensis Augusti tutelaris erat, ut in Kalendario antiquo habetur. Dianæ porrò destinatum crediderim ob duplicem leonis naturam: eius enim anteriores partes, quia robore ingenti præditæ sunt, celestia; posteriores quia debiliores, terrestia referunt. Quæ cause licèt ad confirmationem satis sint, aliam tamen causam ex Arati Phenom. petere licebit, ubi ex Nigidio refert, Leonem, qui in astris locum obtinet, iussu Junonis apud Lunam nutritum, educatumque fuisse. Porrò quid causæ est, cur isti quatuor Leones humeris ac brachijs Deæ insident, sibiquè inuicem respondent? Dixerim non grauatè concordem Solis, et Lunæ virtutem in rebus omnibus producendis, ac fouendis per eos significari; ostendique nexum, quo terrena cælestibus conjunguntur; à quibus dum infusum recipiunt, eadem inter se, et cum cælo aptissimè cohærent.
\clearpage
\subsection{Cancer cur Ephesiæ Dianæ Sacer.}
\paragraph{}
Inter alias imagines, quibus omne cælum distinxit antiquitas, non ignobile est Cancri signum, cuius figuram è collo Dianæ suspensam, et in extrema etiam simulacri parte appictam intueri licet. Hoc autem compactile testaceorum genus multas ob causas Dianæ siue Lunæ acceptum fuisse palàm faciemus. Nam ut sub Leonis effigie Solem Mythologi cognouerunt, sic Aegyptij pro Luna Cancrum pingebant. Causa est, quòd in ipsa genitura mundi Solem cum Leone, Lunam cum Cancro ortam crederent. Ad hæc Cancer cum Lunari sidere haud pauca communia habet: quippe qui srigidus, humidus, aqueus, nocturnus, et fœmineus sit. \emph{Quòd eius à fronte protenduntur apices duo rotundi, et acuti firmi admodùm, sub quibus duo cornua minora articulata}, secundum Rondeletij observationem. Eius etiam corpus rotundum et compactum à Lunari sidere haud multum aberrat, præsertim dum chelas expandit: tunc enim Lunæ corniculantis figuram quodammodo referre videtur. Nec ultimum decus Lune cornua, cui omne genus animantum cornigerum sacrauit antiquitas, ob corniculatum globum. Cornua etiam inter precipua Isidis symbola sunt: eiquè abiectissima insecta consecrata, ut illud Scarabeorum secundum genus, quod bicorne est. Præterea ut Luna ob velocem cursum incedere videtur; sic Cancer ferè solus è crustaceorum genere non natat, sed incedit. Ita Aristot. Hist. Anim. 1. cap. 1. \emph{Cancri quamquàm aquatiles, tamen gressiles sunt, transversoque latere procedunt.} Accedit quòd inter venatores annumerantur Cancri: sub Diane verò tutela omnes venatores fuere. Sed hoc precipuè ad rem facit (quod rerum naturalium scrutatores observarunt) Cancros vim Lune sentire; ea crescente pleniores et probatioris saporis fieri; decrescente verò, minui: sicut et alia conchylia, ut canit Manilius Astronom.
\begin{quote}
\emph{Ad Lunæ motum variant animalia corpus.}
\end{quote}
\vspace*{-4mm}
\paragraph{}
Longevum est etiam animal: et quemadmodum Luna reviuiscere videtur, ità Cancer crusta, seù tegumento se exuit, et statim revirescit; quòd est à Physicis animadversum. Cancri præterea ad nocturni luminis splendorem accurrunt, et noctu ut plurimum pasci solent. Quæ profecto satis euincunt, Cancris multiplicem inesse cum Lunari sidere consensionem; nec sine mente, consilioque illi fuisse adsignatos. Haud tamen omittendum, duxerim, quod Golzius nullo citato auctore refert, Cancrum apud mystas prudentiam denotasse, et hanc ob causam è collo Dianæ Ephesiæ suspensum. Idem apud Platonicos hominis ortum viamque designat: volunt enim animas per Cancri januam in humana corpora dimitti; et veluti Capricornum Deorum, sic Cancrum hominum portam vocant, quòd per eam egredientes animæ humanum in corpus transeant. Et hec ad nostrum emblema aptari possunt secundum Physiologos, qui Lunare lumen naturæ generationi, ac vite præesse volunt. Istud etiam hìc attexatur. Apud Brutios peculiari et insigni coronamento Diane caput Cancri testa ornatum, ut ex eorum nummis palàm est. Antonini Pij numisma ab Aegyptijs signatum in aversa parte expressum Cancrum habet chelas expandentem, et falcatam Lunæ sectiunculam apprehendentem. Eodem planè schemate, et pari symbolo Cancer sculptus est in veteri gemma, quæ in Pinacotheca Eminentissimi Cardinalis à Balneo adservatur. Verùm ista de Cancro sufficiant. Vocant nos ad se bine Victoriæ Dianam corollis et palmis insignientes: earum igitur explanationem aggrediamur.
\clearpage
\subsection{Victoriæ Binæ Dianam Ephesiam cur corollis, et palmis insigniunt.}
\paragraph{}
Binæ Victoriole Dee pectus exornantes nos sollicitant, ut è sanctioribus Ephesiorum promptuarijs causas repetamus, ob quas præsertim Dianæ Ephesiæ Victorie tribuerentur. Etenim non sine mysterio id ab Ephesijs prestitum. Hoc enim argumento cuncta submitti Lune potestati innuebant. Vel certè corollas memoriam esse beneficiorum à Natura siue Cerere generi humano collatorum volebant. Expresse verò sunt alis passis, cum corollis et palmis, qui solitus est Victoriarum habitus, ut liquet ex Claudiano de Laud. Stil.
\begin{quote}
\emph{Illa duci sacras victori panderet alas,}\\
\emph{Et palma viridi gaudens.}\\
\end{quote}
\vspace*{-8mm}
\paragraph{}
Victorie igitur imagines adeò gratæ, accepteque fuerunt Dianæ Ephesie, ut non modò eas ad pectus decorandum admiserit, verùm etiam ab his
coronari exoptarit; ut fidem facit Gordiani Pij æreus nummus ab ipsismet Ephesijs signatus, in quo Polymamma Diana visitur, dextra scipionem preferens, quæ è binis Victoriolis utrinque volitantibus coronatur: ante ipsam verò muliebris figura est cornu copie gestans. Singularis hìc ereus nummus cum infinitis propemodùm alijs in Museo Eminentissimi Cardinalis Francisci Barberini asservatur. Inter præcipua Victorie argumenta palma surgit, que rectè Cereri tribuitur: cum per palmam iustitiam designari palam sit: eò quòd fructus reddat pari cùm folijs equilibrio: id quod Pierius Valer. observavit. Hanc autem legum inventricem esse liquet vel ex Virgilio Aeneid. 4.
\begin{quote}
\hspace*{15mm}\emph{mactant lectas de more bidentes}\\
\emph{Legiferæ Cereri, Phœboque, Patrique Lyæo.}\\
\end{quote}
\vspace*{-8mm}
\paragraph{}
Palmam preterea ad alia numina sub nostro simulacro contenta non inuitus traham. Et hanc arborem inprimis ipsamet Luna tanquam propriam facilè sibi vindicabit: cùm palmam hoc peculiare cum Luna habere rerum naturalium scrutatores observarint, ut sola omnium arborum per ortus Lunæ singulos progeneret. Lunam item vitæ humanæ præesse ex mystis præmonui: atqui palma pro ipsiusmet vitæ symbolo ab Orpheo commendatur: nec ineptè quidem, cum hæc arbor trecenta sexaginta commoda mortalibus elargiri feratur: quod optimè noverant Babylonij, qui suis in hymnis huiusce arboris proprietates decantare solebant. Iam si ramorum eius naturam attentiùs perscrutemur, illis cum terra peculiare quid inesse deprehendemus. Hi etenim quò maiori pondere grauantur, eò plùs in altum feruntur, surguntque adversùm pondus, ut Aristot. 7. Problem. et Plutar. 7. Sympos. referunt: pari etiam ratione, quò plus tellus sulco premitur, eò uberiorem frugum copiam agricolis suppeditat. Cæterùm cum Isim in Dianæ nostro simulacro reperiri non semel inculcarim, huic etiam palmæ appictæ quadrare possunt. Nam Isidi ex recondita hierophantum doctrina calceamenta ex folijs huiusce arboris intexta, Apuleio locupletissimo teste, probantur: \emph{pedes} (inquit) \emph{ambrosios tegebant soleæ palmæ victricis folijs intextæ}. Et præterea in solemni eius pompa, is qui tertius incedebat, palmam auro subtiliter foliatam attollebat. Hæc de palmis dicenda habui.
\clearpage
\subsection{Encarpus e Collo Dianæ, dependens cur è varijs fructibus.}
\paragraph{}
Terra cùm non hominum solummodò animaliumquè benigna sit parens, verùm etiam omnium frugum fructuumquè productrix, appositè propterea Ephesij encarpum, seù jugamentum lemniscatum ex varijs florum fructuumque generibus confectum è collo Dianæ dependere finxerunt, utpotè cùm hominibus ea suppeditarit, ac dispertierit, quibus vesci queunt. Inter præcipuos verò fructus, ex quibus sertum compositum esse videtur, poma et papauera emicant, et illa quidem convenienter, quia ob rotunditatem terræ figuram referunt. Hæc autem multiplicem ob causam Dianæ, seù mauis Cereri accepta, oblataque fuerunt. Papauer in primis apud omnes scriptores pro fertilitatis symbolo habetur: unde et Ovid. lib. 2. Metam.
\begin{quote}
\emph{Antes fores antri fecunda papauera florent.}
\end{quote}
\vspace*{-4mm}
\paragraph{}
Huius etiam fructus telluris imaginem exprimere videretur, nam inæqualitas illa, seù turbinatum Papaueris corpus valles et confragosa montium ramenta denotat. Interna verò hominibus, et subditis assimilantur, uti Phurnutus retulit. Poterit et legiferæ Cereri attribui propter illa receptacula æquis interuallis inter se distincta. Callimachus in hymno ad Cererem papauer ei assignat. Apud Clementem Alexandrinum inter præcipua dona, quæ Cereri offerebantur, papauer recensetur. Quam fortè ob causam Virgilius papauer Cereale tanquam Cereris donum exoptatissimum prædicarit.
\begin{quote}
\emph{Nec non et lini segetem, et Cereale papauer}\\
\emph{Tempus bumo tegere.}\\
\end{quote}
\vspace*{-8mm}
\paragraph{}
Ob vim autem somniferam rectè Lunæ nocturnoque tempori conveniet: unde et illi apud scriptores somniferi cessit appellatio. Apud Ovidium nox cum papauerea corona depingitur: hoc enim tempore somnus propter humectationem potissimùm vires suas exercet. Reliquæ verò florum fructuumque figuræ, quas assequi facile non est, nihil aliud quàm omnis generis frugum copiam designare videntur, quam terra seminibus grauidata parit, funditque liberaliter: flos etenim in plantis frugem paulò post succrescentem pollicetur. Dianæ etiam fructus grati acceptique fuerunt, ut Pausanias in Achaic. testatum reliquit. Apud Patrenses enim Dianæ Laphriæ sacra facientes pomiferarum cuiusuis generis arborum fructus offerre solemne habebant. Non mirum igitur, si encarpus collo Dianæ appensus varijs fructibus et floribus adornetur. Verùm ne hic contextus incompositus dilabatur, vitta seù fascia tenui astringitur: ut per vittam obliquus flexuosusque Lunæ cursus intelligeretur.
\clearpage
\subsection{Glandes Indices Primi Cibi dum terra mansit inculta.}
\paragraph{}
Encarpi extrema querneæ glandes tanquam elenchi coronant, quæ Cereris beneficia satis superque arguunt: cum in primis rerum incunabulis hoc fructuum genus mortalibus alimenta præbuerit; sed Cereris beneficio in meliora, et mitiora commutata fuerunt. Quod apertè declarat Ovidius in Fastis:
\begin{quote}
\emph{Prima Ceres homine ad meliora alimenta vocato}\\
\hspace*{10mm}\emph{Mutauit glandes utiliore cibo.}\\
\end{quote}
\vspace*{-8mm}
\paragraph{}
Quò facit et illud Virgilij Georgicorum initio:
\begin{quote}
\emph{Liber, et alma Ceres, vestro si munere tellus}\\
\emph{Chaoniam pingui glandem mutauit arista.}\\
\end{quote}
\vspace*{-8mm}
\paragraph{}
Antiquitùs agricolis in more positum fuit, ut quotiescumque Cereri rem divinam facerent (faciebant autem, dum spicæ frugem emittere ceperant) querneis coronis uterentur. Cuius generis coronamenti meminit Athenæus lib. 15. Eundem feriunt scopum Maroniani versus Georg. lib. 1.
\begin{quote}
\emph{Falcem maturis quisquam supponat aristis,}\\
\emph{Quam Cereri tota redimitus tempora quercu}\\
\emph{Det motus incompositos, et carmina dicat.}\\
\end{quote}
\vspace*{-8mm}
\paragraph{}
Non possum autem hìc silentio præterire grauissima illa verba, quæ in invocatione Apuleiana ad Lunam habentur: \emph{Regina celi, siuè tu Ceres alma frugum parens originalis, quæ repertu lætata filiæ, vetustæ glandis remoto pabulo, miti commonstrato cibo nunc Eleusinam glebam percolis.} Postquàm verò huiusce Deæ beneficio terra nouis frumenti seminibus locupletata luxuriauit, glans contemni cœpta est: taleque fuit mortalium in glandiferas quercus odium, ut inde natum sit illud, ἅλις δρυὸς \emph{Satis quercus}. Et hæc fortè fuit ratio, cur querneas glandes huic encarpo tanquam in Deæ tropheo protomystæ suspenderint. Sed et Hecate, quæ à Luna non abludit, corona è quercu plexa decorari gaudebat, sicut Aeschylus declarat. Addamus insuper, quòd glandes appictæ Naturæ providentiam adumbrent: siquidem glandiferæ quercus primis hominibus alimenta subministrarunt. Ad hoc igitur Naturæ beneficium alludere volentes alium fructum glandibus associarunt, quem, ut ex eius figura orbiculari et oblonga apparet, dactylum esse autumo; neque præter rationem, cùm paulò ante palmam Cereris delitijs adiudicatam fuisse dixerimus; cuius fructus tanta est facultas, ut præterquam quòd in esu sit suavissimus, ex eo etiàm vina, et panem conficiant Orientales, et quadrupedibus edendum præbeant.
\clearpage
\subsection{Mammæ Nutritionis Sive alimentorum symbola.}
\paragraph{}
Aegyptiorum Theologia, Lunam matrem mundi, à Sole gravidatam fœcunda generationis principia diffundere superstitiosè credidit, ut in libros Aristotelis περὶ ἑρμηνείας scribit Ammonius. Ab hoc disciplinæ fonte illos suam Isim essormasse reor: et fors inde factum ut hinc 	τιθήνη et πανδεχής, ex Platone apud Plutarchum lib. 1. de Iside et Osiride existimetur: quòd sit scilicet nutrix, et susceptrix generis humani. Proindè non mirùm si frequentèr eius imago pensilibus, et prominentibus mammis conspicua in priscis monumentis repræsentetur. Etiam apud Macrobium continuatis uberibus depingi paulò ante diximus. Ad Aegyptiorum imitationem Ephesij simulacrum Deæ Naturæ Polymammæ cum infinito scilicet uberum numero composuerunt. Cuiusmodi à Diuo Hieronymo, et Minucio Felice in Octavio depictam fuisse superiùs indicavimus. Apud Arnobium Cererem cum grandibus mammis efformatam intueri licet: idemque paucis lineis interiectis huic mammas item promptas adscribit. Pausanias in Messen. Dianæ nutricis, quæ παιδοτρόφος indigitabatur, mentionem facit. Et apud eundem in Attic. legere est, Athenienses delubrum Telluri puerorum nutrici exædificasse. Apud Aristophanem verò hec puerorum altrix etiam vocitatur. Lucretius paritèr Cererem mammosam facit hoc versu:
\begin{quote}
\emph{At gemina et mammosa, Ceres est ipsa ab Iaccho.}
\end{quote}
\vspace*{-4mm}
\paragraph{}
Geminam dixit acsi opimam et amplam innuere voluisset. Ex adductis nemo non videt, Isidi apud Aegyptios, et Macrobium; Dianæ apud Ephesios, et Messenios; Telluri apud Athenienses; Cereri apud Lucretium et Arnobium, nutricis seù mammosæ elogium æquè fuisse attributum: nec ab harum consortio Rhea eliminanda erit, cùm et ipsa matris titulo gaudeat. Conspirare autem hæc omnia videntur, ut hasce omnes Deas sub unico Dianæ nostræ simulacro copulatas firma probatione retinere possimus. Atqui non indecenter Diana, seù mauis Ceres Polymamma à priscis illis mythologis efficta, quàm superiùs Δήμητραν dictam ὡς δὴ μητέρα οὖσαν ex vetustissimis scriptoribus protulimus: quòd utiquè mater altrixquè sit omnium, quæ in ea vitam degunt. Alij γενητείραν dicunt, quòd omnia pariat, et recipiat. Sed et à Virgilio alma dicitur, ab alendo nimirum ut Servius interpretatur: benignis enim Telluris Lunequè viribus alimenta præcipuè deberi ueteres agnoscebant. Huius autem singularia beneficia Columella lib. 3. cap. 22. decantat his verbis. \emph{Quibus alma Tellus annua vite, væluti æterno quodam puerperio læta mortalibus distenta musto dimittit ubera.} Cæterum hæc nostra multimammia genuinus est prouidentissimæ Naturæ typus. Apud priscos enim papillæ rotundæ atque prominentes fæcunditatis erant indicia: et hoc præsertim hieroglyphico Aegyptij ad Isidis, seù Cereris fœcunditatem, ac rerum omnium affluentiam designandam usos fuisse Apuleius auctor est. Nam veluti lac profusum ex uberibus saginat fetus, ità etiàm è gremio telluris fruges productæ mitia alimenta mortalibus præbent. Proindè non inconsultè inter alia fercula, quæ in pompa Isiaca à sacris Aegyptiorum antistibus præferebantur, vasculum papillæ simile ab Apuleio recensetur. Luna quippe partubus præesse creditur, quia illius ope, propter vim propriam humoris, fetus adiuuatur, et in utero intumescit. Per innumeras demùm mammas Naturæ adscriptas infinitum non hominum solummodo, verùm etiàm animantium, nec non vegetabilium numerum, quæ insimul alit, adumbrare haud dubiè voluerunt. Cum videlicèt penè totum Deæ corpus varijs animalium figuris refertum, et diversis fructuum generibus insignitum repræsentetur. Sed et Plutarchum audiamus referentem, quòd Lunare lumen generandi, atquè humectandi vim habeat; ac tam animalium fetibus, quàm plantarum pullulationibus conducat. Unde Apuleio sæpiùs laudato Luna, siue Ceres omniparens, quasi omnia pariens et gignens dicitur. Nec verò ab hac semita discedit Mantuanus poëta, dum dixit.
\begin{quote}
\emph{Nec non et Tityon terræ omniparentis alumnum.}\\
\emph{Cernere erat.}\\
\end{quote}
\vspace*{-8mm}
\paragraph{}
Enimuerò cùm omnia ab ea alantur, perinde atquè à nutrice et matre, signanter maternum nomen illi inditum est. Quò etiàm epitheto Luna ab Orpheo cohonestatur, apud quem modò φερέκαρπος, modò τελεσφόρος indigetatur. Addi, si graue non est, insignem Lucretij lib. 2. ad hanc rem locum.
\begin{quote}
\emph{Deniquè celesti sumus omnes semine oriundi:}\\
\emph{Omnibus ille idem pater est, unde alma liquentis}\\
\emph{Humorum guttas mater cùm terra recepit,}\\
\emph{Fœta parit nitidas fruges, arbustaquè læta,}\\
\emph{Et genus bumanum: parit omnia secla ferarum}\\
\emph{Pabula cum præbet, quibus omnes corpora pascunt}\\
\emph{Et dulcem ducunt vitam, prolemquè propagant:}\\
\emph{Quapropter meritò maternum nomen adepta est.}\\
\end{quote}
\vspace*{-8mm}
\paragraph{}
Luculentèr hoc identidem præfatus poëta et Philosophus ostendit alibi, ad quem lectorem remitto. Firmicus Maternus Mathem lib. 3. omnem subsantiam humani corporis ad potestatem lunaris luminis pertinere auctor est. Idemquè lib. 4. Lunam humanorum corporum matrem scientèr appellat. Luna etiam, eodem referente, terræ imperium ex vicinitate sortita, omnium animantium corpora et concepta procreat, et generata dissoluit. Diuus autem Augustinus de Ciuit. Dei lib. 7. Tellurem dici matrem asserit, quia plurima pariat; magnam, quia cibum cunctis subministret. Quæ omnia ambarum manuum porrectione iterum indicantur: liberalitas enim passis manibus olim figurabatur, ut scribit Diodorus Siculus lib. 4. Et has quidem ambas nobis porrigit, ut nos continuis beneficijs impleat, et iuuet in adversis. Verùm his obiter allatis ad alia emblemata, quibus reliquum Dianæ corpus stipatur, deflectere libet. Statuæ umbilicum tres cerui occupant, de quibus cum abundè supra dictum, Sphinges explicandas suscipiemus.
\clearpage
\subsection{Sphinges.}
\paragraph{}
Utrumque Deæ latus duæ Sphinges muniunt, quæ ab Ausoniana descriptione non discrepant: iuuat autem ipsummet vatem audire in ternario numero.
\begin{quote}
\emph{Terruit Aoniam volucris, leo, virgo triformis}\\
\emph{Sphinx, volucris pennis, pedibus fera, fronte puella.}\\
\end{quote}
\vspace*{-8mm}
\paragraph{}
Inter vetustissimos verò scriptores non defuerunt, qui caudam draconis illi adiunxerint: sed in illa non morabor: consule si lubet Herodotum, Aelianum, Plutarchum, Solinum, Diodorum, Plinium, Clementem Alexandrinum, et Palæphatum, qui caninum etiam illi caput assignant. Potes prætereà genuinam eius effigiem ex infinitis propemodùm monumentis antiquis observare. Hoc dicam, Sphingem Aegyptium fuisse inventum, quam ex virgine, atquè leone compositam finxerunt: hac præsertim ratione ducti, quòd Quintili et Sextili mensibus, sub quorum tutela Leo et Virgo sunt, Nilus exuberet, ut Bellonius observauit. Apud Aegyptios item constans fama est, Sphinges præ foribus ædium sacrarum Isidis et Osiridis tanquam silentij, taciturnitatis, ac prudentiæ symbola fuisse collocatas, ut Synesius orat. de prouid. memorat: quibus signis templa Deorum adeuntes, ut tacitè vota facerent, monebantur, nec arcana numinum divulgarent, sed mysteria sub his latitare præsentirent. Hanc autem Aegyptiorum doctrinam ad Ephesios transijsse ex germana Isidis, et Cereris cognatione autumare licet. Nam sicut Sphinx Memphitica, sacra ceremoniasque seclusas esse, et silentio tegi oportere innuebat; sic non absimili ratione Cereris initia occultanda, obsignandaquè esse apud Ephesios figura Sphingis Dianæ simulacro insculpta præmonebat. Quæ ut credibilia habeam, impellunt me tacita illa et operata Cereris apud omnes ferme gentes initia, et sacra: quæ nulla re magis quàm silentio constare Justinus memorat. Ad quæ alludens Apuleius scribit: \emph{Cætera, quæ silentio tegit Eleusis Atticæ sacrarium.} Porrò mysteria Eleusina dicebantur, quòd apud Eleusinos primùm inventa fuissent, in quibus silentij fides adhibenda erat. Μυστήρια, etenim ut Suidas interpretatur, ἀπὸ τοῦ μύειν 	γὸ στόμα dicebantur, ab ore videlicet claudendo: quòd deceat eos, qui audiunt divinas ceremonias, os obturare, nec ulli mortalium enunciare, ne oculis profanorum obuiæ polluantur. Initiationes etiam hæc Cereris mysteria dicebantur, ut Arnobius observauit: quia semper profanos arcebant, ut arbitris carerent. Omni exceptione maius erit hoc Horatij lib. 3. Car. super hac re testimonium.
\begin{quote}
\emph{Vetabo qui Cereris sacrum}\\
\emph{Vulgati arcanæ, sub ijsdem}\\
\emph{Sit trabibus, fragilemquè mecum}\\
\emph{Soluat faselum.}\\
\end{quote}
\vspace*{-8mm}
\paragraph{}
Ad cuius etiam mentem Tibullus canit.
\begin{quote}
\emph{Non ego tentaui nulli temeranda Deorum}\\
\hspace*{10mm}\emph{Audax laudandæ sacra docere Deæ.}\\
\end{quote}
\vspace*{-8mm}
\paragraph{}
Verùm si in sacris Cereris hæc religiosi arcani observatio tenebatur, Cybeleia sacra non minori cultu, aut silentio peragebantur. Quò spectant illa vatis,
\begin{quote}
\hspace*{15mm}\emph{hinc fida silentia sacris.}
\end{quote}
\vspace*{-4mm}
\paragraph{}
Cùm autèm nocturno tempore altum sit ubiquè silentium, hac de causa Cereri, ac Cybeli noctu sacra celebrabantur. Quanquàm iuxtà M. Varronis, et Divi Augustini sentenriam multa in Cerealibus mysterijs tradebantur ad fruges maximè pertinentia, quæ omnia peculiari taciturnitate obsignanda erant. Apuleius item Eleusina mysteria, lætificas messium cæremonias appellat. Sphinges porrò ex humano ferinoquè corpore constabant: facie plerumque muliebri, nonnumquam etiam viriti; quæ Androsphinges Herodoto dictæ. Quò symbolo non ineptè declaratur, Cereris beneficio ferinam illam et agrestem mortalium prisci æui barbariem in humaniorem, et urbaniorem vitam commutatam fuisse. Nam Ceres præterquàm quòd frumentum inuenit, eius etiam subigendi rationem edocuit; et leges dedit, quibus justitiam colere mortales didicerunt: his autem beneficijs nulla maiora reperiri possunt. Unde non sine causa ab Ovidio dictum est;
\begin{quote}
\emph{Prima dedit leges, Cereris sunt omnia munus.}
\end{quote}
\vspace*{-4mm}
\paragraph{}
Sed et alia ratio de huiusce animalis symbolo hìc expresso subesse poterit, si diligentiùs priscorum monumenta perpendantur. Aenigmata quippè et res abstrusas Sphinge designari nemo nescit. Naturæ autem arcana in tenebris iacere; obliquas asperasquè semitas esse, quæ mortales ad rerum naturalium intelligentiam introducunt, quis ignorat? Certè non oportet nos crassa pinguiquè Minerva abditas rerum causas naturæqué mysteria inquirere: Sed temporis diuturnitate subtilioriquè indagine opus est, ut quantulamcumque cognitionem referamus. Iam Sphingis symbola Gryphes excipiant.
\clearpage
\subsection{Gryphes, Dracones, Aliæque Feræ.}
\paragraph{}
Gryphem fictitium animal esse notum est, ideoquè in eius descriptione variant auctores. Pleriquè ex Aquila et Leone compositum volunt: Aristæus Proconnesius apud Pausaniam incuruum illi rostrum aquilarum more, auritumquè caput adscribit. Quin et Gryphem Aegyptium fuisse commentum vetustissima illa ænea Bembi tabula docet. In ea nempè gryphes multis varijsquè formis cælati apparent: quod indicium est, hoc animalis genus Isidi gratum acceptumquè fuisse. Quare non mirum, si Dianæ Ephesiæ attributi etiàm fuerint. Quamuìs non desunt qui inter peculiaria Solis symbola reponant: ob igneam præcipuè vim, qua præditum esse memorant: ut non inconcinnè etiam Lunæ, quæ Solis soror et coniux perhibetur, tribuantur, scilicet ob frequentem eius cum Sole coniuctionem, et à quò ignea fit. Iterum si ex Aquila et Leone gryphes compositi, Aquila rectè ad Solem, Leonis verò pars posterior (quòd et Macrobius, annotauit) quia imbecillior, ad Lunare sidus referri poterit. Sic igitur sub hoc figmento utriusque sideris socias vires, et effectus in omnibus rebus procreandis fouendisquè Ephesios prænotasse autumarim. Adhæc gryphes ob eximiam pernicitatem Lunæ competere videntur: quòd et de velocissimis alijs feris paulò ante referebamus. Post gryphes mystica sunt aliarum ferarum symbola, quæ Pantheras, vel leopardos, aut Tigres referre videntur; sed ob similitudinem quarumnam potissimùm sint, arduum est dignoscere: ideò his non immorabor, sed ad notiora pedem referam; præsertim cum in his nihil aliud quàm velocern lunæ cursum designari putem.
\clearpage
\subsection{Dracones.}
\paragraph{}
Sacrorum Ephesiorum præsides multa (ut præfati sumus) ab Aegyptijs mutuati, non mystico tantummodo Sphingis symbolo, verùm etiàm alijs primarium ciuitatis numen cohonestare volentes, geminos Dracones antiquissimo eius simulacro adiunxerunt. Nequè hoc sine recondito aliquo sensu factitatum crediderim: cùm inter cætera animantium genera, quæ summo in honore apud Aegyptios olim fuerunt. Dracones siuè Serpentes enumerentur; quorum effigies in Isiaca Bembi tabula non semel appictæ conspiciuntur. Ac ne priscorum vatum subsidio carere videamus, Ovidium dabimus, qui serpentem in pompa Isidis describit:
\begin{quote}
\emph{Plenaquè somniferis serpens peregrina venenis.}
\end{quote}
\vspace*{-4mm}
\paragraph{}
Juvenalis Sat. 6. etiam serpentes simulacro Isidis comitem inducit hoc versu:
\begin{quote}
\emph{Et mouisse caput visa est argentea serpens.}
\end{quote}
\vspace*{-4mm}
\paragraph{}
Tanta autem fuit apud priscos veneratione serpens, ut inter præcipua gentium mysteria decantaretur. Idipsum luculentèr Clemens Alexandrinus, Julius Firmicus, et Justinus Mart. ostendunt. Nec defuere, qui serpentibus divinitatem inesse dixerint: in quorum numero Taautes apud Eusebium, idquè ex Phœnicum Theologia. Quare cum ad Græcos eadem religio, vel potiùs superstitio translata fuerit, Ceres potissimùm serpentes elegit. Omninò Apuleiana Ceres draconum in pinnato curru conspicua incedere fingitur. Orpheus verò in hymnis Cererem in curru, Draconesque cam trahentes describit: quod splendidè Ovidius Fast. lib. 4. expressit, dum Cereris ex Sicilia profectionem describit.
\begin{quote}
\emph{Quò simul ac venit, frenatos curribus angues}\\
\hspace*{10mm}\emph{Iunxit, et æquoreas sicca pererrat aquas.}\\
\end{quote}
\vspace*{-8mm}
\paragraph{}
Tale quoquè est illud prælaudati vatis, dum Deam Celei Menolinæ hospitium deserentem recenset:
\begin{quote}
\emph{Dixit, et ingrediens nubem trahit, inquè dracones}\\
\hspace*{10mm}\emph{Transit, et alifero tollitur axe Ceres.}\\
\end{quote}
\vspace*{-8mm}
\paragraph{}
Non pauca marmorea monumenta eodem planè schemate, quo hìc depingitur, Cererem affictam referunt: causam verò huiusce figmenti inquirere conducit. Draconis symbolo terram significari credidere multi, quòd eam toto corpore penè perrepens verrere videatur. Sed quoniam non Cereris modò numen, verùm etiàm Dianæ, sub hac statua agnoscebatur, Diana ipsamet adeunda erit, ut rationes ob quas serpentes seu dracones in complexum receperit, inueniamus. Pausanias in Arcad. scripsit, inter alia signa, quæ in Dominæ siuè Heræ fano reponebantur, Dianæ unum repositum fuisse ceruina pelle velatæ, pendente ex humeris pharetra, altera manu lampadem, Dracones duos altera gestante. Naturæ numen prudentissimum fuisse ex veterum elogijs superiùs ostendimus. Prudentiæ enim symbolum Serpens, quem et insuper sapientiæ indicem vetustas esse voluit. Hunc præterea in delicijs fuisse ferunt propter peculiarem illam excubandi vigilandique vim, qua præditus est, et ob acutissimam oculorum aciem, quæ omnia Lunæ indefessæ competunt: siquidem suo splendore noctu cuncta collustrat. Per eius insuper eximiam celeritatem, flexuosos meatus, lapsusque erraticos, Lunæ iter sinuosum designari potest. Dracones demùm cornibus è fronte prominentibus insigniti; et hanc fortassis ob causam sub Dianæ tutela admissi fuerunt, quia omne cornigerorum genus Luna, uti iam dictum est, sibi vendicat.
\clearpage
\subsection{Boves.}
\paragraph{}
Peculiari quadam religione Boves Cereri, Lunæ, nec non et Isidi consecrauit antiquitas. Cereri quidem ob agriculturam, quia semina vix absque à ratione proveniunt. Et hoc unico symbolo telluris fœcunditas demonstratur: huiusque sub effigie à nonnullis gentibus, et præsertim à Phrygibus cultam fuisse ex priscis monumentis deprehendimus. Aegyptij quoquè hieroglyphicis literis, cùm terram significare volunt, ponunt bovis figuram. Dianam quinetiam in insula quadam sinus Persici tauri sub imagine summo in honore habitam fuisse Dionysius Alexandrinus memorat. Apud priscos verò tanta fuit huiusce animalis veneratio, ob summam illius in rusticis laboribus sustinendis patientiam ac tolerantiam, ut tàm capitale esset bovem aratorem necasse, quàm ciuem: cuius rei exemplum habeatur apud Plinum. Huius etiam encomium Varro lib. 2. de re rust. decantauit his verbis: \emph{bos socius hominum in rustico labore, et Cereris est minister}. Non absimilia ab Aeliano var. histor. lib. 5. referuntur, cuius hæc sunt: \emph{bovem aratorem, qui jugum trahit, vel in aratro, vel in palustro ne mactes: quia ille etiam agricola est, et humano generi laborum socius}. Pythagoras apud Ovidium mansuetam, cicuramque huius animalis naturam explicans in hæc verba prorupit:
\begin{quote}
\emph{Quid meruere boves, animal sine fraude doloque}\\
\emph{Innocuum, simplex, natum tolerare labores.}\\
\end{quote}
\vspace*{-8mm}
\paragraph{}
Hanc igitur ob causam in sacris Cereris interdictum erat ne bos adhiberetur, propter varia commoda, que mortalibus afferre solet. Iterùm Ovid. in Fast.
\begin{quote}
\emph{A bove succincti cultros remouete ministri.}\\
\hspace*{10mm}\emph{Bos aret, ignauam sacrificate suem.}\\
\emph{Apta jugo ceruix non est ferienda securi,}\\
\hspace*{10mm}\emph{Vivat, et in dura sæpe laboret humo.}\\
\end{quote}
\vspace*{-8mm}
\paragraph{}
Cur verò hoc animal Lunæ sit adscriptum, ante alios nobis rationem suggerit Luctatius Grammaticus his verbis. \emph{Luna verò quòd propiùs Tauro adhæreat, vacca idest bove figurabatur.} Eius autem exaltationem esse in Tauro Julius Firmicus, et Porphyrius testantur. Lunæ currui boves Ausonius ad Paulinum adsignat:
\begin{quote}
\emph{Iam succedentes quatiebat Luna iuuencos.} Et Claudianus:\\
\emph{Quo Phaëton irrorat equos, quo Lupa iuuencos.}\\
\end{quote}
\vspace*{-8mm}
\paragraph{}
In nummis antiquis Sept. Severi, Antonini Caracallæ eius filij, nec non et Juliæ Piæ Luna stat in biga, quam gemini boves trahunt. Bovis item cornua Lunæ præsertìm fuisse gratissima testatur Orpheus hymno Dianæ. Quin et ab illorum figura peculiare nomen sortita est ταυρωπὸς enim et ταυροπὸλος dicebatur, vel à boum cornibus, vel ab aspectu quo apparet, cùm corniculata est: aut certe quia id animal terræ culturæ deputatum est. Boves item apud Aegyptios Isidi in delicijs præcipuè fuerunt, uti monuit Herodotus, referens illis morem fuisse boves mares Osiridi, fœminas verò Isidi dedicare: cuis auctoritati Diodorus Siculus adstipulatur. Inter alia autem antiquitatis analecta, quæ in Musæo meo asseruantur, Isis ex nigro lapide effigiata visitur, supra cuius caput bovina cornua assurgunt, quibus sumnia retum abundantia, et fertilitas designatur: nam secundum Salomonis 14. Proverb. sententiam, \emph{Ubi plurimæ segetes, ibi manifesta est fortitudo bovis}. In apparatu sacrificali, seu pompa Isiaca, bos inter alia fercula à Protomystis præferebatur, ut in marmore perueteri observavi. Dianæ porrò apud Romanos, non fecùs quàm Isidi apud Aegyptios, boum cornua sacrata deprehendimus. Hi etenim in vestibulo Templi Dianæ Montis Aventini ex vetustissimo ritu boum cornua affigebant. Quid si omnes hasce Deas bove gaudere diximus? Numquid etiam in ipsius Naturæ parentis delicijs esse porerunt? Certè quidem: nam ad illam omnium ciborum, et alimentorum, quibus mortales suftentantur, officium pertinere antiqui putarunt. Bovi quippè nomen à nutriendo factum volunt. BΩ enim nutrio sonat: si quidem labore suo in terra exercenda nos continuo nutrit. Vetus autem Glossa super Exod. cap. 22. beneficia, quæ à bove percipimus, enumerans: \emph{Bos} (inquit) \emph{inprimis immolatur, arat, pascit carnibus, lac dat, et corium diversis usibus ministrat}. Ad hæc multiplicia commoda Hesiodus respiciens bovem inter familiæ partes reposuit. Is enim libro, qui inscribitur Opera et Dies, domum ex viro, muliere, et bove aratore constare indicauit. Quàm verò appositè Ephesij boves laboriosissimos utilissimis Apibus associarint, in hoc marmoreo Naturæ simulacra mox ex sequentibus innotescet.
\clearpage
\subsection{Apes.}
\paragraph{}
Solæ ex volatilium, et infectorum genere Apes Dianæ consecrari, ipsiusquè corpus symbolico argumento cohonestare meruerunt. Et hæ quidem ternæ utrimquè effictæ, ac rosis immixtæ latera Deæ muniunt. Nec immmeritò sane: cùm juxta præceptum Plinij rosæ juxta alvearia sint collocandæ. Cur vero Dianæ, Cereri, Terræ, et rerum parenti Naturæ in delicijs fuere? An cùm virginitatis cultrix sit Diana, Apes ei adscripte ob incorruptam virginitatem? Castas profecto Apes Petr. Damianus epist. 15. lib. 1. describit, cum inquit: \emph{Apes sobolem successuræposteritatis enutriunt, ut virgines perseverent.} Quin et naturali quodam instinctu res Venereas adeò aversantur, ut eos, qui recèns operam Veneri dederunt, acriùs inuadant, quod Plutarchus in coniugalibus præceptis annotauit. Consona etiam sunt illa, quæ de integritate earum virginali cecinit Virgilius Georg. lib. 4.
\begin{quote}
\emph{Illum adeo placuisse apibus mirabere morem,}\\
\emph{Quod nec concubitu indulgent, nec corpora segnes}\\
\emph{In Venerem soluunt, aut fætus nixibus edunt.}\\
\end{quote}
\vspace*{-8mm}
\paragraph{}
Quod ipsum luculentèr Quintilianus in declamatione, quæ \emph{Apes pauperis} inscribitur, ostendit his verbis: \emph{Non illas libido progenerat, domitrixq; omnium animalium Venus}. Eucherius autem Apem virginitatis symbolum esse tradit. Vide, si lubet, Albertum Magnum lib. 17. de Antract. 2. cap. 2. et miram Apum castitatem agnosces. Quare cùm Diana summoperè virginum moribus, ac consortio delectetur, non mirùm si purissimas Apiculas in complexum tanquam aptissimas. comites receperit. An non Dianæ exemplo Apes in montibus degunt, et locis ab hominum consuetudine remotis gaudent? Aliam causam habe: Apes rorem ex floribus legunt, favos inde conficiunt; non ineptè igitur Dianæ seu Lunæ alumnæ dicuntur: cum Aleman poëta Lyricus, vel ipso Macrobio Satur. lib. 7. teste, rorem aëris ac Lunæ filium esse dixerit: et peculiari encomio Luna roscida nominetur. Aliud occurrit, quod hìc attexam; Apes propter eximiam velocitatem Lunæ tribui: scitè enim ab Ovidio agiles nuncupantur. Verùm libetne, ut quid iam Apis cum Cerere peculiare habeat, disquiramus? Porphyrius in antro Nympharum Apes Cereris ministras statuit: numquid ob puritatem? hoc equidem sentire Pindarus in Pyth. videtur: cuius interpres ad hæc verba,
\begin{quote}
Οἳ δὲ ἐν τούτῳ λόγῳ χρησμὸς ὄρθησε μελίατας.
\end{quote}
\vspace*{-4mm}
\paragraph{}
\emph{Propriè quidem}, ait, \emph{Apes Cereris ministras appellauit propter earum puritatem: impropriè verò omnes alias animantes.} Idem Pindarus alibi Cererem in solemnibus sacris Apibus delectari scribit: quo factum crediderim, ut cùm Ceres maximè à sanguine abhorreat, mella inter alia libamina sibi asciuerit. Pausanias in Eliac. author est, Eleos cùm in Prytaneo monte Cereri sacra facerent, thure et melle subactum triticum adoleuisse. Idemquè in Arcadicis refert, Cereri Phigaliensi favos fuisse oblatos. Et hoc instituti genus ad Romanos usquè transijt: nam quotiescumquè Ambaruale sacrum pro maturis frugibus fiebat, faui cum libamentis offerebantur. Adeò ad impetrandas fruges Apes valere credebant. Quò mirè facit quod de Trophonij Oraculi origine Pausanias in Beotic. prodidit, cuius verba adscribere non pigebit: \emph{Cum unum}, inquit, \emph{et alterum annum nullis terra imbribus irrigaretur, è singulis civitatibus Delphos missi sunt, qui opem implorarent. His siccitatis remedium exposcentibus Pythius Apollo imperauit, ut Lebadeam venientes à Trophonio mali auxilium quærerent. Lebadeam itaquè profecti, cum Oraculum reperire non possent, Saon quidam Acrephniensis, collegarum natu maximus, cum Apum examen conspexisset, quocumquè illæ divertissent, sequi statuit. Ubi igitur eas ad speciem quandam aduolantes vidit, et ipse subiens oraculum, quod quærebant, illud esse intellexit.} Quo invento, quæ iamdudum inculta, et squallens terra iacuerat, imbribus tandem saturata brevi florida, ac frugifera fuit. Quonam tandem jure Apes Cybelem attingant, videamus. Papauer, ut Eusebius monet, Civitatis est symbolum, ob multitudinem scilicet granorum, quæ in eo, tanquam in civitate ciues stipantur. Pari ratione Apum alueare civitatis seu reipublicæ congruum potest argumentum constitui. Apum quippe labor, et vita indefessa, nihil aliud est quàm exemplar reipub. optimis legibus, et civili disciplina institutæ. Quin ut hominum frequentia conservantur urbes, sic in aluearibus Apes: adeò nempè frequentiæ, et societatis amatrix est Apis, (quod Mausonius annotauit) ut si sola relinquatur, desiderio tabescat, et moriatur: quod etiam innuit Cicero Offic. lib. 1. \emph{Apum examina non fingendorum causa congregantur, sed cùm congregabilia natura sint, fingunt favos}. Igitur sicut inter ciues officia reipublicæ dividuntur, sic etiam Apes inter se labores, et opera in mellificando partiuntur: nulliquè licet otiari, sed ignauos abs se propellunt. Tangit hanc earum naturam Virgilius:
\begin{quote}
\emph{Ignauum fucos pecus à præsepibus arcent.}
\end{quote}
\vspace*{-4mm}
\paragraph{}
Non immerito igitur sub Cybeles famulatu Apes stauendæ, sub cuius tutela urbes. Addatur et istud, Cybelæ et Cereri æquè communia fuisse cymbala: at quis nescit Apes dissipatas, et displicatas, ut loquitur Varro de re rust. lib. 3. cymbalis et plausibus in unum locum reduci? hæc sciens Claudianus cecinit.
\begin{quote}
\hspace*{15mm}\emph{qualis Cybeleia quassans}\\
\emph{Hyblæus procul æra senex revocare fugaces}\\
\emph{Tinnitu conatur Apes}\\
\end{quote}
\vspace*{-8mm}
\paragraph{}
Apibus igitur et Cybelæ æra communia, sed et Lunæ quoquè: consueuerant quippe ueteres delinquente Luna æra pulsare: cuius moris meminit Manilias Astron. lib. 1.
\begin{quote}
\emph{Ultima ad Hesperios infectis volveris alis,}\\
\emph{Seraquè in extremis quatiuntur gentibus æra.}\\
\end{quote}
\vspace*{-8mm}
\paragraph{}
Ceres non minùs digno quàm proprio elogio Munifica dicebatur: ita etiam Apis munifica nuncupanda, gratis enim nobis mellificat, et sine impendio labores suos mortalibus elargitur. Deniquè ipsius Telluris feracitas ubertasquè non potuisset alio melius, quàm Apis symbolo subintelligi. Verùm offert se præcipua quædam mihi ratio, cur Ephesinæ Dianæ simulachro Apes insculptæ fuerint, eamque insinuat Philostratus his verbis; \foreignlanguage{greek}{Τί οὖν αἱ Μοῦσαι δεῦρο; τί δὲ ἐπὶ ταῖς πηγαῖς οὖ Μέλητος; Α᾽θηναῖος τὴν Ι᾽ωνίαν ὅτε ἀπῴκιζον, Μοῦσαι ἡγοῦντο τοῦ ναυτικεοῦ ἐν ἔιδει μελιττων}. \emph{Quidnam huc Musæ? Quid ad Melitis fontes? Cum Athenienses Ionicas colonias ducerent, Musæ classis duces fuere, sub Apum specie.} Quod et ante Philostratum Himerius Sophista annotarat, ut à Leone Allatio linguæ Græcæ peritissimo accepi. Quo factum ut in huius beneficij recordationem Ionicarum Urbium princeps Ephesus has etiam Apes, principis sui numinis simulacro insculpserit. Hæc de convenientia symbolorum satis: nunc earundem opportunitates explicabimus.
\clearpage
\subsection{Apum et Mellis Commoda.}
\paragraph{}
Placetne ut Apis dotes et mellis commoda perscrutemur? Naturam veluti matrem providam, et totius generis humani susceptricem cum multis mammis supra produximus. Verùm cùm ipsius lactis usus ad reliquam hominis vitam educandam sufficere minimè posset, indulgenter varia fructuum genera subministrauit, ut faciliùs mortales sustentarentur. Ad hæc autem exoptatissima Naturæ dona sagax Ephesiorum religio respiciens, non sine mente cum infinitis illis papillis uberrimam fructuum copiam, quibus sertum lemniscatum refertum conspicimus, velut ex eius sinu productam mammis associauit. Cùm autem sine lactis usu, et frugibus hominis vitam posse sustentari Ephesij cognouissent; beneficio videlicet ac edulio mellis; ideò Apes Naturæ simulacro tanquam ministras hominibus necessarias adiunxerunt. Ipsemet Jupiter ante omnes, Antonino Liberali teste, Apum benificentiam sensit; illi quippe in antro Cretensi vagienti Apes mella tanquam divinum nectar pro alimento dederunt. Sed convenientius sit ex Lactantio lib. 1. cap. 2. dicere, Melissam regis Cretensis filiam unà cum sorore Amalthea Jovem puerum educasse; illam scilicet melle, hanc verò lacte caprino. Post Jovem Beroë quoquè mellis alimonio nutrita: sic refert Nonnus Dionys. At enimuerò non uni Jovi, nec soli Beroæ mellis edulia communia: verùm etiam primitivus mos erat, ut peculiariter hoc cibo pueri pascerentur. Ideò Moschius jubet, ut nutrix os infantis melle illinat. Apud Ovidium Fast. lib. 4. etiam Triptolomo infanti lac, poma, et mella edenda dantur.
\begin{quote}
\emph{Mox epulas ponunt, liquefacta coagula lacte,}\\
\hspace*{10mm}\emph{Pomaquè et in ceris aurea mella suis.}\\
\end{quote}
\vspace*{-8mm}
\paragraph{}
Hieroni autem, cùm iussu patris expositus fuisset, quòd ex ancilla natus crederetur, humanæ opis indigenti Apes in os mella congessere: quo ostento pater ab haruspicibus admonitus filium recipiendum duxit, moribus, et disciplinis illis erudiendum curauit, quibus postmodum ad præsignatam regni maiestatem facile peruenit. Quarè Apis eodem nutricis epitheto decorari potest, quo Diana ab Orpheo cohonestatur; utraque enim παιδοονρόφος, hoc est, puerorum nutrix est, quo etiam nomine Diana apud Diodorum vocatur: quòd scilicet ad eam infantiam, et ciborum officium pertinere veteres existimarent. Attexatur et illud quoquè, Pindarum, cùm è paterna domo pulsus et expositus fuisset, ab Apibus educatum, teste Aelian. de var. hist. Et eius filias paterno præsidio et opibus carentes è benigna Veneris manu casei, et mellis edulia suscepisse.

Verùm enimuerò hoc pretiosi nectaris alimentum non infantibus primulæ ætatis tantummodò exhibitum , quinimo grandæuis, et longæuis etiam hominibus pergratum salutarequè fuisse priscorum paginæ nobis suggerunt. Huiusce cœlestis liquoris meminit et ipse Homerus, apud quem Machaoni, Nestoriquè annoso, mel et cepa tanquam lautissimæ dapes usui fuerunt; in quorum schola Nestoris uxor enutrita, Patroclo cepam cum melle recenti aurea in lance apposuit. Hìc etiam audire iuuat Possidonium referentem, Mysios religioni indulgentes ab animantibus abstinuisse, et eam ob causam pecus omne vitasse; melle verò tantum, et lacte vitam sustentasse. Prædictis calculum suum adijcit Eustathius in Iliad. 10. et frequentissimum ueteribus fuisse mellis usum refert, Pythagoræque exemplum adducit, qui solo melle, tanquam divino pharmaco, contentus diù vixit. Similia prodidit Aristoxenus philosophus et musicus nobilissimus, qui Pythagoræ sectatores ut plurimum melle usos enarrat. Hùc etiam traho illud Athenæi de Democrito, cui non modo mellis suffitus, verùm etiam ipsius vapor naribus admotus tàm salutare beneficium contulit, ut per multos annos sanus, et incolumis permanserit: unde interrogatus, quonam pacto diù quis sanus esse posset; Si interna, inquit, melle rigaris, oleo autem extima. Zenoni verò philosopho non cupediarum nobiles illi artifices edulia parabant, sed modico pane, et melle mensam onerare studebant, ut in eius vita Laërtius retulit. Sic etiam educati fuere Antiochus medicus, et Romulus Pollio, qui scientiam Apicianæ popinæ explodentes, melleis alimentis absquè alijs ferculis in ultimam usquè senectam vitam protraxerunt. Cæterùm valeat apud nos Diophantis authoritas, qui de re rustica librum conscripsit, asserentis in longa æui spatia vitam eos producere, qui continuo mellis esu uterentur. Hanc verò unicam ob causam Cyrnios longæuos dici volunt, ut ex Athenæo observare licet, quòd assidua mellis esitatione delectati fuerint. Potest ex præfatis hoc pretioso nectare hominis vita absquè ullo alio cibo sustentari. Audi præterea, si grave non est, ipsummet Hippocratem medicinæ parentem, de melle disserentem. \emph{Vinum et mel optima judicata sunt hominibus, si juxta naturam et sanis, et debilibus cum temporis oportunitate, ac mediocritate exhibeantur}: tùm alibi, \emph{mel cum alijs quidem comestum et nutrit, et bonum colorem exhibet}. Eadem Celsi mens suit, atquè adeò non paruum nutrimentum humanis corporibus à melle conferri scripsit; quæ ne temerè, et sine ratione dicta videantur, age sis mellis proprietates expendamus.
\clearpage
\subsection{Mellis Vires et Proprietates.}
\paragraph{}
Principiò quidem mellita medicamenta, diviniq; huius nectaris potiones quantum in medicina valeant, ipsimet medicinæ parentes Hippocrates et Galenus abbundè prodidere. Quæ Plinium etiam non latuere: tradidit quippe mel conservandi facultate pollere, ulcera vetusta expurgare: quod et Porphyrius significauit. Apuleius autem Platonicus in libello, quem de viribus herbarum contexuit, mirificas mellis virtutes in varijs ægritudinibus, morborumq; curationibus refert, observatque liquorem illum cum herbarum succis et mandragora mixtù saluberrimum esse. Porro inter præcipua remedia, quæ ebrijs medici olim adhibere solebant, panem melle oblitum Macrobius Satur. lib. 1. c. 7. recensuit. Mel quoq; desiccandi vim habet, firmatque vulnera ut idem autor Sat. lib. 7. cap. 12. observauit, his verbis: \emph{Namquæ udanda sunt, corporis, vino foventur; quæ siccanda sunt, melle deterguntur}. Cum potiones amaræ præbendæ olim erant, poculi ora melle illinebantur. Possem infinita alia recensere; sed longioris id operæ, et aberramenti prolixioris foret. Addam tantummodò, corpora humana, non solùm dum viva sunt, Apum muneribus recreari, et iuuari; sed postquam etiam anima destituta sunt, beneficio mellis seruari. Constat etenim, defunctos in melle sitos à fœtore et putredine defendi. Valet inprimis apud nos Xenophontis Ελληνικῶν lib. 5. autoritas, asserentis Agesipolin Lacedemonium graui morbo aftectum, eoque die septimo consumptum, in melle positum, et Lacedæmonem reuectum, regiaque sepultura fuisse donatum. Statius in Siluis scribit Alexandri Magni cadauer melle litum putredini non fuisse obnoxium. Plinius etiam refert Hippocentaurum Claudio Cæsari ex Aegypto illatum in melle conditum fuisse, ex more rituque illius gentis: qui mos Babylonijs etiam et Assyrijs familiaris fuit. Quam consuetudinem non parùm iuuant illa Varronis περὶ ταφῆς apud Nonium in nomine, Vulgus: \emph{Quare}, inquit ille, \emph{Heraclides Ponticus plus sapit, qui præcepit ut comburerent, quàm Democritus, qui ut in melle servarent: quem si vulgus secutus esset, peream si centum denarijs calicem mulsi emere possimus.}

Verùm ne insipida quæ protulimus esse videantur, paucula tanquam bellaria apponam, quæ ad victus lautitiam, epularumque magnificentiam spectant. Frequentissimum apud veteres fuisse mellis usum in obsonijs, et condimentis vulgatæ autorum paginæ abundè testantur. Apicius insignis ille Opsophagus ante omnes idipsum non paucis in locis significauit, quem vide, si animus est tibi ista pernoscere. Ad parandas certè in lautioribus conuiuijs delitias, bellaria mellita excogitata, ac in secundis mensis apposita. Sic Varro apud Aul. Gellium lib. 13. cap. 11. bellaria secundæ mensæ mellita esse debere præcipit. Quò trahi possunt illa dulciarij pistoris apud Martialem verba.
\begin{quote}
\emph{Mille tibi dulces operum manus ista figuras}\\
\hspace*{10mm}\emph{Extruit, huic uni parca laborat Apis.}\\
\end{quote}
\vspace*{-8mm}
\paragraph{}
Placentæ etiam mellitæ auidè inter conuiuas expetitæ: unde Horatius epist. 2.
\begin{quote}
\emph{Pane egeo, iam mellitis potiore placentis.}
\end{quote}
\vspace*{-4mm}
\paragraph{}
Placentæ ut plurimum ex farina hordacea, caseo, et melle simul commixtis parabantut. Edulia autem illa, quæ in melle cocta sunt, maximè concoctionem in ventriculo iuuare docet Plinius lib. 20. cap. 9. Idemque alibi refert semen tritum candidi papaueris cum melle in secundis mensis apponi solitum: plura prætermitto, quàm describo. Ad vinum melle temperatum venio, quod vini genus in lautioribus conuiuijs olim expetitum: Unde est illud notum Martialis in Xenijs.
\begin{quote}
\emph{Attica nectareum turbatis mella falernum,}\\
\hspace*{10mm}\emph{Misceri docet hoc à Ganymede merum.}\\
\end{quote}
\vspace*{-8mm}
\paragraph{}
Primus omnium mortalium Aristæus mella vino miscuit; miscendi autem rationem Plinius lib. 14. cap. 90. de re rust. Columnella edocent. Sed quoniam hæc potiùs gulæ irritamenta, quàm res humano generi necessarias respicere videntur, ad alia luculenta Apum munera, quæ in vitæ humanæ subsidium munificè concessa sunt, pedem referre lubet.

Ceræ non infrequentiores, quàm mellis usus: hac Deorum, heroumq; imagines formabantur; exornabantur atria: hinc maiorum vultus ducti; accensi Deorum funales; Pictorum tabulæ inceratæ: hinc obserata vulnera: unde Valerij Flacci versus,
\begin{quote}
\emph{Vel pice, vel molli concludere vulnera cera.}
\end{quote}
\vspace*{-4mm}
\paragraph{}
Pugillares seu tabellæ ceratæ, in quibus stilo æneo, aut ferreo literas exarabant. Plautus in Asinaria.
\begin{quote}
\emph{Nec ulla sit cera, ubi facere possit literas.}
\end{quote}
\vspace*{-4mm}
\paragraph{}
Præter allata Apum munera Propolis potest adiungi: sed ne longior sim, ad Plinium lib. 11. cap. 7. te remitto: qui rectè dixit Apes hominis causa genitas esse; nam quicquid laboriosissima Apum familia operatur, in communem hominum utilitatem cedit. Ex dictis liquere satìs puto, cur Dianæ seu Naturæ sacræ Apes. Istud tantum corollarij loco attexam, in sanctiore Eminentissimi Cardinalis Francisci Barbarini Ceimeliarchio gemmam videri in qua tres Apes cum aratro expressæ, quarum duæ junctis sub jugum ceruicibus terram proscindunt; altera verò coloni vice fungens stimulo comites excitat. Quo hieroglyphico tanquam muta quadam poësi animos segnes non solùm ad labores incitare, verùm etiàm tacitis religionis Eleusinæ præceptis instituere prisci sine dubio voluerunt.
\clearpage
\subsection{Rosæ.}
\paragraph{}
Rosæ binæ hinc atquè hinc dispositæ, solertis ingenij arbitrio Apibus intermixtæ fuerunt, ut omnium florum fructuumquè primitias designarent: flores certe fructus promittunt: rosæ præsertim, quas poëtæ indiscriminatim pro quibuslibet ferè floribus ponunt. Horum vestigijs Dianæ opifex inhærens rosam illi tribuit, credo ad terræ fertilitatem designandam. Ac licet rosas tanquam sibi proprias Venus vendicet, non obstat tamen quin et has Cereri adjudicemus. Percelebris enim est illa vestis Cereris apud Apuleium ex varijs florum generibus contexta. Rosacea item illa corona, de qua plenius supra disseruimus. Quid si et hic rosarum ornatus memoriam raptus Proserpinæ inculcet? Ferunt quippè eam Siculis in campis rosas legentem à Plutone raptam. Ut ut sit, Diana rosis ornata certè inducitur, nempè quòd earum calices nocturno tempore ut plurimùm explicari soleant. In pompa vero Magnæ matris rosæ ante Deæ simulacrum spargebantur: ut à Lucretio lib. 2. traditum observavimus.
\begin{quote}
\emph{Aere atquè argento sternunt iter omne viarum,}\\
\emph{Largifera stipe ditantes, pinguntque rosarum}\\
\emph{Floribus, umbrantes matrem, comitumque cateruas.}\\
\end{quote}
\vspace*{-8mm}
\paragraph{}
Pertinent igitur ad Cybelem et Lunam rosæ: nec immeritò, cum florum Reginæ sint, omnem vegetantem naturam, quæ succum et alimentum è terra ducit, adumbrare creduntur.
\clearpage
\subsection{Fasciæ.}
\paragraph{}
Inferiora statuæ tribus reuincta sunt fascijs, non dubiè quin ex Aegyptiorum disciplina, prout ex priscis illorum monumentis ac etiam ex hieroglyphica Bembi tabula liquet; qui Isidem versicoloribus vittis toto corpore vinctam adumbrabant. His autem vittis Lunæ in elementorum coagmentatione varias operationes, et effectus mysticè designabant: vel certè, quæ Heliodori sententia est, varias Lunæ facies, et aspectus indicabant. Possis et circulos illos seu coronas, quibus Luna sæpe cingitur, in fascijs intelligere. Aut fortè per has innuere voluerunt, res quasque è terra natas cursu temporis rursum in terram redire. Luna Verò, ut supra annotauimus, vitæ mortisquè dominium habere credebatur. Quod si velis has tænias ad Cererem referre, non pugnabo, cum hæc omniparens, et commune rerum omnium sit sepulchrum. Ideoquè corpus eius fascijs obligatum fingi potuit, ut innueretur, semina medio anni spatio sub terra veluti in utero Cereris occuli, indeque per occultam vim reddi cum fœnore.

Soli pedes lustrandi supersunt. Sunt autem ex marmore nigro in se contracti, et eo modo compositi, quo omnium ferme Deorum simulacra Aegytios formasse refert Herodotus. Ocreati non sunt ut in alijs Dianæ statuis, sed nudi, impedimentisquè soluti ut præcipuè tanti numinis propensa voluntas, maternique affectus erga mortales dignoscerentur.

Hæc fere dicenda habui de his symbolis, et hieroglyphicis: nec diutiùs in ijs immorabor: cùm liquido iam constare credam, hæc omnia ab Ephesiorum hierophantis præsertim electa fuisse, ut latentes rerum naturalium causas, sub his quasi sub arcanis inuolucris obtegerent, honestoq; philosophiæ obtentu velarent: quo faciliùs cæcutientibus imperiti vulgi oculis imponerent. Sive ut per hanc figurarum symbolorumque varietatem, Dianam suam augustiorem redderent; sub cuius effigie Naturæ rerum omnium procreatricis numen, ut in limine operis ex Diuo Hieronymo attigimus, contineri volebant. Porro Naturam ipsam non modò Ephesij, verùm etiam pleræque aliæ gentes tanquam Deam divinis honoribus affecerunt. Et Macrobius refert Isim supremum Aegyptiorum numen, nihil aliud esse, quàm terram, naturamue rerum. Epicurus verò mundi huius gubernatorem solam naturam statuebat: ut ex Minucio Felice in Oct. colligere est: \emph{Etiam} (inquit) \emph{Epicurus ille, qui Deos otiosos fingit, Naturam tamen superponit}. Eadem quoque mens fuit Senecæ lib. 4. de benef. qui Naturæ divinitatem tribuit his verbis. \emph{Natura hoc mibi præstat; non intelligis te, cùm hoc dicis, mutare nomen Deo? Quid enim aliud est Natura, quam Deus, et divina ratio toti mundo, et partibus eius inserta?} Hùc trahi posset Lactantij auctoritas, Naturam pro Deo primitus habitam declarantis: consule eum, si lubet, aut si plura Naturæ encomia desideras, hymnos qui sub Orphei nomine circumferuntur, adi. Ceterùm tanta fuit Dianæ Ephesiæ gloria, tanta religio, ut non à finitimis tantummodò, verùm etiàm à remotissimis populis eius effigies peculiari studio coleretur: ità refert Strabo, qui et in extremitate Ferrarij Hispanie promontorij fanum Dianæ Ephesie fuisse prodidit. Pausanias autem multis in locis cultam eam fuisse observauit: scribit enim de Corintho: \emph{In foro, ubi plurima sunt templa, Diana est Ephesia cognomento.} Eodem quoque teste in porticu Megalopolitana signum Dianæ Ephesiæ positum fuit. In Arcadicis Templum Dianæ Ephesiæ apud Eleam Arcadiæ oppidum extare retulit. Massiliæ quoquè Justinus hanc Deam adoratam fuisse memorie prodidit. Demùm non pauce eius marmoreæ statuæ, quæ hodie Romæ visuntur, satis superque indicant, hanc peculiari cultu Romanos fuisse prosequutos. Verùm absolutis tandem celeberrimi huiusce simulacri symbolis et partibus, operæ pretium facturum me puto si pauca de templo, ubi signum illud asseruabatur, retulero.
\clearpage
\subsection{Templum Dianæ Ephesiæ.}
\paragraph{}
Ephesus, ut vult Stephanus, Ioniæ est urbs clarissima; Lydie, ut Herodotus: Asiæ certè lumen Plinio dicta est. In ea Dianæ templum nobilissimum, quod cœteris pulchritudine, et varietate marmorum, columnarum numero, ac ipsius etiam structure concinnitate, nec non et copiosa supellectile prelatum fuit, atque inter orbis miracula annumeratum. Placet igitur cius quantulamcumque adumbrationem ex Plinio lib. 36. cap. 19. exhibere. \emph{Magnificentiæ} (inquit) \emph{vera admiratio extat templum Dianæ Ephesiæ ducentis, vel ut alij volunt, quadringentis viginti annis factum à tota Asia: in solo id palustri fecere, ne terræ motus sentiret, aut hiatus timeret. Rursum ne in lubrico atque instabili fundamenta tantæ molis locarentur, calcatis ea substrauere carbonibus, dein velleribus lanæ. Universo templo est longitudo quadringentorum vigintiquinque pedum, latitudo, ducentorum viginti. Columnæ centum viginti septem à singulis Regibus factæ; sexaginta pedum altitudine, ex ijs triginta sex cælatæ.} Quales fuerint trabes, ex Vitruuio suprà indicauimus. Valuæ autem ex cupresso compacte, quæ et ad quatuor usque ætates intacte permanserunt, ut Theophrastus hist. plant. lib. 5. annotauit. Cœterum tanta fuit templi maiestas, ut ansam Plinio dederit asserendi, \emph{Cœtera eius opiris ornamenta, plurium librorum instar obtinere}. De primis eius auctoribus inter se non conveniunt scriptores. Livius à civitatibus totius Asiæ factum ex vulgari fama refert. Alij maluere ab Amazonibus extructum: quò faciunt versus illi Nonni in Dionys. quos in limine sermonis adduximus. Pindarus apud Pausaniam in Achaicis idem fatetur, atque adeò eleganti carmine Amazonibus bellum Theseo inferentibus hoc opus assignat. Id etiam Hyginus insinuat, et addit insuper, ab Otrita Amazone Martis coniuge dedicatum. In ea quoque mente Pimandri interpres, Pomponius Mela, et Solinus lib. 1. cap. 41. quibus tamen neutiquam consentit Pausanias in Achaicis: sed Eræsum hominem indigenam, Ephesumque Caystri fluminis filium tantæ molis auctores facit: quin et ab Epheso urbem denominatam asserit. Quocumque auctore excitatum fuerit, tantæ fuit magnificentiæ apud priscos, ut inter septem orbis miracula adnumeratum, non ultimum locum habuerit. Antipater illud spectaculis omnibus anteponere non dubitauit. Eadem fiducia Hyginus primum illi locum inter orbis miracula tribuit. Non me tamen fugit, alios esse qui secundum illi locum assignent. Callimachus poëta illustris in hymno Dianæ, nihil præcellentius et admirabilius à Sole oriente conspici dixit. Sed omnium elegantissimè huius Templi augustam magnitudinem depingit Philo Byzantius antiquissimus scriptor libello de 7. orbis miraculis, cuius locum harum elegantiarum studiosi unà mecum debebunt V. Cl. Lucæ Holstenio, qui eum libellum iam dudum cum alijs editioni paratum servat: unde hæc interim amicissimi viri beneficio cum eiusdem versione Latina profero.
\clearpage
\begin{center}
\foreignlanguage{greek}{
ΘΕΑΜΑ ΠΙ.
}
\end{center}
\begin{center}
\foreignlanguage{greek}{
Ο῾ἐν Ε᾽Φέσῳ ναὸς τῆς Α᾽ρτέμιδος.
}
\end{center}
\begin{quote}
\foreignlanguage{greek}{
Ο῾τῆς Α᾽ρτέμιδος ναὸς ἐν Ε᾽Φέσω μόνος ἐστὶ Θεῶν οἶκος • πεισθήσεται γὲρ ὁ θεασάίμευος τὸν τόπον ἐνηλλάχθαι, νὶλ τὸν οὐράνιου τῆς ἀθανασίας κόσμον ἐπὶ γῆς ἀπηχθεῖσθς. Γίγαντες γὰρ οἱ των Α᾽λωέως παίδων οἱ τὴν εἰς οὐρανὸν ἀνάβασιν ἐργάσαντο, ὄερσι χωννύοντες τὸν οὐ ναὸν, ἀλλ᾽ Ο῎λυμπον • ὥστε τῆς μεὺ ἐπιβολῆς τολμηρότερον εἶναι τὸν πόνον, τοῦ πὸνον δὲ τὴν τέχνκν • γὸ γὰρ ἔδαΦος τῆς ὑποκειμεύης γῆς λύσας ὁ τεχνίτης, καὶ τὰ βάθη των ὀρυγμάτων καταβιβάσας, εἰς ἄπειρον ἐβάλετο τὸν κατώρυγα θεμεγίωσιν, ὀρῶν λατομίας δαπανηίσας εἰς τὰ κατὰ γηῦ καλυπτόμευα των ἔργων • Ε᾽ρείσας δὲ τὴν ἀσφάλειαν ἀσάλευτον, καὶ προυποθεὶς τὸν Α῎τλαντα Τοῖς βάερσι των μεγλόντων ἐπαπερείδεσθαι, πρῶτον μεὺ ἔξωθεν ἐβάλετο κρηιοπίδα δεκάβαθμον διεγείρων πρὸς βάσιν μετεωροφανές, καὶ περὶ ... λείπει.
}
\end{quote}
\clearpage
\subsection{Vi Miraculum. Templum Dianæ Ephesiæ.}
\paragraph{}
\emph{Dianæ Ephesiæ templum unicum est Deorum domicilium: quisquis enim spectauerit, credet permutatis inuicem locis mundum cælestem immortalium deorum in terras demigrasse. Nam Gigantes, vel Aloidæ cælum conscendere aggressi aggestis montibus non templum, sed Olympum struxere: ita ut labor inceptum, ars laborem audacia superet. Artifex enim dimoto solo, quod suberat, actisque in immensam profunditatem fossis fundamenta altioribus cuniculis iecit: ita ut montium lapicidinas operibus subterraneis exbauriret. Sed strato inconcussæ soliditatis firmamento, et præsupposito Atlante ad superincubituri operis pondera sustinenda, principio quidem crepidinem decem graduum extrinsecus posuit, quæ basis eleuatioris vice fungeretur.}

Reliqua deinde huius capitis perierunt, magno sanè damno: haberemus enim exactam totius structuræ descriptionem, qualem in cæteris expressit scriptor antiquissimus, qui ipsum templum haud dubiè viderat ante conflagrationem. Sed quod in Philone temporum iniuria nobis negauit, aliunde resarcire conabor. Ea propter ipsius Templi formam studiosorum oculis exhibere decreui, duplicis nummi beneficio: quorum unus ex metallo pulcherrimo et forma grandiuscula in Musæo Eminentissimi Cardinalis Francisci Barberini asservatur; alter verò ex schedis Jo. Jac. Chiflleti Patritij Bisuntini depromptus est.

Primus ut vides, templum octo columnis instructum cum Dianæ Ephesiæ icuncula repræsentat. In huius priori parte Hadriani vultus est cum hac inscriptione: \foreignlanguage{greek}{ΑΥΤ. ΚΑΙΣ. ΑΔΡΙΑΝΟΣ. ΣΕΒ.} Hoc idem numisma à Joanne Sambuco mutilum prolatum est. Alter autem nummus appictus templi etiam Ephesini schema, ut literæ circum orbem indicant, ostendit. Quem eò lubentiùs hìc adieci, quòd ambitus exterior templi dispositas circumcirca columnas ostentet, et crepidinem δεκάβαθμον oculis spectandam exhibeat, cuius Philo Byzantius meminit. Ipsius autem templi exædificatio Ionica est; id quod præter vulgata numismata Vitruvius etiam confirmat his verbis: \emph{Primumquè ædes Ephesi Dianæ Ionico genere ab Ctesiphonte Gnossio, et filio eius Metagene est instituta: quam postea Demetrius ipsius Dianæ servus, et Pœnius Ephesius dicuntur perfecisse.} Strabo prioris templi architectum Cheresiphrona, (fortè pro Ctesiphonte) statuit; Cheremocratem verò post Herostrati incendium restituisse pronunciant. Constat autem septies idem fuisse restauratum; idquè inprimis affirmat Plinius.

\emph{Ut autem Templi huius omnium, quæ unquam fuere toto Orbe, celeberrimi, aliqua species è suo funere, imò è suo duplici rogo enitescat, superstites nummos duos in lucem proferimus, templi ipsius ectypo insignitos, quos Auctor noster superiùs descripsit. Cum tamen, ob eorum exiguam magnitudinem, seù paruitatem, ad exprimendam structuræ elegantiam, Ionicos scilicet modulos, gradus, ambitum, aliaque ad symmetriam spectantia impares sint; universam Templi faciem in maius producere, atque in ampliorem formam redigere, quoad fieri potuit, satius duximus, prout in binis proximis tabulis, Petri Sancti Bartoli stylo graphicè, et concinnè, adumbratam, spectandamque subijcimus. Ad maius verò studiosorum priscæ elegantiæ oblectamentum, ipsis duobus nummis tertium, et quartum addidimus, Dianæ cultum, et sacra designantes, duobus pariter subsequentibus tabulis expressos, ut unà simul juncti nil expetendum oculis relinquant.}
\clearpage
\paragraph{}
\emph{Prima Tabula ex nummo Hadriani, qui in ditissima Barberina Gaza adservatur, templi frontem octo columnis fultam, cum Dianæ icuncula in medio repræsentat, Luna et Sole, pari fratris, et sororis cultu desuper affulgentibus, cum epigraphe \foreignlanguage{greek}{ΕΦΕΣΙΩΝ}. Similem Ephesini templi faciem ostentant non tantum Hadriani Nummi, sed et Antonini Pij, et Marci Aurelij ab Ephesijs signati; olim apud Franciscum Gottifredum, nunc in Regio Thesauro Christinæ Auguste; quorum primus Hadriani coronam in fastigio appictam refert, siue ad ornatum, siue ad cultum, quam nostræ editioni addidimus.}

\emph{Secunda Tabula ex schedis Chiflletianis exteriorem templi ambitum designat cum alis; alæ autem columnæ sunt à lateribus ambientes ædem, de quibus Vitruvius, qui etiam de Ionica eius exædificatione loquitur. Tanta verò Templi huius fuit magnificentia, et maiestas, ut centum viginti septem columnis fulciretur à singulis Regibus factis, sexaginta pedum altitudine, ex quibus triginta sex celatæ admirationi erant (quales hìc Romæ nostras videmus Traianam, et Antoninianam anaglyphico opere insculptas) et has a Regibus factas opinamur. Cætera admiranda ex supra relatis Auctoribus agnosces, et ex Plinio præcipuè, ac Philone Byzantio, qui crepidinem decem gradibus, basis vice, eleuatam refert, prout in nostris Tabulis observamus.}

\emph{Tertiæ Tabulæ typus desumptus est ex nummo Caracallæ seu potius Elagabali, ut placet Clariss. Viro Andreæ Morellio in suo egregio opere, cui titulus est: Specimen Universæ Rei Nummariæ antiquæ ex Regio Ludouici Magni Thesauro. Imperatorem laureatum ad tripodem cum patera Sacra Dianæ peragentem refert, juxta templum, siuè sacellum tetrastilon, in cuius medio Ephesinæ Deæ simulacrum extat, inscripto singulari, ac magnifico titulo} \foreignlanguage{greek}{ΕΦΕΣΙΩΝ ΜΟΝΩΝ ΑΓΑΣΩΝ ΤΕΤΡΑΚΙ ΝΕΩΚΟΡΩΝ} Ephesiorum, qui soli ex omnibus quater Neocori fuere. \emph{De quo nummo ipsum Morellium consule.}

\emph{Quarta Tabulæ ex nummo Juliæ Domne, Ephesi Urbis Genium, siue malis Amazonem Smyrnam nominæ, Urbis ipsius. conditricem, capite turrito, hastaque armatam repræsentat. Hanc et Reginam, et ipsius Deæ Sacerdotem fuisse tradit Strabo. De Tauro, qui ante Dianæ simulacrum sistitur. Vide supra Bovis symbola, ac mysteria ab Auctore exposita.}
\clearpage
\vspace*{\fill}
\begin{figure}[H]
\centering
\includegraphics[width=0.95\textwidth,keepaspectratio]{figures/06.jpeg}
\end{figure}
\vspace*{\fill}
\clearpage
\vspace*{\fill}
\begin{figure}[H]
\centering
\includegraphics[width=0.95\textwidth,keepaspectratio]{figures/07.png}
\end{figure}
\vspace*{\fill}
\clearpage
\vspace*{\fill}
\begin{figure}[H]
\centering
\includegraphics[width=0.95\textwidth,keepaspectratio]{figures/08.png}
\end{figure}
\vspace*{\fill}
\clearpage
\vspace*{\fill}
\begin{figure}[H]
\centering
\includegraphics[width=0.95\textwidth,keepaspectratio]{figures/09.png}
\end{figure}
\vspace*{\fill}
\clearpage
\paragraph{}
At si tantus fuit in substructione splendor, non minor fuit donariorum opulentia: quorum tanta fuit copia, ut nullum aliud siue divitijs, siuè oblationibus cum eo conferri posset. Adde quòd ibidem tanquam in ærario tutissimo ditissimi quiq; thesauros reponebant. Sic Dion orat. 32. \emph{Nostis haud dubiè Ephesios, apud quos multæ pecuniæ privatorum reconditæ in Dianæ templo non solum Ephesiorum sed et hospitum, et hominum undequaquè venientium, partim et populorum, et Regum, deponunt autem omnes securitatis gratias; nemine unquam auso loco inferre iniuriam, tametsi plurima bella gesta sint, civitasquè sæpiùs fuerit capta.} Ex quibus lucem accipient verba Laërtij in Xenophonte: \emph{Profectus}, inquit, \emph{deinde Ephesum, dimidium auri, quod secum tulerat, Magabyzo Dianæ Sacerdoti servandum tradidit, quoad reverteretur.} Quanta autem in veneratione olim fuerit hoc fanum, vel hinc argui potest, quòd huic uni post omnia templa Asiatica exusla Xerxes Rex Persarum pepercerit: quamuìs immensis opibus affluere sciret. Cæterùm quidam Herostratus, vel ut alijs placet, Heroastus, ut nominis immortalitatem consequeretur, eodem die, quo Magnus Alexander natus, flammis templum illud vastauit. Cuius incendij meminere Gellius lib. 2. cap. 6. Valerius Maximus lib. 8. cap. 15. et pleriquè alij: obliterato autem nomine incendiarij, ab universo Asiæ conventu restitutum est. Ac deinceps, ut Julius Capitolinus scribit, usq; ad Gallieni Imperatoris tempora integrum remansit, sub cuius imperio Gothis Asiam inuadentibus templum ipsum penitus spoliatum, et incensum est. Sed ante ultimum hoc incendium Nero Ephesinæ gazæ inhians Xerxis barbariem rabiemquè longè superauit. Omnia quippe pretiosa dona, immensam auri argentiquè vim, et simulacra inde, nec non ex omnibus Achaicis templis abstulit, ut Tacitus lib. 15. testatus est. Huic demùm templo haud paruam celebritatem asyli prærogatiua conciliauit. Id enim juris habuit, ut qui ad illud confugissent, siuè ære alieno obstricti forent, eo solverentur: siuè tenerentur servitute, liberi fierent. Idquè non soli templo concessum: sed Alexander etiam ad stadium usque privilegium extendit. Mithridates autem paululum plus spatij addidit. Marcus Antonius maiorem etiam urbis partem comprehendi volvit. Sed postea Augustus, ut flagitiosis impunitatis spes nulla esset, sustlit. Ephesij religionis veræ ignari superstitionis scientissimi omne suum incrementum Dianæ numini ac eius templo acceptum ferebant. At absurdissimis Ephesinæ Theologiæ dogmatibus explosis, falsoquè numine repudiato jure maiori gloriari possunt Ephesij, quod Divi Pauli Apostoli trium annorum spatio auditores, miraculorumquè, quæ ibi operatus est, testes oculati esse meruerint: quod ipsorum Urbs non ementiti numinis sed divini præconis præsentia illustrata fuerit: qui non inuolucris et ambagibus supremi Dei cultum, ut illi portentosi hierophantæ involuebat: sed solida, et aperta veræ sapientiæ documenta apertè diffundebat. Cuius beneficio Urbs ipsa non iam omnigenum animalium septum et stabulum, sed divini eloquij facta fuit sacrarium, eius inquam eloquij et sapientiæ, quæ non ex superstitiosa Aegyptiorum doctrina, sed exlimpidis et salutaribus veri Dei fontibus fluxerat: quæ non illic omnium animantium nutricem; sed omnium parentem et conservatorem verum doceret. Nec minus ad Ephesiorum laudem facit, quod Joannis Apostoli monitis, et cathedra insignes Metropolisquè honore decorati fuerint: quod in ea tertia Synodus œcumenica habita Nestorium Verbi incarnati, et Beatæ Mariæ Dei genitricis hostem damnarit, quòd denique ipsa Genitrix Dei Maria Urbem illam sua præsentia cohonestarit. Procul igitur, procul Mezabolici et euirati ministri, siquidem puri et integri veræ fidei mystæ Deo vero ibidem servierunt. Huic honor sit, et laus, et gloria in sæcula.

Finis.
\clearpage
\subsection{Lector.}
\paragraph{}
\emph{Ne quid eorum, quæ ad religionem Ephesiæ Dianæ visa sunt pertinere, periret, Signa etiam atque Numismata, que ipsius Dianæ insignia non ineptè quis dixerit, excudenda curauimus. Tria Signa prima marmorea in ædibus Farnesiorum seruantur: quartum marmoreum apud Serennissimum Leopoldum Etruriæ Principem: alia duo marmorea apud Leonardum Augustinum Senensem: reliqua in gemmis Camilli Maximi Junioris, atque Leonardi Augustini: quæ sequuntur Numismata ex Ceimeliotheca Barberina promuntur.}
\clearpage
\vspace*{\fill}
\begin{figure}[H]
\centering
\includegraphics[width=0.9\textwidth,keepaspectratio]{figures/10.png}
\end{figure}
\vspace*{\fill}
\clearpage
\vspace*{\fill}
\begin{figure}[H]
\centering
\includegraphics[width=0.9\textwidth,keepaspectratio]{figures/11.png}
\end{figure}
\vspace*{\fill}
\clearpage
\vspace*{\fill}
\begin{figure}[H]
\centering
\includegraphics[width=0.9\textwidth,keepaspectratio]{figures/12.png}
\end{figure}
\vspace*{\fill}
\clearpage
\vspace*{\fill}
\begin{figure}[H]
\centering
\includegraphics[width=0.9\textwidth,keepaspectratio]{figures/13.png}
\end{figure}
\vspace*{\fill}
\clearpage
\vspace*{\fill}
\begin{figure}[H]
\centering
\includegraphics[width=0.9\textwidth,keepaspectratio]{figures/14.png}
\end{figure}
\vspace*{\fill}
\clearpage
\section{Lucae Holstenii Epistola ad Franciscum Cardinalem Barberinum de Fulcris seu verubus Dianæ Ephesiæ simulacro appositis.}
\begin{center}
\scshape\textbf{Romæ, apud Joannem Baptistam Bussottum. 1688.\\Superiorum Permissu.\\Francisco Cardinali Barberino\\Optimo Studiorum Patrono.\\Lucas Holstenius Fel}
\end{center}
\paragraph{}
Cum eruditum opusculum Claudij Menetreij domestici olim tui, solertissimi rei antiquariæ promicondi, singulari beneficio in publicam lucem proferri jusseris, quo Dianæ Ephesiæ statuas ænigmatis reconditis inscriptas luculenter explicauit; non abs re futurum existimaui, si cogitationem de antiquis eiusdem Dianæ numismatis pridem mihi subortam eadem opera ad te deferrem. Feci hoc eò libentius, et excusatius ut spero, quòd argumentum Menetreio intactum, nec ulli Antiquariorum hactenùs observatum complectatur. Tui autem judicij hanc scriptionem facio, quòd nemo te rectius intelligat, quàm multiplicem usum accurata antiquitatis cognitio ad reliquum omne studiorum genus ornandum, augendumque præstet. Cum enim recte judicares, nihil in veterum scriptis tam abstrusum, aut obscurum delitescere, quin lucem aliquam à priscis monumentis mutuari possit, ea tibi causa fuit, cur Bibliothecæ, quàm celeberrimam instruxisti, veterum tabularum, signorum, numismatum, et inscriptionum supellectilem copiosam adiunxeris; ut universam eruditæ antiquitatis memoriam doctorum hominum oculis spectandam subijceres. Nunc rem ipsam cognosce. Nummi veteres Dianæ Ephesiæ effigie signati, quorum nonnullos tabellæ adiunctæ exhibent, eo a statuis Menetreij industria productis differunt, quòd Deam mammarum pondere onustam fulcris quibusdam siue destinis, quos scipiones nodosos existimes, utrinque subrigant. Eaque in re ita constanter consentiunt omnes quotquot vidi, vidi autem quamplurimos, ut simulacrum Ephesium olim eodem modo suffultum fuisse omnino affirmandum putem. Neque vero ea de re nos dubitare permittit Minucij Felicis eruditissimi scriptoris locus, ubi Dianæ Ephesiæ signum mammis multis verubusque extructum commemorat. Sed in loco isto peruertendo Criticorum acumen mirificè hactenùs sese exercuit: adeò quidem, ut nisi vindiciæ ex libris et monumentis antiquis opportunè dentur, posthac futurus sit intestabilis. Rem clarius explico, ut quàm infelicitèr Felix ab editoribus exceptus, habitusque fuerit, penitus perspiciatur.

Princeps Romana editio a Fausto Sabæo ad Vaticani codicis fidem procurata ità locum, quem dixi exhibet: \emph{Diana interim est altè succincta venatrtx; et Ephesia mammis multis et verubus extructa; et Trivia trinis capitibus et multis manibus horrifica.} Ita triplex unius Dianæ numen notis, signisque peculiaribus accuratè ex fabulosa Gentilium theologia distinxit eruditus scriptor. Lectionem primæ editionis religiosè servavit Basileensis, et Heidelbergensis Franc. Balduini; tùm Romana posterior, que ex Fuluij Ursini recensione prodijt; nisi quod coniecturam suam vir accuratissimè doctus margini appingens, non \emph{verubus} sed \emph{uberibus} legendum moneat. Desiderij Heraldi Parisiensis advocati editio duplex deinde Romanam lectionem expressit, sed altera \emph{veribus} una literula mutata pro \emph{verubus} exhibet, ex ipso, ut notat, veteri manuscripto, qui Leonis 10. munificentia ex Vaticana Bibliotheca in regiam migrauit. Missus enim fuit codex ille singularis ad Franciscum regem una cum excuso exemplari, quod eius nomini editores Romani inscripserant. Leuis illa mutatio Heraldum perpulit, ut ipse vel \emph{uberibus} cum Ursino, vel \emph{tuberibus} cum Josepho Scaligero legendum censeret; in priorem tamen coniecturam propensior. Et hi quidem coniecturis eatenùs indulserunt, ut textu Minuciano manus interim abslinerent. At cæteri deinceps non æque fuerunt religiosi. Nam Joannes Uvouverius, ciuis meus, doctrina et judicio alioquin præclare instructus, Ursini coniecturam ita amplexus est, ut in contextum Minucij pro verissima recipere non dubitarit. Huius exemplum postea Rigaltius Lutetiæ, et nuperrime iterata editione Bataui sunt secuti. Nec parum præsidij huic confidentiæ attulisse videtur Justi Lipsij auctoritas, quem correctionem istam ad Taciti Annal. lib. 3. comprobasse sciebant. Et hi quidem omnes cum presso ustigio Ursini semitam calcassent, diversam institit Geuerhartus Elmenhorstius, homo sane haud indiligens: qui Romanam lectionem mutare non ausus, traiectione verborum locum misere luxauit in editione Hamburgensi maiori, ubi Minucij verba ita deformata leguntur: \emph{Diana interim est altè succincta venatrix verubus exstructa; et Ephesia mammis multis; et Trivia trinis capitibus etc.} Sed hæc leuiora sunt præ illis, quæ iam ante ad eundem Minucij locum commentus fuerat Petrus Faber, celebris legum antistes, lib. 3. semestr. cap. 3. Is acri disputatione lacessens S. Hieronymum, quòd Dianam Ephesiam multimammiam a venatrice, quæ arcum tenet, alteque succincta est, distinxerit; magno quidem sed irrito conatu euincere studet, Cererem mammosam veteres nouisse, Dianam mammosam ignorasse. Ideoque Minucij verba, ne Hieronymi caussam iuuent, distorquendo corrigendoque sic interpolat: \emph{Diana interim est alte succincta venatrix Ephesia; et mammis multis Ceres exstructa.} Atqui hoc est, non deprauatum castigare locum, quod ipse de se prædicat, sed integrum corrumpere. Abstinuisset utique infelici concertatione tantus vir, si Dianæ Ephesiæ signum vetus, aut nummos, vel per transennam inspexisset. Adiungam superioribus celebrem nunc in Gallijs virum, fori huius antiquarij regem, Joan. Tristanum. Is tomo 1. exhibet nummum \foreignlanguage{greek}{ΚΑΔΟΗΝΩΝ}, siue Caduenorum, (ita enim Stephano, Plinio, sibique ipsi alibi vocari ostendam) Dianæ Ephesiæ imagine signatum; eique illustrando vexatum Minucij locum ingenij periculum et ipse facturus adducit. Repudiatis igitur verubus, quam non genuinam Minucij vocem, sed Uvouverij, omnia alia sentientis, commentum existimat; uberibus quidem legendum censet, sed longè alio quàm cæteri omnes sensu. Etenim cum inter mammas et ubera aut nullum, aut non nisi ineptissimum discrimen à grammaticis statui rectissime perspexisset, ideoque vix sine vitio tautologiam istam accurato, et pressæ dictionis scriptori tribui posse, per quam ingeniosè mammas multas, uberesque Dianæ Ephesiæ de turgidis multoque lacte distentis censuit explicandas. Quo quidem Ursini, et quotquot eius vestigia posteà legerunt criticorum deprauationem apertè iugulat; Minucij tamen mentem non est assecutus, quòd recenti et interpolatæ editioni, quâ usus videtur, fidem temerè adibuerit.

Cæterùm istis omnibus in perspicuo Minucij loco oculos, mentemquè glaucoma præstrinxit, quòd nequaquam exploratum haberent, quo significatu verua antiquis linguæ Latinæ auctoribus propriè dicta acceptaquè fuerint. Cùm enim assatoria illa, siue transfixoria, ut Papias vocat, quorum in coquina usus est, et quæ in Gallica Minucij interptetatione meritò ridet Tristanus, ad Dianam Ephesiam extruendam nihil quidquam facere viderent; non nisi de verutis siue iaculis Dianæ venatricis Minuciana verua accipi explicarique posse uno omnes, quod miror, consensu censuerunt. Unde factum, ut tot summi et incomparabiles viri ad vocem illam interpolandam certatim conspirarint. Ego vero adversus coniuratum agmen verua ista extra vitium esse affirmo, eaque nec Dianæ Ephesiæ, nec Minucio subtrahenda pertendo. Nam verua hæc quibus Dianam suam exstruebant Ephesij, non pila aut veruta sunt, sed fulmenta ferrea oblonga, quæ brachijs supposita totam mammosi pectoris molem sustinebant.

Cùm enim antiquissimùm hoc signum ad Aegyptiorum simulacrorum instar pedibus esset arctè compressis, tantilla basis superimposito corporis ponderi ferendo impar, adminiculis suffulcienda fuit; quibus subtractis universam molem fatiscere et collabi necessum erat. Ea fulcra siue sustentacula, quod ex ferro longius producta essent, Minucius propria et eleganti voce verua dixit, non sequioris, quo vixit, sed Augustæi sæculi usum secutus. Nam Glossarium Latinum optimæ notæ, quòd in tribus vetustissimis codicibus Vaticanis extat, verua virga et virgulas ferreas interpretatur. Eamque explicationem veram ac genuinam esse res ipsa me docuit: cuius etiam nunc te meminisse arbitror Eminentissime Cardinalis, quòd juxtà mecum oculis eam olim usurpaueris. Anni enim sunt, ni fallor, quindecim, cum Jesuitarum socictas ad S. Andreæ in Quirinali, dum novæ ædificationis fundamenta moliretur, lapides aliquot Tiburtinos prægrandes ordine quadrato dispositos offenderet: quorum duo ità erant inscripti, uti cum te jubente et amminiculante ex saxo scabro, et male polito excepi.
\begin{quote}
HAEC. AREA. INTRA. HANCCE\\
DEFINITIONEM. CIPPORUM\\
CLAUSA. VERIBUS. ET. ARA. QUAE\\
EST. INFERIUS. DEDICATA. EST. AB\\
IMP. CAESARE. DOMITIANO. AVG\\
GERMANICO EX. VOTO. SUSCEPTO\\
QUOD. DIV. ERAT. NEGLECTUM. NEC\\
REDDITUM. INCENDIORUM\\
ARCENDORUM. CAUSA\\
\end{quote}
\vspace*{-8mm}
\paragraph{}
Cippi illi utroque latere bina foramina, et veruum siue virgarum ferrearum vestigia plumbo circumsusa seruabant; quibus olim inter se coniuncti aream interiorem ità clauserant, ne aditus vulgò pateret. Ibi tum reipsa perspexi, egregiè falsum esse N. Rigaltium hominem naris emunctissimæ, dum ad Finium regundorum scriptores verua ista stipites instar subularum præacutos explicat, quæ nihil acuminis, aut cuspidis habuisse oculis manibusque cognoueram. Eodem vetustatis sensu Marcellus, antiquus rei medicæ auctor, sanguinis profluuio ex naribus sistendo præscribit cap. 10. \emph{Vertu ferreum candens in aceto adsiduè extingue, et fumum eius naribus ducito.} Quis non videt simpliciter hìc et absolutè ferri in virgam oblongam producti massam intelligi, qua forma fere omne ferrum rude et infectum vulgo venit.

Verùm nihil ad Minucij mentem, et rei de quà agitur illustrationem adferri potest aptiùs versibus Prudentij ex priori contra Symmachum carmine; quibus geminorum fratrum Castoris et Pollucis simulacra describit eo habitu gestuque, quò tunc in sua sibi æde ad viam sacram visebantur. Versus isti sunt:
\begin{quote}
\hspace*{15mm}\emph{Gemini quoque fratres}\\
\emph{Corrupta de matre nothi, Ledeia proles,}\\
\emph{Nocturnique equites, celsæ duo numina Romæ}\\
\emph{Impendent retinente veru; magnique triumphi}\\
\emph{Nuntia sussuso figunt vestigia plumbo.}\\
\end{quote}
\vspace*{-8mm}
\paragraph{}
Graphice depingit Dioscuros currentium gestu itâ effigiatos, ut extra perpendiculum et basin prominentes spectantibus non sinc horrore, ac metu impendere viderentur. Impendent enim quæ suprà caput iamiam casura pendent, ut recte ait Valla. Timor autèm, et admiratio apud rude et superstitiosum vulgus religionis opinionem conciliabat. Eum metu poëta Christianus ridet, cum nullum esset ruinæ, aut fugæ periculum; quòd veru, hoc est vectis siue uncus ferreus a tergo infixus eos retineret, pedesque basibus applumbati moveri non possent. Duplex hoc retinaculorum genus, queis numinum simulacra veluti vinculis constricta defigebantur, Arnobius similitèr lib. 6. nationibus exprobrat. \emph{Si permanendi,} inquit, \emph{necessitatem patiuntur, quid miseriùs his esse, aut quid infeliciùs poterit, quàm si eos in basibus ità unci retinent et plumbeæ vinctiones?} Utrumque etiam conjungit lex. 2. Dig. de sepulcro violato. \emph{Celsus querit, si neque applumbata fuit statua, neque adfixa, an pars monumenti effecta sit, an vero maneat in bonis nostris.}

Sed Prudentius quoque Minucij fatum et malam criticorum manum ut experiretur, eiusdem vocabuli non satis recte observata significatio fecit. Unde iam olim in peruetustis membranis Vaticanæ bibliothecæ, \emph{retinente solo}, pro \emph{retinente veru}, substitutum videre est: quàm lectionem Aldus, alijque eius fidem secuti expresserunt. Georgius autem Fabricius, cum in suis exemplaribus, \emph{retinente veru} constanter scriptum reperisset, nec tamen proprium vocis usum apud veteres satis haberet perspectum, ad correctionem, sacram criticorum ancoram, confugit. Quocirca cum Dioscuros hastis siue pilis ad decursionem volgò armari sciret, veru hoc ex rudi et obtuso ferro in verutum cuspide spicauit; et interpolata dictione, \emph{impendent, retinentque veru,} de suo quo pollebat ingenio procudit: eaque lectio exinde plerasque recentiores editiones insedit. Tantum vero absum eam ut probem, ut contra priscam et genuinam mordicus tueri non dubitem, cùm rei ipsius perspicuitate, tùm veterum librorum auctoritate fietus: in quibus facilè principem statuo codicem præstanrissimum Urbinatis Bibliothecæ, quæ nuper Alexandri Septimi sapientissimi Pontificis immortali beneficio Vaticanæ accessit; in quo ità scriptum, reperi.

Hæc ad Dianæ Ephesiæ statuas, nummosque veteres illustrandos scripsi ut mutua eorundem ope duo veterum scriptorum loca minùs rectè hactenùs intellecta à criticorum corruptelis vindicarem: tùm verò ut illustri exemplo ostenderem, antiquitatis studium non metiendum inani delectatione, sed suum illi constare fructum si rectè colatur.

Finis.
\clearpage
\vspace*{\fill}
\begin{figure}[H]
\centering
\includegraphics[width=0.9\textwidth,keepaspectratio]{figures/15.png}
\end{figure}
\vspace*{\fill}
\clearpage
\vspace*{\fill}
\begin{figure}[H]
\centering
\includegraphics[width=0.9\textwidth,keepaspectratio]{figures/16.png}
\end{figure}
\vspace*{\fill}
\clearpage
\section{Jo. Petri Bellorii Notæ in Numismata tum Ephesia, tum Aliarum Urbium Apibus Insignita.}
\begin{center}
\textbf{De Ape, ac Lyra}
\end{center}
\begin{quote}
\emph{Aspice dulcis Apis, Lyra dulcis in Astra locantur:}\\
\emph{Hæc cantu, illa favis, munus utrumque Deum.}\\
\end{quote}
\vspace*{\fill}
\begin{figure}[H]
\centering
\includegraphics[width=0.75\textwidth,keepaspectratio]{figures/17.png}
\end{figure}
\vspace*{\fill}
\clearpage
\section{Apum Auspicia.}
\begin{center}
\scshape\textbf{Eminentiss. ac Reverendiss.\\Principi Francisco Barberino Cardinali.}
\end{center}
\paragraph{}
Apum auspicijs cæptum absoluendum est opus, Eminentissime Princeps, quibus regnantibus, Roma, atque Christianus orbis universus rores melleos, è cælo cadentes, divinum nectar excepit. Fabula est Prænestæ tuæ fortes, quondam melle nobilitatas: BARBERINAE APES dulcioribus mellificantes favis, nunc sanè nobilitant: fas erit tamen miraculum illud, inter vetera auspicia, ex Cicerone recensere: \emph{eo loco ubi nunc fortunæ sita ædes est, mel ex olea fluxisse dicunt, Haruspicesque dixisse summa nobilitate illas sortes futuras}. Quid de Platonis, ac Pindari infantia? fama est quoque vagientis Jovis ore mellificasse Apes nutrices, regnique eius augures extitisse:
\begin{quote}
\emph{Dictæo Regem cæli pauere sub antro.}
\end{quote}
\vspace*{-4mm}
\paragraph{}
Justinus de Hierone: \emph{paruulum, et humanæ opis indigentem, Apes congesto circa jacentem melle, multis diebus aluere, ob quam rem responso Haruspicum admonitus pater, qui Regnum infanti portendi canebant, puerum recolligit, omnique fludio ad spem maiestatis, quæ promittebatur instituit.} Apes Antonini statuis insidentes eidem Imperatori futuro augures fuerunt. Capitolinus: \emph{eius statuas in omni Hetruria examen Apum repleuit}: quare Artemidorus: \emph{Apes insidentes capitibus significant futuros duces atque Imperatores}. Dionysio quoque examen Apum considens in equi juba, Imperij, fuit præsagium. Aelianus: \emph{Cum apprebenderet eius jubam ad ascendendum, Apum examen continuò manum circundedisse, atque de his interroganti, respondisse Dionysio, Galeotas Monarchiam ea re præsignificari}. Apes \emph{sedisse in castris Drusi Imperatoris} memorat Plinius: \emph{cum prosperrimè pugnatum apud Arbalonem est} Julianus Episcopus de Uvamba in Hispaniæ Rege: \emph{Hic regio iam cultu conspicuus ante altare divinum consistens, ex more fidem populis tradidit, deinde curuatis genibus, oleum benedictionis, per sacri Quirici Pontisicis manus, vertici eius refunditur: et benedictionis copia exibetur; ubi statim signum hoc salutis enituit: namque mox è vertice ipso, uti oleum ipsum perfusum fuerat, evaporatio quædam fumo similis, in modum columnæ sese erexit, et ex capite ipso Apis visa est prosilijsse, et aeris alta petijsse, quæ, utique signum fuit secturæ felicitatis.} et Rodericus Toletanus: \emph{Visa est Apis deius capite prosilijsse, et ad cælos continuò euolasse: et qui diligentius cogitabant, intelligebant, per eum Gothorum regnum feliciter exaltandum, et in pacis dulcedine gubernandum.} His Childerici Francorum Regis regias Apes subnectam, quas desumpsimus ex Jo. Jacobi Chifletij eruditissimis commentarijs in eiusdem Regis Anastasim, siuè Thesaurum sepulcralem Tornaci Neruiorum effossum. Inter cimeliorum reliquias, Apes supra tercentæ inuentæ sunt, quondam Francorum Regum insignia, postea in Lilia (quæ nunc sunt) commutatæ. Multa ad probandum Chitletius adducit argumenta: Apes aureæ, aurea Lilia: cælestia Lilia, cælestes Apes: Lilia in æthere pinguntur, Apum campus æther: Lilium Regium flos, Apes Regiæ. Quemadmodum ergo regalia Childerici antiquitus aureæ insignibant Apes, expansis alis, veluti ab æthere deductæ, sic aurea modò in æthere ipso splendent Lilia, Apes ipsas deciduas imitantia. Cæterum post Childericum Regem, symbolicas Apes primus palam usurpauit Ludovicus 12 Rex Francorum stirpe Valesius. Jo. Baptista Mantuanus de triumphali eius introitu in Ciuitatem Genuensem anno 1507 ità cecinit:
\begin{quote}
\emph{In medio Rex victor equo sublimis in alto}\\
\emph{Murice conspicuus, rutilanti splendidus auro}\\
\emph{Signabatur Apum sparsim toga tota figuris;}\\
\emph{Cumque Apibus Regnator Apum fulgebat in ostro.}\\
\end{quote}
\vspace*{-8mm}
\paragraph{}
Henricus 3 Rex Valesiorum postremus ijsdem Apum impressa typis, bina edidit numismata: alterum argenteum hoc præfert lemma PLEBIS AMOR. REGIS CUSTODIA: alterum aureum: REGE INCOLUMI MENS OMNIBUS UNA. Postremò Henricus 4 Rex Borboniorum primus anno 1608 aureum numisma eudit eum Apibus Regem suu stipantibus, et hoc symbolo AMORE NON TERRORE.

Hinc cætera inter animantia, Apes tantum Regem suum summo studio, summaque veneratione prosequuntur: hinc \emph{regias Imperatoribus futuris extruunt amplas, magnificas, separatasque domos}: Hinc \emph{celsior regibus ipsis in fronte macula, quodam diademate candicans}: ut Plinius, et Virgilius diligentissimè speculantur:
\begin{quote}
\emph{Præterea Regem non sic Aegyptus, et ingens}\\
\emph{Lydia, nec populi Parthorum, aut Medus Hydaspes,}\\
\emph{Observant; Rege incolumi mens omnibus una est.}\\
\end{quote}
\vspace*{-8mm}
\paragraph{}
Et ut Medus ille, Cyrum Apum Regi comparando: \emph{mirificus eis amor erga principem.}

Nec tantùm Regna, et Imperia, verùm etiàm Heroas, et divinitatem portendi ab Apibus traditum est. Herodianus: \emph{Onesilo contigit divinitas, eique sacrificia quot annis sunt institutas; propterea quod in eius defuncti caput Apes mel cogessissent.}
\begin{quote}
\emph{His quidem signis, atque hæc exempla sequuti}\\
\emph{Esse Apibus partem divinæ mentis, et haustus}\\
\emph{Aetherios dixere.}\\
\end{quote}
\vspace*{-8mm}
\paragraph{}
Sed iam in Hymetto, sacra litaui, Eminentissime Princeps, nunc Apes tuas, in Heliconijs Musarum collibus, omni genere doctrinæ virentibus, immortali tymo, ac rore, depastas colo; solemnemque tuum illum in literas, ac literatos, tutelarem genium veneror, quibus, et dulcissima mella instillas, et favos. Patere verò Apes ipsas, argumento, auspicioque regias, humili me, rudique stylo, complexum: patere, auspicibus ipsis, me nomini tuo vota nuncupasse, cui dudum obsequium omne meum, et cultum deuoui.

Emin. Tuæ

\emph{Observantiss. humillimusq; seruus Jo. Petrus Bellorius.}
\clearpage
\vspace*{\fill}
\begin{figure}[H]
\centering
\includegraphics[width=0.9\textwidth,keepaspectratio]{figures/18.png}
\end{figure}
\vspace*{\fill}
\clearpage
\subsection*{Tabula 1.}
\begin{enumerate}
    \item Apis Dianæ symbolum. Exaduerso: \foreignlanguage{greek}{ΕΦ. ΕΦΕΣΙΩΝ} \emph{Ephesiorum}. Lyra Apollinem refert. Porrò hi salutares Dij ijsdem titulis, et pari veneratione colebantur; omniumque primi Ephesij (Deli fabula repudiata) Apollinem, ac Dianam apud se natos gloriabantur. Tacit. Ann. l. 3. \emph{Primi omnium Ephesij adiere memorantes, non ut vulgus crederet, Dianam atque Apollinem Delo genitos, esse apud se Cencbrium amnem, lucum Ortygiam, ubi Latonam partu grauidam, et Oleæ quæ tum etiam maneat, adnixam edidisse ea numina}. Ephesijs præterea celebris cultus fuit Larissæi Apollinis, cuius templum, in ipsorum agro commemorat Strabo. Multiplex verò est ratio, qua Apis Ephesiæ Artemidi consecrabatur; namque illa virginitatem colit, rorem Lunæ ex floribus legit; atque Ephesinæ Cereri summoperè grata est, ut Menetreius adnotauit in Apis hierogrammatismo, et infra ex alijs Ephesiorum nummis, eodem hieroglyfico insignitis abundè patebit.

    \item \foreignlanguage{greek}{ΕΦ.} \emph{Ephesiorum.} Apis, ut in superiori nummo; exadauerso: \foreignlanguage{greek}{ΜΕΝΙΠΠΟΣ} \emph{Menippus}, et Ceruus animal Dianæ venatrici dicatum: quare ipsa Dea Elaphiea Homero ἐλαφηβόλος, et Euripidi ἐλαφηοκτόνος, seu Ceruicida fuit cognominata. Libanius or. 32. asserit Ephesiorum moris fuisse Ceruam insculpere suis nummis, ob beneficia in ipsos à Diana collata. \emph{Apud Ephesios etiam nummus Ceruam expressam fert, pro remuneratione magnorum Deæ beneficiorum.} De Palma arbore in sequentis tabulæ 5 nummo infra dicendum.

    \item \foreignlanguage{greek}{ΕΦ.} \emph{Ephesiorum} \foreignlanguage{greek}{ΕΥΚΡΙΤΟΥ} \emph{Eucriti}. Nomen Græcum, quod diversum in diversis nummis Ephesiorum legitur, sine dubio, eius est, qui summo cum honore, Dianæ templo, sacrisque præerat; siue ille Asiarches, (ut quidam existimant) siue potius Grammateus fuerit; utriusque enim mentio extat in actis Apostolorum cap. 19. ubi de huius templi, numinisque cultu, ac veneratione agitur. Nam scribæ siue Grammateos magna isthic conspicitur auctoritas, in reprimendo populi impetu, et seditione, religionis causâ, exorta, sedanda. Apuleius quoque l. 11. inter Isiacos Sacerdotes, unum præcipue Grammatea dictum commemorat. Ad hæc in nummis antiquis Grammateos tanquam præcipui Magistratus nomen legitur; Asiarcham verò non Prouiuciæ Præfecturam, sed præcipui Sacerdotij munus gessisse, eruditè admodum summi viri, et antiquitatis universæ peritissimi Antonius Augustinus ad Modestinum de Excusat; et Jacobus Cuiacius ad l. 1. responsorum Papiniani iam pridem docuerunt; præerat siquidem Asiæ Communibus Sacris, Ludis, ac publicis Conuentibus. Nummos verò, nec dum editos ab alijs, quò rariores, eo libentius hisce notis atteximus ex Gotifrediando Museo, nunc Augustæ Christinæ Reginæ.

    Nummus Triumvirorum \foreignlanguage{greek}{ΕΦ. ΑΡΧΙΕΡΕΥΣ ΓΡΑΜ. ΓΛΑΥKΩΝ. ΕΥΘΥΚΡΑΤΗΣ} \emph{Ephesiorum Archiereus Grammateus Glaucon Euthycrates.} Hic idem primus flamen, et Grammateus fuisse videtur.

    Nummus M. Aurelij \foreignlanguage{greek}{ΕΠΙ. ΕΥΑΡΕΣΤΟΥ ΓΡΑ. ΤΡΑΛΛΙΑΝΩΝ.} \emph{Sub Euaresto Grummateo Trallianorum.} Asiarcham autem nummus Smyrnensium refert \foreignlanguage{greek}{ΕΡΙ. M. ΑΥΡ. ΤΕΡΤΙΟΥ. ΑΣΙΑΡΧΟΥ. ΣΜΥΡΝΑIΩΝ ΠΡΩTΩΝ ΑΣΙΑΣ.} \emph{Sub Marco Aurelio Tertio Ascarcha Smyrnensium Principum Asiæ.}

    Et nummus Bassiani Antonini \foreignlanguage{greek}{Λ. ΑΙΛ. ΠΙΓΡΗΣ. ΑΣΙΑΡΧΗΣ. Γ. ΑΝΕΘΗΚΕΝ. ΛΑΟΔΙΚΕΩΝ. ΝΕΩΣΣΡΩΝ.} \emph{L. Ael. Pigres Asiarcha 3 posuit Laodicensium Neocororum.}

    \item Caput laureatum, alatum Victoriæ est, sin malis Mercurij, et Apollinis unitatem, et connexionem, in ipso exprimi. Horum numinum communem aram, in templo Jovis Olympij fuisse tradit Pausanias, ac de Mercurio Sole plura docet Macrobius, et ex eo, alijsque Aleander. Tridentis, et Apis notæ indicant nummum ipsum cusum in aliqua Siciliæ, aut mari adiacenti civitate: quòd enim fit in marinis locis præstantius mel esse Plinius observat, et infrà patebit ex alijs nummis Apis emblemate decoratis, ad indicandam mellifluam, felicemque regionis ubertatem. In adversa nummi parte quadriga impressa est, cum Victoria, et litteris: L. JULI. BRUSI, de quo consule Fuluvium Ursinum.

    \item Romæ caput galctum, cum nota 10 quæ denarij nota est: exadverso Apis, de qua in antecedenti nummo. Biga, et quadriga celebritatem ludorum designat: unde bigati, et quadrigati.

    \item Apollinis caput laureatum, cum Apis, ac tridentis nota, ut in 4 superiori nummo extra urbem cuso. Adiecta est littera K singularis, quemadmodum nummi alij, dupplici, vel unica littera signati reperiuntur, ut \foreignlanguage{greek}{ΚΟΡΑΣ}, per unicum K adnotatur. In altero huius denarij latere P. CREPUSI. IIIIXX. inscriptum est, quem Crepusium Triumvirum monetæ eundendæ fuisse Ursinus autumat. Eques hastatus victor Celete, siue desultorio equo videtur, siuè Crepusiorum aliquis Equestri Statua donatus, ex eiusdem Ursini sententia.
\end{enumerate}
\clearpage
\vspace*{\fill}
\begin{figure}[H]
\centering
\includegraphics[width=0.9\textwidth,keepaspectratio]{figures/19.png}
\end{figure}
\vspace*{\fill}
\clearpage
\subsection*{Tabula 2.}
\begin{enumerate}
    \item Caput caprina pelle velatum Junonem Sospitam præfert, quæ Lanuuij colebatur: de ipsa Ovidius l. 6. fast. et Cic. pro Murena, et lib. 1. de leg. \emph{Tam hercle} (inquit) \emph{quam tibi illam nostram sospitam, quam tu nunquam, ne in somnis quidem vides, nisi cum pelle caprina, cum hasta, cum scutulo, cum calceolis repandis.} Pegasus, qui in altero latere impressus est, Syracusis, nummum fuisse cusum demonstrat; cuius urbis Pegasus fuit insigne, quemadmodum Apes in Syracusijs, alijsque Sicularum Urbium, nummis exhibentur. L. PAPI. \emph{Lucius Papius} Triumuir fuit monetalis, qui ut Papiam gentem Lanuuio oriundam indicaret, Lanuuinæ Junonis simulacrum in denarijs expressit.

    \item \foreignlanguage{greek}{ΝΕΣΠΟΛΙΤΗΣ.} \emph{Neapolis.} Bos humano ore conspicuus Hebo est Neapolitanæ præcipuum, aliarumque Campaniæ, et Siciliæ Urbium tutelare numen: de quo Macrobius: Apollinem, Liberumque unum, eundemque Deum esse probans: \emph{Item liberi patris simulacra partim puerili ætate, partim iuuenili fingunt. Præterea barbatà specie, senili quoque, ut Græci eius, quem Bacchapæam, item quem Brissæa, appellant, et ut in Campania Neapolitani celebrant Hebona cognominantes.} Eiusdem numinis ratio, et figura, promissa barba, ad radiorum similitudinem, virtutem Solis in terrena demonstrat. Victoria quæ super Hebonis caput coronam præfert, Ludorum Quinquennalium celebritatem appingit, de quibus Strabo l. 5. Neapolim ipsam describens. \emph{Hoc tempore sacrum Quinquennale certamen, Musicum, et Gymnicum, per aliquot dies agitur, ludis Græcorum nobilissimis æmulum.} Apis Campani mellis est index; Campania etenim, ob rosarum, florumque præstantiam præcipuè commendatur. Muliebre caput exaduerso impressum Cererem, seu Dianam representat, cum in alijs nummis similibus legatur APT. \emph{Artemis}, quæ Diana est Lucina, ut conijcere licet ex pusilla imagine, retro facem præferente.

    \item \foreignlanguage{greek}{ΕΦ. ΣΚΩΠΙ.} \emph{Epesiorum Scopi}: in auverso latere \foreignlanguage{greek}{ΚΗΡΙΑΙ. ΕΩΑΣ. ΠΡΟΣ. ΠΑΛΥΡΙΝ.} \emph{Cerificantes orientalis ad Palurin}, quem siue montem, siue flumen esse credi potest, sed felicioribus ingenijs divinandi locum relinquimus.

    \item Apis ab uno, ab alio latere Formicæ binæ: nummus est, siue tessera plumbea sine literis, sedulitatis, et industrie symbolum continet Apem, et Formicam, quo etiam ipsæ sacræ literæ utuntur, ut Virgilium Juvenalem, aliosque transeam, Phocilides.
    
    \begin{quote}
    \emph{Nil bene, si desit, geritur, sudorque, laborque,}\\
    \emph{Non ipsisque Deis: virtus sudore iuuatur.}\\
    \emph{Victum desertis antris, telluris egentes}\\
    \emph{Formicæ vadunt, redeuntque per inuia, quando}\\
    \emph{Desectæ campis, complent, nunc horrea fruges,}\\
    \emph{Hordea conuectant tritici, vel grana labore}\\
    \emph{Assiduo, sequiturque ferens sua dona ferentem.}\\
    \emph{Inque hyemem victum proprium de messe reponunt.}\\
    \emph{Impigrum, paruumque genus, inultique laboris}\\
    \emph{Tractat APIS studiosa suum, prudensque laborem,}\\
    \emph{Siue cauæ in petræ speleo, in arundinibusue,}\\
    \emph{Stat in ventre caui, jucundo Roboris antro}\\
    \emph{Floribus efficiens, bene olentia cerea tecta.}\\
    \end{quote}
    
    Sanctus Paulinus ep. 30. \emph{Quantum de rure ad eruditione animæ trahi possit, ipsa rerum opifex Sapientia docet, cum sectatores suos ad Formicam, et Apem mittat, quæ utraque ruris animalia sunt: illa de frugibus vitæ prouida, et ista de floribus mellis operaria.}

    \item \foreignlanguage{greek}{ΕΦ.} \emph{Ephesiorum.} \foreignlanguage{greek}{ΔΗΜΗΤΡΙΟΣ} \emph{Demetrius.} Palma arbor, in hoc alijsque Ephesijs nummis signata, Latonæ partum, ac Dianæ natale testatur, ut præter Homerum, Theognis poëta Latonam parturientem Apollinem, et Dianam, palmam manibus fuisse complexam canit. Erat apud Ephesum lucus Ortygia, ubi Latona partu gravida Dianam edidit; Apollinem tamen in Delo natum canit Homerus hym. in Apol.
    
    \begin{quote}
    \emph{Salue ò beata Latona, quoniam peperisti præclaros liberos,}\\
    \emph{Apollinemque regem, et Dianam sagittis lætam:}\\
    \emph{Hanc quidem in Ortygia, illum verò aspera in Delo.}\\
    \emph{Inclinata ad longum montem, et Cynthium collem}\\
    \emph{Juxta Palmam.}\\
    \end{quote}
    
    Pausanias l. 7. cap. 9. Palmas in Ionia nasci affirmat, Ionumque regionem cæli clementia frui, quod idem colligitur etiam ex Strabone l. 14. et ex Dionysio Periegete. Apud Franciscum Gottiftedum V. C. Extat Gordiani nummus, cum Diana Ephesia in naui, cuius prore Imperator insistens, coronam eius capiti imponit cum epigraphe ab uno latere \foreignlanguage{greek}{ΑΥΤ. Κ. Μ. ΑΝΤ. ΓΟΡΔΙΑΝΟΣ.} \emph{Imperator Cæsar Marcus Antonius Gordianus} ab alio \foreignlanguage{greek}{ΕΦΕΣΙΟΝ ΚΑΙ. ΑΛΕΧΑΝΔΡΕΩΝ. ΟΜΟΝΟΙΑ.} \emph{Ephessorum, et Alexandrinorum concordia.} Ex quo posset conijci Palmam Alexaudriæ symbolum esse, cum Dianæ Ephesiæ simulacrum tot hieroglyphicis insigne ab Isiacis Hierophantis ex Aegypto in Ioniam traductum videatur.

    \item \foreignlanguage{greek}{Μ. ΟΠΕΛ. ΑΝΤ. ΔΙΑΔΟΥΜ. Κ.} \emph{Marcus Opelius Antoninus Diadumenianus Cesar} \foreignlanguage{greek}{ΕΦ.} \emph{Ephesiorum.} Alter nummus, in honorem patris Macrini, ab Ephesijs fuit cusus, cum literis \foreignlanguage{greek}{ΕΦΕΣΙΟΝ ΠΡΩΤΩΝ. ΑΣΙΑΣ.} \emph{Ephesiorum Principum Asiæ}, ob aliquod beneficium in Ephesios ab Imperatore collatum.
\end{enumerate}
\clearpage
\vspace*{\fill}
\begin{figure}[H]
\centering
\includegraphics[width=0.9\textwidth,keepaspectratio]{figures/20.png}
\end{figure}
\vspace*{\fill}
\clearpage
\subsection*{Tabula 3.}
\begin{enumerate}
    \item Caput Lysimachi diademate, atque arietinis cornibus exornatum, quæ Regum Macedonum erant insignia, ad imitationem Alexandri Magni, qui Jovis Ammonis filius haberi, et cognominari voluit. In aversa parte nummi appicta est, imago Palladis victricis, cum literis \foreignlanguage{greek}{ΒΑΣΙΛΕΩΣ. ΛΥΣΙΜΑΧΟΥ. ΛΥΣΙΜΑΧΕΩΝ.} \emph{Regis Lysimachi Lysimachensium}, in memoriam Regis ipsius Urbis conditoris. Fuit Lysimachia celebris civitas, in Thraciæ Cherronensi faucibus.

    \item Caput laurea redimitum Phileteri est, ludrico certamine victoris. In aversa parte inscriptum est: \foreignlanguage{greek}{ΦΙΛΕΤΑΙΡΟΥ} \emph{Phileteri.} Hic Eunucus Lysimachi, eiusdem thesauris, Pergamoque urbe validissima potitus, Attalicorum Regum fuit auctor, ut fusè narrat Strabo. Pallas victrix coronam præfert, quemadmodum in antecedenti nummo, et Lysimachiæ cusum denarium ostenditur, ut ex nota M. que in tres literas dividitur \foreignlanguage{greek}{ΛΥΣ} \emph{Lysimachensium.}

    \item \foreignlanguage{greek}{ΑΡΚΑΔΙΩΝ ΔΗΜΩΣ} \emph{Arcadum populus} Caput laureatum Lycei Jovis est simulacrum: Arcades enim ab Jove genus ducere, ipsumque in Lyceo monte educatum asserebant. Aquila Jovis aliger est, infra quam Apis cernitur mellifluæ regionis argumentum. Cusus fuit hic nummus ab Arcadum populis, fortasse Megalopoli, sic enim vocabant magnam Urbem, in quam coierant aliæ Urbes Arcadiæ.

    \item \foreignlanguage{greek}{ΒΡΕΤΙΩΝ} \emph{Brutiorum.} Jupiter hastæ innixus pedem super columnæ epistylium tenet: ab alio latere, caput Junonis apparet, cum Apis tessera in feracissimo regionis agro. Juno autem hæc, sine dubio, Lacinia est, cuius templum celebre, propè Cotronem extabat, sanctum omnibus circa populis, ut ait Livius lib. 24.

    \item \foreignlanguage{greek}{ΔΥΡ. ΣΤΡΑΤΩΝΟΣ}. Hic, et sequentes nummi \foreignlanguage{greek}{ΔΥΡ}. syllaba inscripti\\ \foreignlanguage{greek}{ΔΥΡΡΑΧΙΩΝ} \emph{Dyrrachium.} notant. Stephanus hanc Urbem Illyricam, antea Epidamnum dictam memorat ab Epidamno heroe. Strabo sie: \emph{Epidamnus à Corcyrræis condita, quæ nunc à peninsula, cui imposita est, nomen Dyrrachij tenet.} Figura quadrata ad similitudinem aræ structa conspicitur \foreignlanguage{greek}{ΣΤΡΑΤΩΝΟΣ} nomen est Prætoris, sicuti \foreignlanguage{greek}{ΑΛΚΑΙΟΣ} \emph{Alcæus}, alterius magistratum gerentem notat. Boves præcipue magnos fert Epirotica terra, quas Pyrricas vocant, ipsisque laudem maximam auctores tribuunt, quare Bos cum lactente Vitulo, impressa est in denario, ad indicandam quoque agri fertilitatem; ac pari emblemate Apis mellis copiam præbet, ut sequenti Tabula clarius patebit, ex aluearijs, in consimili numismate insculptis.

    \item \foreignlanguage{greek}{ΕΦ.} \emph{Ephesiorum} \foreignlanguage{greek}{ΑΡΜΟΝΙΟΥ} \emph{Armonij.} Ceruus, et Palma, de quibus in precedentibus.
\end{enumerate}
\clearpage
\vspace*{\fill}
\begin{figure}[H]
\centering
\includegraphics[width=0.9\textwidth,keepaspectratio]{figures/21.png}
\end{figure}
\vspace*{\fill}
\clearpage
\subsection*{Tabula 4.}
\begin{enumerate}
    \item \foreignlanguage{greek}{ΔΕΛΦΟΥ} \emph{Delphi.} Semicapra in hoc nummo signata est, fortè quod hoc animal primum oraculum Delphicum manifestauerit ut notat Ludouicus Nonius, Diodori nixus auctoritate. Tradit etiam Pausanias in Phocicis Cleoneos pestilentia laborantes, oraculi monitu, sole primum Oriente, Caprum immolasse, ac sedata lue, æneum Caprum Apollini Delphico obtulisse. Pharetra Pythij Apollinis insigne est Pythonem iaculantis, in cuius honorem Peana instituta sunt. Apis Dianæ dicata est, ipsamque Ephesinam Deam designare videtur. Sed fortè Apum aliud extat argumentum; quippe proditum est ædiculam Delphici Apollinis ab Apibus è cera, et pinnulis fuisse compactam, ut Pausanias refert \foreignlanguage{greek}{ΞΕΝΩΦΑΝΗΣ} \emph{Xenophanis} Delphici Pretoris nomen est.

    \item \foreignlanguage{greek}{ΔΥΡ. ΟΒΡΥΜΟΥ} \emph{Dyrrachium Obrimi.} De hac urbe in antecedenti Tabula dictum est. In aversa nummi facie, alucaria bina, totidemque Apum examina appicta sunt, quæ dubio procul, feracissimi, et ditissimi agri ubertatem indicant. Nam preter Bovem, et lactentem Vitulum in singulis Dyrrachij nummis; siue arista, siue botrus vue, siue aratrum impressum est. Præ cæteris verò Dyrrachiensis ager, mellis divitijs, et Apum gloria sese efferebat, ut in hoc, aliisque nummis, mellifluo stemmate decoratis.

    \item \foreignlanguage{greek}{ΔΥΡ. ΔΑΜΗΝΟΣ} \emph{Dyrrachium Daminus}, exadverso \foreignlanguage{greek}{ΦΙΛΩΝ} \emph{Philo.} Spica, et Apis ubertatis symbola sunt, ut in antecedenti nummo adnotatum est.

    \item M. PLAETORIUS. CESTIUS. Sortis, siue Fortunæ simulacrum cum tessera SORS. Denarium hunc extra Urbem cusum suspicari possumus, quod Apis in co sit impressa. Non omittam felix mellis præsagium de Prænestinis sortibus, de quo Cicero l. 2. de Diu. \emph{Eo loco ubi nunc fortunæ sita ædes est, mel ex olea fluxisse dicunt, Haruspicesque dixisse, summa nobilitate illas sortes futuras.}

    \item \foreignlanguage{greek}{ΒΟΙΩΤΩΝ} \emph{Boetorum.} Apis, et Ceruus Ephesiæ Dianæ stemmata sunt. Eiusdem Deæ fanum in Boetia extitisse credibile est, ad eius quod Ephesi erat, exemplar constructum, ut in alijs permultis Græciæ civitatibus.

    \item \foreignlanguage{greek}{ΔΕΛΦΩΝ} \emph{Delphorum}, ut in nummo ante adnotato.
\end{enumerate}
\clearpage
\vspace*{\fill}
\begin{figure}[H]
\centering
\includegraphics[width=0.9\textwidth,keepaspectratio]{figures/22.png}
\end{figure}
\vspace*{\fill}
\clearpage
\subsection*{Tabula 5.}
\begin{enumerate}
    \item \foreignlanguage{greek}{EYKPITOY} \emph{Eucriti.} Stat Ceruus ad Palmam, eodem nomine, quo tertius prime Tabulæ Ephesius nummus inscriptus est. Aversa facies Apem in laurea corona præfert, in Pythijs Ephesij victoris index, siue argumentum est solemnium ipsius Deæ sacrorum. Refert quippe Macrobius ex Alexandro Aetolo, quanto studio populus Ephesius, dedicato templo, Dianæ curauerit, præmijs propositis, ut qui tunc erant Poëtæ ingeniosissimi, in Deam carmina diversa componerent.

    \item \foreignlanguage{greek}{MEΣΣANIΩN} \emph{Messanensium.} Lepus in plerisque Messanæ nummis impressus est, eiusdem Urbis animal insigne. Apis designat mellis præstantiam, qua plurimum Sicilia à Plinio commendatur. Rheda junctis Mulis acta, sacros ludos repræsentat, de quibus Pausanias. \emph{Inter rhedarum, et carpentorum aurigas tantum interest, quod his alia sunt insignia, et masculos equos agitant: rhedam trahebant Muli jugales bini, inuento neque eleganti, neque prisco.} Agesias victor curru Mularum in Olymipijs à Pindaro luadatur.

    \item \foreignlanguage{greek}{EΦ. XOPIΣKOΣ} \emph{Ephesiorum Choriseus.}

    \item \foreignlanguage{greek}{AΠTA} \emph{Apta.} Quid hoc nominis sit, aut quo referendum, pro certo affirmare non ausim; si tamen loci, aut civitatis alicuius nomen existimemus, non aliam quam Aptam Juliam deinde cognominatam Coloniam, in Provincia Narbonensi, non longe supra Massiliam, et Aquas Sextias sitam reperio. Cuius suspicionis causam dupplicem habeo; primum quòd civitates illæ Græcæ originis, nummos quoque suos Græcis literis inscripserint; tum verò quod omnis illa Galliæ, et Hispaniæ ora Dianæ Ephesiæ cultum à Massiliensibus Ionum Asiaticorum colonis acceperint, ut Strabo non uno loco indicauit. De hac Apta Julia Colonia à Julio Cæsare deducta in Provincia: vide Jacobum Sponium in suo eruditissimo opere, cui titulus est \emph{Miscellanea Eruditæ Antiquitatis}, duas afferens Inscriptiones eiusdem Colonie.

    \item \foreignlanguage{greek}{META}. \emph{Metapontinorum} ex celebri magnæ Græciæ civitate Metaponto, quos genus ab Heraclidis duxisse hic nummus indicat: in cuius larere averso legitur \foreignlanguage{greek}{HPAKΛEIΔΩN} \emph{Heraclidarum}. Sita fuit in ubere regione melle, et frugibus nobili, ut Apis, et Arista indicant. Caput galeatum alatum imago est Metabi Herois, huius Urbis conditoris, de quo Stephanus, et alij ad principium Velleij Paterculi.

    \item \foreignlanguage{greek}{EΦ. NIKOΛOXOY} \emph{Ephesiorum Nicolochi.}
\end{enumerate}
\clearpage
\vspace*{\fill}
\begin{figure}[H]
\centering
\includegraphics[width=0.9\textwidth,keepaspectratio]{figures/23.png}
\end{figure}
\vspace*{\fill}
\clearpage
\subsection*{Tabula 6.}
\begin{enumerate}
    \item \foreignlanguage{greek}{ΜΕ}. \emph{Megarensium} interpretandum opinior cum Paruta, qui hunc nummum inter alios Megarensium Siculorum exhibuit, in quibus itidem Pallas insculpta cernitur; Megara enim à Megarensibus Atticæ regionis nomen, et originem duxit, in quorum arce (ut Pausanias refert) templum, et signum Mineruæ extitit, ipsiusque delubrum Deæ Victoriæ cognomento. Hæc verò Siciliensis civitas Hybla, aut Megara Hyblensia vocata, de qua ad sequentem nummum.

    \item \foreignlanguage{greek}{ΜΕΓΑΛΑΣ ΥΒΛΑΣ} \emph{Magnæ Hyblæ.} Celebris est Hyblæi mellis præstantia; magnæ verò cognominis causam sic refert Pausanias: \emph{Fuere Hyblæ Siciliæ civitates duæ Gereatis cognomine una, altera maior. Retinent hac etiam nunc ætate prisca nomina, et earum altera in agro Catanensi planè deserta est: altera in ijsdem finibus, ad vici formam redacta. In hac fanum est Siculorum celebritate religiosam. Deæ quam Hybleam vocant, dicatum.} Deæ Hybleæ imago, sine dubio est, quæ in nummo conspicitur.

    \item \foreignlanguage{greek}{ΙΕΡΩΝΟΣ} \emph{Hieron.} Biga Pythiam, seu Olympiacam Hieronis designat victoriam, cuius etiam in 6 huius Tabulæ Syracusio nummo quadriga est nota. De curru Hieronis in Olympijs dedicato hæc memorat Pausanias: \emph{Sunt etiam sua de Olympicis victorijs Hieronis monimenta Dinomenis filij, qui fratri Geloni in Syracusanorum tyrannide successit. Hæc dona non sunt ab Hierone missa, sed votum Deo persoluit Dinomenes Hieronis filius. Currus Onatæ Aeginetæ; Calamidis, qui utrinque sunt equi, et equestres pueri opera sunt.} Pindarus verò Hieronem in Pythijs curru victorem celebrat ode 1 et 2 Pyth. ut in 6 huius tabulæ nummo. Fuit Hieron. curru victor Pythiade 29. Caput aristis, et arundine coronatum Arethusam repræsentat, juxta quam Apis est mellifluæ Siciliæ, ac Syracusiij mellis index.

    \item \foreignlanguage{greek}{ΕΦ}. \emph{Ephesiorum} \foreignlanguage{greek}{ΓΙΜΗΣΙΑΝΑ} \emph{Gimesiana}: considerandum, num forte\\ \foreignlanguage{greek}{ΓΑΛΛΗΣΙΑΝΑ} à Monte Gallesio, quem Epheso proximum Strabo commemorat.

    \item \foreignlanguage{greek}{ΤΑΥΡΟΜΕΝΙΤΑΝ} \emph{Tauromenitanorum}: Apollinis tripus: in aversa verò facie Apollinis Archagetæ caput laureatum: nam in alijs AP. legitur, cui Apis est appicta.

    \item \foreignlanguage{greek}{ΣΥΡΑΚΟΣΙΩΝ} \emph{Syracusanorum.} Arethusæ caput, ut in 3 huius Tabulæ nummo. Quadriga eiusdem Hieronis victoriam appingit in Syracusarum laudem, ut Pindarus Ode 2 Pyth. \emph{O magnis constitute Urbibus Syracusæ bello potentis Martis delubrum, virorum, equorumque ferro gaudentium felices nutrices. Vobis à præclaris Thebis hoc epinicij adferens venio. Quadrigæ terram concutientis denunciationem, per quam Hieron victoriam obtinens, procul fulgentibus coronis Ortygiam reliquit, sedem fluvialis Dianæ, citra cuius opem, illas habenis varias non domuie admirandis manibus suis equas.}
\end{enumerate}
\clearpage
\vspace*{\fill}
\begin{figure}[H]
\centering
\includegraphics[width=0.9\textwidth,keepaspectratio]{figures/24.png}
\end{figure}
\vspace*{\fill}
\clearpage
\subsection*{Tabula 7.}
\paragraph{}
\foreignlanguage{greek}{ΑΥΤ. ΚΑΙ. ΣΕΒ. ΑΝΤΩΝΕΙΝΟΣ}. \emph{Imperator Cæsar Augustus Antoninus.} Caput Antonini Pij laureatum. \foreignlanguage{greek}{ΣΥΡΙΩΝ}. \emph{Syrorum.} Capita Marci Aurelij, et uxoris Faustinæ inuicem adversa cum arista intermedià: infra appicta est Apis, et astrum. Supra caput Faustinæ, legitur: \foreignlanguage{greek}{ΚΟΡ}, idest \foreignlanguage{greek}{ΚΟΡΗ} \emph{Core, præstanti forma puella}, quo epitheto cognominata fuit Proserpina, et in hoc nummo à Syris per adulationem, Faustina, pulchritudine magis, quàm pudicitia conspicua. Supra caput Marci Aurelij duæ aliæ leguntur literæ \foreignlanguage{greek}{ΙΡ}. idest \foreignlanguage{greek}{ΙΡΟΣ} Ionicè pro ἱερὸς : \emph{Sacer, Sanctus, Divinus,} ob eius, ut inquit Capitolinus in omni vita, et à prima infantia sanctitatem. Cusus fuit hic nummus post Faustinæ cum Marco Aurelio nuptias à Syris, siue à Communi Syriæ; nam hæc regio Tetrapolis dicta, Strabone teste, quatuor ante alias habuit illustres Urbes à Seleuco conditas, Antiochiam, Seleuciam, Apameam, et Laodiceam, quæ propter ipsarum homoniam, et concordiam, Sorores appellabantur. Eodem sensu in nummo Traiani \foreignlanguage{greek}{ΚΟΙΝΟΝ ΣΥΡΙΑΣ} \emph{Commune Syriæ} ab ijsdem Syris cuso in honorem eiusdem Imperatoris cum capite Syriæ Deæ turrito. Hic perrarus Nummus mediæ magnitudinis in Reginæ Christinæ Augustæ Thesauro adservatur.

Apis sub Faustinæ, et sub M. Aurelij capite Stella insculpta est, spicà inter utrumque ipsorum medià, ubertatis, et feracitatis eiusdem regionis indices. Syria enim celi, solique amænitate commendatur, celebrisque præ cunctis ros est Syriacus, quem Apes ad mella conficienda ex rosis, et floribus legunt. De spica vero, et Apes siue de Ape Cereris ministra vide supra in Apis Hierogrammatismo, quæ ex Pindaro, et Pausania refert Menetreius.

Succedunt infra sex vetustæ Gemmæ anulares incisæ, Apum argumenta referentes, quæ in Barberina Dactyliotheca adservantur.
\begin{enumerate}
    \item Prima gemma Corniola aratrum exhibet, duabus Apibus jugalibus, tertia Coloni instar, vomeri insistit ad regimen cum ferula. Quo emblemate continentur tria humanæ vitæ bona; scilicet Sapientis regimen, Telluris ad victum cultura, et labor, ac mellis dulcedo, et recreatio, quæ tria bona sequenti disticho exprimuntur in Poematibus Urbani 8 sapientissimi Pontificis, clauo Ecclesiæ Apibus præsidibus, atque mellificantibus:
    
    \begin{quote}
    \emph{Supremum regimen, culte sata jugera terræ,}\\
    \hspace*{10mm}\emph{Mellis opus, tria sic tres potiora notant.}\\
    \end{quote}
    
    \item Apis Solis faciem radiatam referens, Solarem virtutem in perficiendo melle indicat, cui aptatur Virgilianum illud de Apibus ipsis: EXERCET SUB SOLE LABOR. Solis facies sic in orbem radiata insigne fuit Urbani Octavi literatissimi, ac splendidissimi Principis, ad cuius virtutem, et laudem hoc emblema componitur.

    \item LEONI, è cuius ore Apis egreditur, alludere videtur Samsonis ænigma DE FORTI EGRESSA EST DULCEDO. Alij tamen hoc sigillum inter veterum amuleta, ac Mithriaca Mysteria adnumerant.

    \item Apes tres ad truncum semper virentis Lauri aduolantes, atque mellificantes cum lemmate: HIC DOMUS. Carminum ipsius Urbani immortalitatem innuunt.

    \item Venator Amor Auem ramo insidentem arundine percutit, aduolante Ape, cui subijcimus DULCIS AMOR LAEDIT.

    \item Fibula erea antiqua Apis in effigiem formata.
    
    \begin{center}
    De Apibus.
    \end{center}
    
    \begin{quote}
    \emph{Idem amor, atque Apibus eadem experientia parcis,}\\
    \emph{Sed nec agros populare palam, aut incumbere furtis.}\\
    \emph{Cum iuuet è proprio vitam tolerare labore.}\\
    \emph{Justitiam norunt solæ, et servare pudorem:}\\
    \emph{Aequales cunctis operæ, studiumque parandi.}\\
    \emph{Mane novos adeunt flores examine denso,}\\
    \emph{Cœlestemque legunt rorem, atque in tecta reportant.}\\
    \emph{Mox vacuis stipant cellis, ut nectere largo}\\
    \emph{Ignauas ducant hyemes, et frigora temnant.}\\
    \end{quote}
    
    \emph{Ex Jou. Pontano.}
\end{enumerate}
\clearpage
\vspace*{\fill}
\begin{figure}[H]
\centering
\includegraphics[width=0.9\textwidth,keepaspectratio]{figures/25.png}
\end{figure}
\vspace*{\fill}
\clearpage
\subsection*{Tabula 8.}
\paragraph{}
Ad huius operis absolutionem, et elegantiam consentaneum duximus, additis Dianæ Ephesiæ Templo, et sacris, attexere quoq; votiuam Lucernam 1 eidem dicatam à quodam Eutyche Miletopolitarum Stratego, quæ Lucerna in nostro Museo vetustatis luce inextincta refulget. Ansæ apicem habet recuruum in modum falcatæ Lunæ cum titulo: \foreignlanguage{greek}{ΑΡΤΕΜΙΣ ΕΦΕΣΙΩΝ ΕΥΤΥΧΟΥΣ ΑΛΕΞΑΝΔΡΟΥ ΜΕΙΛΗΤΟΠΟΛΕΙΤΩΝ} \emph{Diana Ephesiorum Euthychis Alexandri Miletopolitarum.} Eutychem hunc Miletopolis Urbis Strategum fuisse indicat nummus Commodi 2 ab Erizzo vulgatus, qui ab uno latere exhibet Imperatoris caput cum litris: \foreignlanguage{greek}{ΑΥ. ΚΑΙ. Λ. ΑΥΡΗ. ΚΟΜΜΟΔΟΣ}. \emph{Imperator Cæsar Lucius Aurelius Commodus}; ab alio Mercurium saxo insidentem cum Caduceo; ijdem Eutychis nominibus, in Lucerna incisis, ac Strategi dignitatem referentibus: \foreignlanguage{greek}{ΕΠΙ. ΣΤ. ΕΥΤΥΧΟΥΣΑΛΕΞΑΝΔΡΟΥ ΜΕΙΛΗΤΟΠΟΛΕΙΤΩΝ} \emph{Sub Stratego Eutiche Alexandri Miletopalitarum.} Sic vertendum sanè, haud quaquàm cum Erizzo ipso: \foreignlanguage{greek}{ΕΥΤΥΧΟΥΣ ΑΛΕΞΑΝΔΡΟΥ} \emph{Felicis Alexandri.} In quem errorem fortè non incidisset accuratissimus Nummographus, si Strategi dignitas primoribus literis patuisset. Extat et alius eiusdem Urbis nummus mediocris magnitudinis, in honorem Gordiani Pij à Clarissimo Equite Patino editus 3. \foreignlanguage{greek}{ΑΥ. Κ. Μ. ΑΝ. ΓΟΡΔΙΑΝΟΣ}. \emph{Imperator Cæsar Marcus Antoninus Gordianus} Gordiani caput. In latere adverso Diana Venatrix conspicitur cum epigraphe \foreignlanguage{greek}{ΕΠΙ. Μ. ΤΡ. ΑΥΡ. ΕΡΜΟΥ ΜΕΙΛΗΤΟΠΟΛΕΙΤΩΝ} \emph{Sub Marco Traiano Aurelio Herme, seu Hermete Miletopolitarum}; ex quibus constat, huius Urbis studium, et cultum insignem fuisse in Dianam Ephesiam. Stephanus Miletopolim scribit Urbem esse inter Cyzicum, et Bithyniam, circa Rhyndacum fluvium, deque ipsa Strabo, et Plinius.
\clearpage
\vspace*{\fill}
\begin{figure}[H]
\centering
\includegraphics[width=0.9\textwidth,keepaspectratio]{figures/26.png}
\end{figure}
\vspace*{\fill}
\clearpage
\section{Jo. Petri Bellorii Expositio Symbolici Deæ Syriæ Simulacri.}
\paragraph{}
Apum argumentum vocat nos ad Syriæ Deæ, seu rerum Naturæ simulacru, quod ob oculos ponitur juxta Syrorum doctrinam, ac superstitionem; nosque eò libentiùs ipsum meditamur, quòd hæc Dea cognationem habet cum Ephesia Diana; utraque enim plurimis mysterijs, et symbolis spectabilis est, ac præcipuè Apum insignijs, quibus ambæ decorantur. Simulacrum ipsum ex aere ductum apud Virginium Ursinum Anguillaræ Comitem olim visebatur, eiusque imaginem ad magnitudinem, ac similitudinem, quæ habetur in Pyrrhi Ligorij Codicibus, in Bibliotheca Augustæ Reginæ Christinæ adservatis, desumpsimus. Titulus in basi legitur.
\begin{quote}
MATER. DEOR. ET. MATER. SYRIAE
\end{quote}
\vspace*{-4mm}
\paragraph{}
Matrem Deûm Cybelem, et Deam Syriam unum, idemque numen esse plurima indicant argumenta in hoc simulacro appicta; muralis scilicet, ac radiata Corona in capite, Tympanum, et Colus in sinistra manu; Leones præterea, qui Deæ montano saxo insidenti adstant. Horum indiciorum locuples, et oculatus testis est Lucianus celeberrimi Templi eiusdem Deæ, quod in Syria erat, conditorem requirens; hæc enim refert in Dialogo de Dea Syria: \emph{et multa signa adsunt, quæ ipsam Rheam videri faciunt; nam et Leones ipsam ferunt, ac tympanum habet, et coronam in capite turritam gestat, qualem et Lydi Rheam effingunt.} De colo, sceptro, et radijs paulò inferiùs. \emph{Alterâ quidem manu sceptrum tenet, alterâ colum, et in capite radios gerit, ac turrim.} Coronam turritam Cybeli peculiarem ideò tributam asserunt, quòd prima Urbes muris construxerit, ac munierit, ut inquirit Ovidius vulgatis illis carminibus:
\begin{quote}
\emph{At cur turriferâ caput est onerata coronâ?}\\
\emph{An primis turres Urbibus ipsa dedit?}\\
\end{quote}
Altera corona radiata, quæ ipsius caput eingit, refertur ad Solarem virtutem, quam Sol benefico lumine suo Cybeli, seu Terræ impartitur ad producendas fruges, omnesq; res naturales, quorum Sol nuncupatur auctor, et parens, ideoq; à Macrobio dicitur fons celestis lucis, mens, et temperatio Mundi.

In vertice huius nostri simulacri eminet Cydaris, seu Mitra eius caput exornans, adinstar falcatæ Lunæ, quæ, ut in Scipionis somnio scribit Cicero, in infimo orbe constituta, radijs Solis accensa convertitur. De temperie caloris, humorisque ab his duobus astris ad fecunditatem terræ, humanique generis beneficentiam emanante insignis est locus Senecæ libro de Beneficijs: \emph{Num dubium est, quin hoc humani generis domicilium circuitu Solis, et Lunæ vicibus suis temperetur? quin alterius calore alantur corpora, terræ relaxentur, immodici humores comprimantur; alligantis omnia Hyemis tristitia frangatur, alterius tepore efficaci, et pentrabili rigetur maturitas frugum? quin ad huius cursum fœcunditas humana respondeat? quin ille annum observabilem fecerit circumactu suo, hæc mensem minoribus se spatijs flectens?} Hoc idem sensisse videtur huius simulacri auctor per duodena Zodiaci signa depicta in fascia illa, quæ pallio superimposita, è collo utrinq; dependet ad pectus, et inde reiecta à sinistro brachio diffunditur, fœcunditatem, atque ubertatem terræ designans à Solis et Lune circuitu provenientem; Sol enim per anni tempora Signiferum percurrens quadrifariam dividit: urgente Bruma frugum semina circumcluso calore fouet; adueniente Vere, eadem in frondes, et flores educit, donec Aestiuo, atque Autumnali tempore, fructus omnes ad maturitatem, et perfectionem perducat. Eandem genitalem vim innuere videtur Lunula cum Astro appicta in eadem Mitrà, stellarum influxus in hæc inferiora significans; nisi potiùs Astrum illud Solem ipsum referat, ad demonstrandam horum, ut putabant, æternitatem; nam Aegyptij Sacerdotes æuum innuentes, teste Horo, Solem, et Lunam pingebant. Ipsam autem Deæ Syriæ Lunatam Mitram comprobat vetus monumentum à Clariss. Seldeno in Syntag. de Dijs Syris relatum; in quo Dea sedet inter Leones duos Tiara, seù Diademate Lunato ornata, cui subiungitur inscriptio.
\begin{center}
P. ACILIUS. FELIX\\
D. D. DIA. SVRIAE\\
CUM. SUIS\\
\end{center}
Communionem ampliùs, quam Cybeles habet cum Iside, seu Luna, satìs indicat Sistrum, quod nostrum Simulacrum sinistra manu gerit cum alijs symbolis; atque omnibus palam est, Matrem Magnam, Rheam, Cybelem, Opim, Cererem, Proserpinam, Isidem, et Lunam, quamuìs diversa nomina, in re tamen unum fuisse numen, imò unam, eamdemque rerum Naturam ex Aegyptiorum placitis ac mysterijs. De Iside verò hæc profert Macrobius: \emph{Isis cunctâ religione colebatur, quæ est terra, vel natura subiacens Soli.} Martianus Capella de Lunari globo loquens: \emph{in eo Sistra Niliaca, Eleusinaque lampas; arcus Dyctinnes, tympanaque Cybeles.} Sistri formam describit Apuleius As. Aur. simulacrum Isidis, seu Lunæ appingens: \emph{nam dexterâ quidem ferebat æreum crepitaculum, cuius per angustam laminam in modum balthei recuruatam, traiectæ mediæ paruæ virgulæ, crispante brachio, tergeminos ictus reddebat argutum sonum.} Plutarchus in suo Opusculo de Iside, et Osiride rationes naturales reddit de earumdem virgularum ictibus, et concussione, per quas significari scribit sub Luna cuncta moveri, generari, et corrumpi, et per quatuor virgulas quatuor elementa: \emph{Sistri verò desuper orbicularis prospectus continet quatuor}, scilicet virgulas, \emph{quæ concutiuntur, etenim pars mundi quæ generatur, et corrumpitur, comprehenditur quidem à lunari Sphæra; moventur autem in ea omnia, et mutantur per quatuor elementa, Ignem, Terram, Aquam, et Aërem.} Alij per hoc crepitaculum intelligebant Nili incrementum, Lunæque assiduas vertigines, et conversiones, quibus in cælo movetur. At virgulæ quatuor, quas Plutarchus intelligit ad numerum Elementorum, non respondent nostro Deæ Syriæ simulacro, cuius Sistrum tribus tantùm virgulis conficitur, totidemque numerantur in aliquibus alijs imaginibus, ac præcipuè in supra exhibitis Isidis Phariæ sistratæ ex sententia illorum fortè, qui ignem inter elementa non admittunt ad principia rerum naturalium, supplerique autumant calore Solis. Cum, autem de Sistro sermo sit, non abs re fuerit eius formam ob oculos ponere ex æreo illo veteri, quod in nostro Musæo asseruatur, cuius principalem, ac lateralem faciem ad integram similitudinem hìc exhibemus. Præter illas igitur quatuor virgulas descriptas, in eiusdem Sistri summitate. Felis decumbens fœtibus lactantibus insculpta est, atque in eius ipsius vertice Lunæ globus eminet, connexionem, quam hoc animal habet cum Luna, indicans, juxta descriptionem Plutarchi, eodem loco: \emph{Sistri apssdi in vertice Felem cum vultu humano insculpunt. Infra sub ijs, quæ torquentur, hìc Isidis, illic Nepthydis effigiem. Innuunt his vultibus ortum, et obitum, sic enim elementorum sunt vices, et motus. Fele Lunam ob varietatem laborum nocturnorum, et fœcunditatem huius animalis; siquidem hoc ferunt gignere unum, inde duo, tria, quatuor, quinque, atque ita singulatim ad septem usque adijciunt, ut octo et viginti universa pariat, quæ Lunæ sunt lumina. Atque hoc fortassè sit fabulosius. At in oculis suis pupillæ in plenilunio compleri, atque explicari videntur, contra deducii, inopacarique Luna senescente. Facie humanâ Felis intellectus, et ratio repræsentatur voluminum Lunarium.} Hoc ultimum, symbolum in nostro Sistro non convenit cum descriptione Plutarchi; nam Felis humana facie non est prædita, sed talem habet vultum, qualem omnibus suæ speciei animantibus natura impertita est. Non multis ab hinc annis visebantur Rome Sistra duo, in quoru utroque Felis efficta erat humano ore conspicua; alterum apud Bartholomæum Giulit Nobilem Gallum, alterum in celebri Musæo Eq. Francisci Gualdi, quorum icones perhumaniter, ut soler, harum elegantiorum decus, è suo ditissimo penu exhibuit nobis Illustriss. D. Carolus Antonius à Puteo Sancti Stephani Eques Commendatarius, è cuius descriptione, post allata Plutarchi verba, hæc pauca excerpsimus.

\emph{Quæ descriptio Plutarchi ad unguem cuidam Sistro antiquo, cuius copiam mibi fecit amantissimus, huiusque disciplinæ peritissimus Bartholomæus Giulit vir nobilis Gallus, respondet. In vertice enim instar lusorij reticuli sinuatur, quatuorque ferreas virgulas infixas habet, quæ huc, illuc elabentes sonitum quemdam edebant: ne verò excussæ in terram deciderent, ad uniuscuiusque virgulæ initium, eodem ferro efficta erant utrinque Anserum siue Anguium capita inuicem se aspectantium. In vertice Felis humano capite cælata visebatur, postea verò erat manubrium, ut commodius Sistrum manibus contineri posset. Quam secundo loco iconem damus, erat apud Eq. Franciscum Gualdum, solasque tres ferreas virgulas habebat, quaru media immobilis erat, in reliquis superiori rospondebat.} Quod verò Sistrum exhibemus, è Gazophylacio Cardinalis Virginij Ursini nuper erepti nostris Cimelijs accessit. Sed et aliud erutu modò est Via Aurelia in Suburbana Villa Illustriss. D. Laurentij de Corsinis, ab eodem dono datum D. Leoni Strozzæ, avitâ nobilitate, moribus, optimorumque studiorum cultu in florenti ætate inter Romanos Proceres spectando, in cuius splendidissimo Musæo asservatur, nostro sanè persimile si tantummodò Lunarem globum in Felis vertice addas. Nephthydis, et Isidis effigies, quas Plutarchus infra virgulas describit, ortus et obitus indices, non adhuc inspicere contigit, sed in utroq; Sistri latere Isidis tutulum, seu Mitram Lunari globo insignem.

Sed cum de Sistro abundè locuti simus, reliqua symbola modò percurramus. Sistrum sequitur Colus, quem hæc Dea una simul manu gerit, connexionem indicans, quam ipsa habet cum Fato, et Parcis, quæ nent, et recidunt humanæ vitæ stamina. Sors enim et Fatum Lachesis nomini inest; et hoc intelligitur in nostro simulacro, juxta Luciani descriptionem supra allatam: \emph{alterâ quidem manu sceptrum tenet, alterâ colum.} Nam Dea Syria, præter illa plurima nomina, quibus insignitur, Proserpina quoquè vocabatur, eique Parcas subesse, eiusque imperio parere dictitabant, ut apud Claudianum sic raptam Proserpinam Pluto solatur:
\begin{quote}
\emph{Accipe letbæo famulas cum gurgite Parcas,}\\
\emph{Sit fatum quodcunque velis.}\\
\end{quote}
\vspace*{-8mm}
\paragraph{}
Quarè Proserpinam habuisse communem ædem cum Parcis, ac matre Cerere in Atticis scribit Pausanias; ideò fortè Dea Cybeles, etiam Proserpina fuit cognominata, quòd Physici, teste Macrobio inferius hemispherium terræ Proserpinam vocauerunt.

Post Colum visitur Caduceum, quo aër designatur, et Mercurius; nam cum euis sidus velocissimum sit, ideò hic Deus alatus fingebatur, ipsumque per aëra volitare, atque ire, et redire ad superûm, Inferorumque domus putabant. Cum verò humectans, et calidus aër sit; calore igni, et humore aquæ consociatur; quæ duo Blementorum, Naturæ que dissonas qualitates, ut placet in Timeo Platoni, perfectionem huius mundanæ molis insolubilem conciliant atque deuinciunt. Caduceum quoquè benè jungitur stamini, et colo humanæ vitæ; pertinet enim ad genituram hominum, teste eodem Macrobio: \emph{Argumentum Caducei ad genituram quoquè hominum, quæ Genesis appellatur, Aegyptij portendunt.}
\clearpage
\vspace*{\fill}
\begin{figure}[H]
\centering
\includegraphics[width=0.9\textwidth,keepaspectratio]{figures/27.png}
\end{figure}
\vspace*{\fill}
\clearpage
\paragraph{}
Flagellum præterea, quod Syria Dea pro sceptro manu gerit, usitatissimum est in Aegyptijs Deorum simulacris, ipsamque depulsare mala, atque adversantibus terrorem incutere indicat. Erant porrò aliqui Dij propitiatores, ac defensores ἀλεξίκακοι, atque Averrunci noncupati, ijsquè Typhoniam vim neque resistere, neque noxam inferre posse arbitrabantur.

De Leonibus Cybeles, qui Deam concomitantur, nota res est apud Ovidium in Fastis, Hippomenem, atque Atalantam ipsius irâ Deæ in has feras conversas describentem. Referunt alij fabulam ad rationes naturales, et præcipuè Macrobius sæpè à nobis allatus in Saturnalibus: \emph{Quis enim ambigut, Matrem Deûm terram haberi? Hæc Dea Leonibus vebitur validis impetu, atque firuore animalibus; quæ natura cæliest, cuius ambitu aër continetur, qui vehit terram.} Aliam affert causam Lucretius, adiunctos asserens bijuges Leones currui ipsius Deæ ob emollitam hominum feritatem:
\begin{quote}
\emph{Adiunxere feras, quia quamuis efferra proles,}\\
\emph{Officijs debet molliri victa parentum.}\\
\end{quote}
\vspace*{-8mm}
\paragraph{}
Cui assentitur Ovidius in Fastis:
\begin{quote}
\emph{... feritas mollita per illam}\\
\emph{Creditur: id curru testificata suo est.}\\
\end{quote}
\vspace*{-8mm}
\paragraph{}
Placuit tamen alijs, Leones mansuetos Matri Magnæ adiunctos, ut indicarent, nullum esse terræ genus tàm asperum, atque ferum, quod non subigi, colique possit. De qua re plura Menetreius in Hierogrammatismo Leonum. Nostrum verò simulacrum non curru à Leonibus vectam Deam repræsentat, sed quod rarius est, ipsam insidentem montano saxo, ijsdem feris ad eius pedes, atque eius lateri adhærentibus. Cybele porrò à montibus, in quibus erat culta, diversa nomina sumebat, dictaque fuit Berecynthia, Idea, et Dindymene à Cybele, Berecyntho, Ida, et Dindymo, quin etiam Montana Petrosa, Montigena, et Agdiste fuit cognominata, quòd in finibus Phrygiæ ex monte, quem indigene Agdum appellant, divinitus fuerit informata teste Arnobio.

Rosæ in pectoris zona, atque in clamyde intextæ, Matris Deûm, seù Telluris florenti Vere primitias ostentant. Quare in eius festo, cum Dea per Urbes vectaretur, ante ipsius simulacrum Rosæ spargebantur, juxtà eiusdem Lucretij descriptionem.
\begin{quote}
\emph{Ergo cùm primùm magnas inuecta per Urbeis}\\
\emph{Munificat tacitâ mortaleis muta salute:}\\
\emph{Aere, atque argento sternunt iter omne viarum,}\\
\emph{Largificâ stipe ditantes, ninguntque rosarum}\\
\emph{Floribus, umbrantes Matrem, comitumque cateruas.}\\
\end{quote}
\vspace*{-8mm}
\paragraph{}
Quod idem solemne fuisse in Isidis, siue ipsius Deûm Matris pompâ refert Apuleius, in quâ Sacerdos ante Deæ simulacrum, coronam amænis contextam rosis gerebat, quæ Lucium ferinâ specie exuerunt. Sed præclarius argumentum præbent Apes, quæ in extrema clamydis ora, rosis intermixtæ limbum vicissim pingunt, et venustè quidem, nam et ipsæ florilegæ rosas amant, ac in hortis cum thymo, et violis, ad mellis munera depascunt. Quin et Apes Matri Deûm ità sunt addictæ, ut Cybeleio æris tinnitu mulceantur, ac abeuntes revocentur, ut describit Virgilius:
\begin{quote}
\emph{Tinnituque cie, et Matris quate Cymbala circum,}\\
\emph{Ipse confident medicatis sedibus ipsæ}\\
\emph{Intima more suo se se in cunabula condent.}\\
\end{quote}
\vspace*{-8mm}
\paragraph{}
Simul ergo Apes, et rosæ Deæ Syriæ, et Ephesiæ Multimammiæ dicabantur, et utriusque simulacrum exornant intermixtæ; de quibus consule Menetreium in ipsarum Hierogrammatismo.

De corporis cultu, atque indumentis, quibus hoc Deæ simulacrum insignitur, Cydari nempè, et pallio, ac fascia fimbriata, quæ è collo utrinq; dependet, an ab his emanarint nostrorum Antistitum vestes, Mitra, Planeta, seu Pluviale, ac Stola, videant peritiores. Velo, quod è collo dependet, ac Deæ caput obnubit, noctis umbræ indicantur, quarum Diana est moderatrix, Noctiuaga, et Noctiluca cognominata.

Restat ut reliqua symbola, quæ dextero brachio insculpta hærent, paucis perquiramus. Quid porrò Arcus, Pharetra, atque ignita fax? Non ne his congruunt illa superiùs allata verba Martiani Capelle de Lunari globo loquentis? \emph{In eo Sistra Niliaca, Eleusinaq; lampas, Arcus Dictynnæ, tympanaq; Cybeles.} Quibus verbis cum Luna coniunguntur Isis, Ceres, Diana, ac Cybele, nisi malis pharetram, Arcum, et facem ad ipsam tantùm Dianam pertinere, eamdem Lucinam, et Venatricem nuncupatam.

Ad Cupidinem quoquè, et Venerem hæc tria spectare, non sine ratione quis dicat ad significandam Amoris eiusque Matris insitam vim, quæ cuncta movet in omnibus animantibus, ac rebus naturalibus procreandis, juxta Lucretium in principio operis sui de Rerum Natura, qui sic Venerem inuocat, Amoris flammam pectoribus incutientem.
\begin{quote}
\emph{Denique per Maria, ac Monteis, fluviosque rapaceis,}\\
\emph{Frondiferasque domos auium, camposque virenteis,}\\
\emph{Omnibus incutiens blandum per pectora amorem,}\\
\emph{Efficis ut cupidè generatim sæcla propagent.}\\
\end{quote}
\vspace*{-8mm}
\paragraph{}
Faci proxima harpago, siue uncus appictus est ad denotandam rerum naturalium connexionem, et seriem, nisi malis Gnomonem esse, et regulam ipsarum rerum in hoc mundo concinno ordine dispositarum. Vides inde humero, brachioque, et manui ipsius Deæ hærentia animalia, quæ in terra, atqu in aëre nascuntur: Aprum Martis, Leonem Solis, ac si rectè conijcimus, Coruum Apollinis, Columbam Veneris; hisque subiacentem. Scarabæum consuetum apud Aegyptios Solis symbolum, et amuletum. Inter verò Reptilia in dextera manu, ac digitis conspicitur Lacertus Veris, et redeuntis anni index, item Cochlea, et Papilio: illa ex putri in vitam educta; hæc ex semine perpetuò renascens, immortalitatis, vitæque resurgentis typus. In volâ dexteræ aspice pomum ubertatis, ab ipsa proueniens Deæ fecunditate. Manus verò sic extensa, atque aperta est ad maiestatem, et beneficentiam, supra quam fulmen extremis digitis imminens, nil aliud significare videtur quàm supremi Numinis prouidentiam, ac divinitatem, cuius utriusq; typus est fulmen, ad regimen huius mundanæ molis.

Templum denique cum turrita coronâ in capite eminens, illud esse censemus donarijs, sacris, opibus, et magnificentia Hierapoli celeberrimum, in quo Dea veneratione summâ colebatur. De eius multiplici specie, ac nominibus intellexisse videtur Apuleius, cum per somnium ità loquens se Lucio ipsa palâm exhibuit: \emph{Cuius Numen unicum multiformi specie, ritu vario, nomine multijugo totus veneratur Orbis. Inde primigeni Phryges Pessinunti eam nominant Deûm Matrem, hinc Autochthones Attici Cecropiam Minerva, illinc fluctuantes Cyprij Paphiam Venerem; Cretes saggittiferi Dictynnam Dianam; Siculi trilingues Stygiam Proserpinam; Eleusini vetustam Deam Cererem; Junonem alij, alij Bellonam, alij Hecaten, Rhamnusiam aij, etc.}

His ad explicationem huius mystici simulacri pro nostra tenuitate expositis, si quis detrahat Ligorio fidem in antiquis monumentis, quasi mendacio operam nostram eluserit, sciat is nos de Ligorio unâ cum Clariss. Spanhemio ità sentire, ut illum in vetustis reliquijs colligendis, delineandis, describendis summâ laude dignum existimemus. Siquidem quadraginta, et plura volumina digessit, quæ in Taurinensi Bibliothecâ, et alibi adservantur. Quòd si in tanta rerum mole aliqua minùs fide digna reperiantur, cuncta abijcere, atque euertere magnum Litterarum, ac Rei antiquariæ foret detrimentum. In Simulacro ipso adeò omnia respondent mysteria, ac symbola rebus naturalibus, fabulis, et ipsi Philosophiæ consona, ut puram redoleant antiquitatem vel suprà ipsius Ligorij eruditionem. Quare, ne quid desit, aut desideretur, eius interpretationem huic norstræ adijcimus.
\clearpage
\section{Excerpta ex Pyrrhi Ligorij interpretatione Simulacrí Deæ Syriæ Italico sermone descriptâ.}
\begin{center}
\textbf{\emph{D'Iside, et Matre Syria natura generante etc.}}
\end{center}
\paragraph{}
\emph{Alcuni Filosophi significarono in lei l'universale, il Cielo, et la Terra; fecero la figura sua con tutti i simboli delle ipse, che il ritratto di presente hà dimostrato, secondo quella statuetta antica di bronzo, che haueua il Sig. Virginio Ursino Conte dell' Anguillara, che nelle parti più alte nell' estremità ha figurato nella testa le cose del Cielo, la Mitra, et un altra fabrica Episcopale, et tutelare di ogni suo gouerno. Nella cima della mano destra tiene il fulmiue per significato dell' aere, et delli fulmini che intuonano strepito samente nell' aere. Intorno il mantello hà li Segni Celesti de' Planeti, d'onde Planeta è detto il militare Piuiale de' nostri Christiani Episcopi, che hanno, come militari Pastori il gouerno delle' sue gregi, secondo la celeste regione, et fermissima fede. Poiche questo habito è stato ritratto dalla Dea Syria, et conservato dalla Santa Chiesa, perche in esso si riconosce quell' Altissimo Dio, che hà fatto la Natura.}

\emph{In essa sono la Luna, il Sole, le Stelle, le celesti torri, che circondano la furtezza del celesto reggimento, oue sono i simboli de' Planeti. Nella mano destra, come principale nelle potentie, che gli Astronomi hanno forniato ciascun Planeta, il fulmine di Gioue, la facella del Sole, il carcasso, e l'arco della Luna, la Colomba di Venere, la Falce di Saturno, il Coruo di Apollo, il Cinghiale di Marte. Nell' altre diversità degli effetti delle Stagioni, et cose terrestri ui sono le Farfalle, le Lucerte, li Pomi, la Limaca, et per tutto il manto li Fiori. Nella mano sinistra appoggiata su'l Crepitabulo, il Timpano per li Venti, il Sistro per le Stagioni, li Sonagli per lo strepito dell' aere, il Caduceo di Mercurio per li contrasti, per la pace, per lo parlare, per le congiuntioni indissoluili, per lo metro delle cose che passano per lo Cielo, et per l'Aria. Il fuso delle Parche lo stame della vita, li Pesci per lo Mare, i Leoni per gli Animali, et ferocità di essi, et per la potenza del Sole. Il solio oue siede è sopra un monte per li Monti, et Olimpi della grandezza terrena. Li Segni che hà la Dea intorno il Planeta, ò Mantello, che è figura del coperto, ò padiglione celeste, che copre ogni cosa aerea, e terrena. Alcuni hanno fatto ad essa Matre Syria le Parche istesse per sua cintura, et le Hore intorno il petto, et altri una collana di mammelle, et secondo scriue Luciano, hauer veduto la Matre Syria scolpita in significato di tutti gl' Iddij, e tutte le Iddee, le Muse et celesti Planeti, la Vittoria, la Fortuna, le Parche, il Sole, la Luna particolarmente, et esser Pallade, Marte, Bellona, Venere, di cui hà il cesto di Rose etc.}
\end{document}
